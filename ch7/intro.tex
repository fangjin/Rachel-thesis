%As social media (e.g., Twitter) continues to increase in popularity, they are becoming mainstream venues to stage mass movements, such as protests and uprisings.

As social media (e.g., Twitter) continues to increase in popularity, it is becoming employed
as a social sensor into real-world mass movement event detection. Modeling and studying their adoption patterns gives us insight into investigating social and physical aspects of those events and precursors. This dissertation has presented several approaches and strategies with the goal of detecting and predicting mass movements and further inferring its causality, with given information mixed with real news and rumors. Those include techniques to capture information propagation across multiple spaces, as well as a graph wavelet approach that broadens predictive capabilities to capture group abnormality within dynamic changing networks. Numerous forms of mass movements have been investigated and diverse aspects of modeling and detecting have been addressed.

Using social media as indicators for real-word event detection is indeed helpful tool, however, they do possess limitations, perhaps most notably when applied to a specific event type, such as mass movement studied here. First, modeling protest-related topic propagation on networks is never trivial. One challenge is social protest propagation through online media can spread over large areas more quickly than traditional methods since users are geographically distributed, the other challenges include mass protest information can be spread by multiple social medias and lot of paths, like word of mouth, TV and news broadcast. Second, detect the group abnormality on social media is challenging. One challenge is Twitter's user network embodies many subgraphs based on social ties which is dynamically changing the graph structures since users are active. The other challenge includes real world events are not only correlated with burst signals, but can also exhibit unusually low levels of activity in social networks.
%Thereby how to define abnormality and capture those abnormality within dynamic changing networks?
Despite these restrictions, graph wavelet have in fact provided powerful capacity in capturing graph abnormality (considering burst behavior and absenteeism behavior), even on dynamic changing networks.


A fundamental objective of this dissertation has been to model mass movement adoption behavior, and in doing so, several significant advantages are gained beyond the target. One contribution is the ability to model information propagation across multiple networks/spaces, and capture the propagation speed and possible propagation paths, which is demonstrated in Chapter 2. Another benefit that enhanced the mass movement detection is though group abnormality detection, as introduced in chapter 3. Graph wavelet provides appropriate definition of group abnormality which can cover both burst and absenteeism with different scales, thereby increasing the probability of capturing protest behaviors. Another by-product is the capability of quantify compartment transition dynamics using epidemic model SEIZ, and facilitate the development of screening criteria for distinguishing real movements from rumors happenings on Twitter, as demonstrated in Chapter 5.

Understanding information propagation over dynamic social network is highly-popular for addressing real-world problems in social network analysis. This dissertation analyzes several fundamental questions underlie the propagation-like processes, such as mass movement adoption, rumors transmission. These methodologies can be extended to other applications such as infectious diseases, public health, marketing, and so on. Given current advancements in information propagation, this is a revolutionary time for research in real-world applications using social network analysis.





% comments from prilim
%\begin{itemize}
%\item North: lessons learned are?
%\item North: what does this say about the real phenomenon?
%\item North: help us learn about the semantics of the phenomenon
%\item North: why not detect rumors in other ways? Why not look at the source?
%\item CT: What's the relationship between rumors and absenteeism.. Spatial granularity can be common and can be harnessed. relationship between community and group abnormality,
%\item CT: climate keywords. Static keywords or dynamic keywords?
%\item Yang: why did u choose these particular techniques? no consistent theme across
%\item North: replace "4 problems in.." with a better title
%\item Feng: a treemap to connect the 4 topics
%\end{itemize}
