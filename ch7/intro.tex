%As social media (e.g., Twitter) continues to increase in popularity, they are becoming mainstream venues to stage mass movements, such as protests and uprisings.
\section{Contributions}
As social media (e.g., Twitter) continues to increase in popularity, it is becoming employed
as a social sensor into real-world mass movement event detection. Modeling and studying their adoption patterns gives us insight into investigating social and physical aspects of those events and precursors. This dissertation has presented several approaches and strategies with the goal of detecting and predicting mass movements and further inferring its causality, with given information mixed with real news and rumors. Those include techniques to capture information propagation across multiple spaces, as well as a graph wavelet approach that broadens predictive capabilities to capture group anomaly within dynamic changing networks. Numerous forms of mass movements have been investigated and diverse aspects of modeling and detecting have been addressed.

Using social media as indicators for real-word event detection is indeed helpful tool, however, they do possess limitations, perhaps most notably when applied to a specific event type, such as mass movement studied here. First, modeling protest-related topic propagation on networks is never trivial. One challenge is social protest propagation through online media can spread over large areas more quickly than traditional methods since users are geographically distributed, the other challenges include mass protest information can be spread by multiple social medias and lot of paths, like word of mouth, TV and news broadcast. Second, detect the group abnormality on social media is challenging. One challenge is Twitter's user network embodies many subgraphs based on social ties which is dynamically changing the graph structures since users are active. The other challenge includes real world events are not only correlated with burst signals, but can also exhibit unusually low levels of activity in social networks.
%Thereby how to define abnormality and capture those abnormality within dynamic changing networks?
Despite these restrictions, graph wavelet have in fact provided powerful capacity in capturing graph abnormality (considering burst behavior and absenteeism behavior), even on dynamic changing networks.


A fundamental objective of this dissertation has been to model mass movement adoption behavior, and in doing so, several significant advantages are gained beyond the target. One contribution is the ability to model information propagation across multiple networks/spaces, and capture the propagation speed and possible propagation paths, which is demonstrated in Chapter~\ref{ch:GBM}. Another benefit that enhanced the mass movement detection is though group anomaly detection, as introduced in chapter~\ref{ch:absenteeism}. Graph wavelet provides appropriate definition of group anomaly which can cover both burst and absenteeism with different scales, thereby increasing the probability of capturing protest behaviors. Another benefit is the capability of quantify compartment transition dynamics using epidemic model SEIZ, and facilitate the development of screening criteria for distinguishing real movements from rumors happenings on Twitter, as demonstrated in Chapter~\ref{ch:rumor}.

Understanding information propagation over dynamic social network is highly-popular for addressing real-world problems in social network analysis. This dissertation analyzes several fundamental questions underlie the propagation-like processes, such as mass movement adoption, rumors transmission. These methodologies can be extended to other applications such as infectious diseases, public health, marketing, and so on.

\section{Future Directions}
One of the major attractive areas would be continuing focusing on social network analysis, specifically the information propagation research over dynamic changing social network. Thereby the future research directions will fall into two categories, one is to deepen the existing theory and algorithm, the other is to broaden the current research.

\paragraph{Extend GBM model}
What would happen to the geometric Brownian motion model if the underlying mention network changes over time? How to adopt or modify this model when apply into multiple networks? As well as those theoretical questions, there are also some applications worth further investigation, such as, can we introduce the GBM model into infectious disease domain, for example, zika virus spreading? Assume the Bispace is composed with connection network and the other is physical space, can we train the GBM model to estimate each use's infection probability based on their environment?


\paragraph{Further study graph wavelet}
We hope to extend the graph wavelet applications into other areas, based on the two distinguish properties of graph wavelet. One is the ability to detect graph abnormality, such attribute can be adopted into detecting the wealth gap between rich and poor in one region, identifying the brain neural network abnormality, or detecting traffic congestion through road network analysis; the other property is the ability to identify the central point of a subgraph, which can be employed to rank key players over networks, detect the rumor spreaders in some cascade, or find the source of infection as per certain disease.

\paragraph{Broaden rumor detection scenarios}
In stead of predict a story is true or false, it is more practicable to label how much people tend to believe it. Newspapers would find it very useful, especially when it comes to some breathtaking news yet not being confirmed. Before reporting to the public, they would like to grasp how much the story is believable. Also it is valuable as to decide whether vendors are cheating during online shopping.


\paragraph{Deep understand personalized information propagation.}
We would like to understand how users' behavior lead to the information propagation delay or boost, especially when accompanied with strong sentiment. This may help formulate advisement strategy, if we can find a way to manipulate the information flow. Further, we would like to explore opportunities to extract personalized information spreading pattern. What kind of news may arouse his/her interest, if so, what kind of role he/she may play, what kind of push strategy may stimulate his/her activity? This study is propitious for precision marketing or personalized recommendations, provided refined content filtering.

\paragraph{Build an intelligent disaster detection system}
We would like to build an intelligent system which is efficient at event detection, especially for some diasters, protests, extraordinary events. It cannot only do immediate reporting, but also able to track events and do causality analysis and even do future prediction.
The system has some critical building blocks: natural disaster detection using graph wavelet by detecting group anomaly, rumor or news detection by employing epidemic SEIZ model, story causality analysis, and event coding.
Take the flint water crisis for example, firstly it will identify the water crisis events from social media analysis, then confirm it is true story, next, it will trace all the historical news to identify the causality, finally, it will generate a complete report using event coding.

\paragraph{Combine social network analysis with physical data.}
The advent of social media provides unprecedent opportunities to access vast information which can benefit our research. Given such a convenience, we would like to explore the possibilities to renovate some traditional research, hoping to have some extraordinary discoveries and bring more vitality. Take the vaccine and its adverse effects study for example. Traditionally, vaccine research heavily depends on the raw data collected by CDC, hospitals, patients report, and vaccine adverse event reports. However, this data usually suffers some problems, such as time delay, some information is incomplete. Worse more, most of data is isolated. If we can find a way to combine those statistic data with social media data(Tweets, Facebook, etc.), hopefully, we can pull out more information and form a complete picture, including the information of what kind of people are vulnerable to a specify vaccine. In this way, we will be able to better predict the adverse events, or even help to design new vaccine approaches that minimize or eliminate serious vaccine-related reactions. Given current advancements in data mining, this is a revolutionary time for research in real-world applications using social network analysis.

 
% comments from prilim
%\begin{itemize}
%\item North: lessons learned are?
%\item North: what does this say about the real phenomenon?
%\item North: help us learn about the semantics of the phenomenon
%\item North: why not detect rumors in other ways? Why not look at the source?
%\item CT: What's the relationship between rumors and absenteeism.. Spatial granularity can be common and can be harnessed. relationship between community and group anomaly,
%\item CT: climate keywords. Static keywords or dynamic keywords?
%\item Yang: why did u choose these particular techniques? no consistent theme across
%\item North: replace "4 problems in.." with a better title
%\item Feng: a treemap to connect the 4 topics
%\end{itemize}
