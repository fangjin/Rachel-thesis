%As social media (e.g., Twitter) continues to increase in popularity, they are becoming mainstream venues to stage mass movements, such as protests and uprisings.
\section{Contributions}
As social media such as Twitter continue to increase in popularity, they are effectively becoming social sensors that can be utilized for real-world mass movement event detection. Modeling and studying the related adoption patterns provide new insights for investigating both the social and physical aspects of these events and their precursors. This dissertation has presented several approaches and strategies aimed at detecting and predicting mass movements and further inferring their causality, given information composed of a volatile mixture of real news and rumors. Techniques have been presented that are designed to capture information propagation across multiple spaces, as well as a new graph wavelet approach that broadens predictive capabilities to capture group anomalies within networks. A number of different types of mass movements have been investigated and diverse aspects of modeling addressed.

Using social media as indicators for real-word event detection can indeed be a helpful tool, but several limitations apply, perhaps most notably when applied to a specific event type such as the mass movements studied here. First, modeling protest-related topic propagation across networks is never trivial. One challenge is that social protest propagation through online media can spread over large areas far more quickly than traditional methods since users are geographically distributed, while other challenges include the way mass protest information can be spread by multiple social media types and paths, such as word of mouth, and radio and TV news broadcasts. Second, even detecting group anomalies on social media is challenging. For example, Twitter's user network embodies many subgraphs based on social ties with dynamic graph structures that are constantly changing since users are actively contributing content. The other challenge includes real world events that are not only correlated with burst signals, but can also exhibit unusually low levels of activity in social networks. Despite these restrictions, graph wavelets have in fact proven to offer a powerful capacity for capturing graph anomalies (in terms of both burst behavior and absenteeism behavior), even on dynamic changing networks.


A fundamental objective of this research has been to model mass movement adoption behavior, and in doing so, several significant additional advantages have been gained. One contribution is the ability to model information propagation across multiple networks/spaces, and capture the propagation speed and possible propagation paths, as demonstrated in Chapter~\ref{ch:GBM}. Another benefit gained through enhancing the mass movement detection capability is the opportunity this provided to introduce a new group anomaly detection approach, as described in Chapter~\ref{ch:absenteeism}. The use of graph wavelets made it possible to develop a more appropriate definition of group anomaly that covers both bursts and absenteeism utilizing different scales, thereby increasing the probability of capturing protest behaviors. Another benefit is the ability to quantify compartment transition dynamics using the epidemic model SEIZ, thus facilitating the development of screening criteria that can distinguish real movements from rumors on Twitter, as demonstrated in Chapter~\ref{ch:rumor}.

Understanding information propagation over dynamic social networks is becoming a highly popular way to address real-world problems in social network analysis. The research reported in this dissertation analyzed several fundamental questions that underlie a broad range of propagation-like processes, focusing particularly on mass movement adoption and rumor transmission. These methodologies can easily be extended to applications such as infectious diseases, public health and marketing, among others.

\section{Future Directions}
Given the wide scope of this relatively new field, there are many opportunities for conducting potential fruitful research. For example, one very attractive area would be to continue to focus on social network analysis, specifically research into information propagation over dynamic rapidly changing social networks. Here, future research directions fall into two general categories, with one being to deepen the existing theories and algorithms by exploring them in more detail while the other is to broaden the current research by adopting a wider perspective. Examples of these and other research possibilities are presented below.

\paragraph{Extend GBM model}
What happens to the geometric Brownian motion model if the underlying mention network changes over time? How can this this model be adopted or modified to apply to multiple networks? In addition to these theoretical questions, a number of practical applications are worth further investigation, for example would it be possible to introduce the GBM model into the infectious disease domain to model the spread of the zika virus? Assuming that bispace is composed of both a connection network and physical space, can we train the GBM model to estimate individuals' infection probability based on their environment?


\paragraph{Further studies involving graph wavelets}
It should be possible to extend the graph wavelet applications presented in Chapter 4 into other areas based on the two of the major distinguishing properties of graph wavelets. The ability to detect graph anomalies could be utilized to detect the wealth gap between rich and poor in a given region, identify brain neural network anomalies, or detect traffic congestion through road network analysis. Similarly, the ability to identify the central point of a subgraph could be employed to rank key players within networks, detect those spreading rumors in some cascade, or find the sources of infection for certain diseases.

\paragraph{Broaden rumor detection scenarios}
Instead of determining whether a particular story is true or false, it may instead be more practical to predict how likely people are to believe it. Newspapers in particular would find it very useful, especially when it comes to a breaking news story that has not yet been officially confirmed. Before reporting to the public, they need to know how believable the story is. Also this offers a valuable way to decide whether vendors are cheating during online shopping.


\paragraph{Deepen our understanding of personalized information propagation}
It would be useful to know how users' behavior either delays or boosts information propagation, especially when accompanied by strong sentiments. This may help formulate better advertisement strategies if a better way to manipulate the information flow can be identified. Further, it would be interesting to explore opportunities to extract personalized information spread patterns. What kind of news arouse the interest of individual consumers? What kind of role is he or she likely to play? and What kind of push strategy might stimulate their activity? This type of study would be invaluable for precision marketing or personalized recommendations, providing refined content filtering.


\paragraph{Build an intelligent disaster detection system}
Another interesting option would be to build an intelligent system that is efficient at event detection, especially for disasters, protests, and other extraordinary events. Such a system would not only be capable of immediate reporting, but also able to track events and perform causality analyses and even provide future predictions. This type of system would require several critical building blocks: natural disaster detection could be based on the use of graph wavelets for detecting group anomalies, rumors or news detection could employ an epidemic SEIZ model, story causality analysis and event coding. Consider the Flint water crisis for example. First, it would need to identify the water crisis events from a social media analysis and confirm it to be a true story, after which it would trace all the relevant historical news to identify the causality, leading finally to the generation of a complete report using event coding.


\paragraph{Combine social network analysis with physical data.}
The advent of social media provides unprecedented opportunities to access vast quantities of information that could potentially benefit our research. This opens up new possibilities for reinvigorating traditional research in many areas, hopefully gaining new insights and leading to extraordinary discoveries. Take vaccines and the study of their potentially adverse effects, for example. Traditionally, vaccine research depends heavily on the raw data collected by the CDC, hospitals, patient reports, and vaccine adverse event reports. However, this data frequently suffers from problems such as lengthy time delays and incomplete information. Worse, most of the data is considered in isolation. If we can find a way to combine conventional statistical data with social media data such as tweets and Facebook posts, it may be possible to extract more information and create a complete picture showing what kind of people likely to be more vulnerable to a specific vaccine. In this way, it may be possible to better predict adverse events or even contribute to the design of new vaccine approaches that minimize or eliminate serious vaccine-related reactions. Given current advances in data mining, this is a revolutionary time for research in real-world applications using social network analysis.


% comments from prilim
%\begin{itemize}
%\item North: lessons learned are?
%\item North: what does this say about the real phenomenon?
%\item North: help us learn about the semantics of the phenomenon
%\item North: why not detect rumors in other ways? Why not look at the source?
%\item CT: What's the relationship between rumors and absenteeism.. Spatial granularity can be common and can be harnessed. relationship between community and group anomaly,
%\item CT: climate keywords. Static keywords or dynamic keywords?
%\item Yang: why did u choose these particular techniques? no consistent theme across
%\item North: replace "4 problems in.." with a better title
%\item Feng: a treemap to connect the 4 topics
%\end{itemize}
