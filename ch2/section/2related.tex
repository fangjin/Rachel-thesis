\section{Related Work}
We briefly review related work next, which comes from multiple areas.

\par \noindent
{\bf Social movements:}
Oliver and Myers~\cite{oliver1998diffusion}
develop a foundation for theoretical insights of social movements and
describe the limitations of simplified models. The Arab Spring of  2010
served as a context for many
researchers~\cite{gonzalez2011dynamics, bond201261, tufekci2012social, conover2013digital, saad2013mass}
to study the role
social networking sites play in the spread and recruitment
of participants in protests.
A detailed anatomy of
modern social protests is described by Saad-Filho~\cite{saad2013mass}
with the June 2013 anti-government protests in Brazil as a context.
In this work, we study the processes and sociological impacts of protests
in the modern era, fortified by online social networks and the
communities in and around them.\\

\vspace{-0.1in}
\noindent
{\bf Information diffusion in networks:}
Previous studies have approached the modeling of information propagation and
diffusion in social networks through several means, e.g.,
contagion models (SIR~\cite{castellini2007propagation}\, SISa~\cite{hill2010emotions}), diffusion based threshold and cascade models~\cite{kempe2003maximizing}, rise-and-fall patterns~\cite{matsubara2012rise}, coverage models~\cite{singer2012win}, and survival theory~\cite{rodriguez2013modeling}. A good survey of
different models of information diffusion is presented in~\cite{guille2013information}.\\

\vspace{-0.1in}
\noindent
{\bf External influences:}
We believe that the effects of influences that originate external
to the observed diffusion network, such as mass media
and offline spread of information, can impact the way in which information
flows within the online network.
Myers et al.~\cite{myers2012information} study the emergence of URLs
on Twitter with a probabilistic generative process using both
internal and external exposure curves in a contagion-like model.
Similar attention to the role of external factors is
paid by Crane and Sornette~\cite{crane2008robust} for
tracking the popularity of YouTube videos using a diffusion model.
Iwata et al.~\cite{iwata2013discovering} use
a shared cascade Poisson process model to discover
latent influences in social activities such as item adoption.
Using shared parameters among multiple Poisson processes, they were able to simulate sequences of item adoption events. \\

\vspace{-0.1in}
\noindent
{\bf Brownian motion:}
Zhou and colleagues (e.g.,~\cite{zhou2003distance, zhou2003network,
zhou2004network}) develop the notion of Brownian motion on networks
which they
use to discover communities of hierarchical structure both locally
and globally. We extend this approach in this chapter
to formulate a propagation algorithm based on geometric Brownian
motion (GBM). Borrowed from statistical physics, GBM has been
used heavily in finance to model stock price movements.
Scale invariance and the ability to model abrupt bumps
along propagation paths are the primary motivations for using GBMs
to model stochastic processes~\cite{tankov2004financial}. \\

\vspace{-0.1in}
\noindent
Our work builds on the concepts
introduced in~\cite{zhou2003network, iwata2013discovering, zhou2003distance, zhou2004network} but differs from the other diffusion models
described earlier by considering both the role of communities of
users and the abrupt nature of propagation of volatile information such as mass social protests. We include the notion of bispace where both latent (attributed to external influences) and observed user network influences are considered. We infer propagation rates for communities in the observed network and allow
implicit recruitment of users into protest actions through a Poisson process.
