%!TEX root = ../main.tex
% mainfile: ../main.tex

\section{Discussion}
In this chapter, we have characterized mass protest propagation using a
bispace model comprising an observed mentions network space and a latent space. We have introduced a trust function to simulate propagation in observed space using a geometric Brownian motion diffusion process which can be further extended to support communities with different propagation parameters per community. We
considered the latent space of all interactions outside the mentions network to be a Poisson distribution process. We have shown how the GBM diffusion model offers a new approach for modeling propagation through social networks like Twitter. Through our experiments, we find that the time required for spread of protest information through such networks is dependent on the network's substructures. Furthermore, we find that modeling the diffusion process on a community basis provides better results than the assumption that all nodes in the network spread information in the same way.

In future work, we hope to further characterize the hidden network with the goal of uncovering specific latent variables. Additionally, we envision applying the GBM model to other networks, such as the Twitter follower network, to identify those paths most susceptible to information dissemination. Finally, we desire
to compare propagation of mass protest language against other themes,
such as celebratory events, to aid in determining correlations
between topic or sentiment and the resulting social media diffusion.

%In future work, we hope to further characterize the hidden network to discover latent factors. Additionally, we would like to apply the GBM model to friends networks or other association networks to see which network is more adoptable for propagation paths. Further, we want to compare the mass-protest events propagation with other topics such as celebration events to determine the relevance of topic or sentiment in social media propagation.

%how the community structure constrain or promote the propagation process.
% The other aspect we want to focus on is the change of the properties of the mentions network before and after mass-protest-speech sighting. One of the interesting questions we would like to address is whether people start to counter-mention \textit{mass-protest}. In such cases, one would expect mention-links to be formed between nodes with high geodesic distance.
%We also like to extend the Geometric Brownian motion into dynamic
