\chapter{Protest Detection from Group Abnormality}

%\chapter{Event Detection based on Information Absenteeism}
%Information Absenteeism Detection on Social Graphs
% if you want to define paper specific  macros
% then put everthing between begingroup and endgroup
\begingroup
\newcommand{\score}{S}
\newcommand{\myalgo}{CoolAlgo}


%\begin{abstract}
%Event detection in online social media has primarily focused on identifying
%abnormal spikes, or bursts, in activity. However, disruptive events such as socio-economic disasters, civil unrest, and even power outages, often result in abnormal troughs involving group absenteeism of activity. We present the first study, to our knowledge, that models absenteeism and uses detected absenteeism as a basis for event detection in location based social networks (LBSN) such as Twitter. Our framework addresses the challenges of (i) early detection of absenteeism, (ii) identifying the point of origin, and (iii) identifying groups or communities underlying the absenteeism. Our approach uses the formalism of graph wavelets to represent the spatiotemporal structure and user activity in a LSBN. This formalism affords multiscale analysis, enabling us to detect anomalous behavior at different graph resolutions, which in turn allows identification of event location and anomalous groups underlying the network. We introduce a systematic two-pass detection method using graph wavelets to detect group absenteeism and then check if there is a subsequent activity spike. The effectiveness of our approach is highlighted with three case studies involving Twitter activity over Latin American countries.
%\end{abstract}

%\section{Event Detection from Group Absenteeism}


Social microblogs such as Twitter and Weibo are experiencing explosive growth, with billions of users globally sharing their daily status updates online.
For example, Twitter has more than 255 million average monthly active users (78\% from mobile) as of March 31, 2014, and an estimated increase of 25\% per year\footnote{http://solomozone.com/tag/revenues/}.
Various studies have shown that Twitter is viable as a social ``sensor'', and holds great promise for detecting and forecasting significant societal events~\cite{bugel2013multilingual,sakaki2010earthquake}.
In recent years, a significant body of research~\cite{aggarwal2012event,hong2012discovering,lappas2009burstiness,lappas2012spatiotemporal,sakaki2010earthquake,sayyadi2009event,watanabe2011jasmine,weng2011event,yin2011geographical} has focused on modeling bursts and increases of user activity in social media.

However, real world events are not only correlated with burst signals, but can also exhibit unusually low levels of activity in social networks.
As shown in Figure 1, a protest in the city of Natal, Brazil began at 5:00 PM (local time) at the Museum of the Republic, with people gradually joining the demonstration. %\footnote{http://www.jb.com.br/pais/noticias/2013/06/17/manifestantes-invadem-cobertura-do-congresso-nacional-em-brasilia/}.
On Twitter, there was an uncharacteristic lull in activity or {\it group absenteeism} behavior from 6:00 PM---8:00 PM on the same day.
%Another example comes from December 24, 2013, southern Brazil experienced widespread flash floods. According to news sources, more than 50,000 people were forced to flee their homes in Minas Gerais and Espirito Santo, in the southern states of Brazil. Immediately following the floods, Twitter activity in this region dropped by 51\%, and reached its lowest point that evening.
%Other examples of \textit{group absenteeism} that we observed from Latin American Twitter activity include bus strikes in Brazil on May 21, 2014, the Iquique earthquake in Chile on April 1, 2014, and a major power supply disruption in Argentina on December 30, 2013.

\begin{figure}[t]
\centering
\includegraphics[height=1.1in]{figures/Natal_example1.png}
\caption{Detected group absenteeism in Natal, Brazil beginning at 6:00 PM on June 17, 2013. This absenteeism event coincides with a large protest that happened in the region.}
\vspace{-1em}
\label{fig:natal-protest}
\end{figure}

Investigating this phenomenon of unusually calm behavior online holds enormous potential for understanding localized, disruptive societal events. One recent research work~\cite{chi2015ghost} detects vacant housing areas in China using using Baidu positioning data, which is a practical application of absenteeism study. In this chapter we focus on absenteeism based event detection, and introduce this important topic as a key data mining task for social media analytics.
An \textit{absenteeism} event in social networks can be defined as an event which is characterized by a significant lull in activity such as a sudden, sharp decrease of Twitter volume within a short period of time (and which  often precedes a major burst in re-activity).
This chapter presents the first study to systematically investigate group absenteeism in LBSNs.
Using graph wavelet techniques, we pose this problem as one of group anomaly detection.
%To appropriately incorporate absenteeism concepts into our detection approach, we must first address the following questions:
%\begin{itemize}
%\item What scale should we select to model the absenteeism groups? %Which node should be the central point?
%
%\item What is the most efficient approach to select absenteeism groups that are spatially and temporally localized?
%
%\item How do we model an absenteeism signal for event detection? Even though we have clear examples of real world events which can explain the observed absenteeism, not all absenteeism occurrences can be associated with underlying events. Therefore we must be able to differentiate absenteeism from noisy signals for event detection.
%\end{itemize}

Graph wavelets display two outstanding advantages to study the above
questions: scalability and low computational complexity.
In this scenario, the data objects are embedded in a general graph as vertices.
By employing wavelet transforms on the graph, we can construct a wavelet function with a graph structure, and we are able to select absenteeism groups at different scales.
Lastly, we propose a two-pass group anomaly detection method that first detects absenteeism, and then checks if there is a subsequent burst in activity within a specific time period.
By comparing correlations between the wavelet coefficients of both of these groups, we are able to relate observed absenteeism to a possible real world event.


Our contributions are:
\begin{itemize}
\item We propose the method to modeling group absenteeism as a basis for event detection.
%Even though burst has been extensively discussed in previous work, however, the absenteeism has different patterns and plays an irreplaceable role in event detection.
\item We incorporate graph wavelets as a mechanism to detect the most anomalous subgraphs at different scales. We demonstrate how this is a powerful technique for social media analytics.
\item We propose a novel two-pass event detection method that uses correlation scores between the group depicting \textit{absenteeism} and the group demonstrating increased activity to probabilistically determine the likelihood of an event.
\end{itemize}

%The rest of the chapter is organized as follows. Section~\ref{sec:related} reviews related work and existing methodologies and Section~\ref{sec:preliminaries} formalizes the research problem. In Section~\ref{sec:algorithm}, we first discuss the graph wavelet formalism for group absenteeism detection, and subsequently demonstrate how it can be used for two-pass event detection. Section~\ref{sec:experiment} presents extensive experiments for event detection, and the chapter concludes with a summary of the research in Section~\ref{sec:conclusion}.


\subsection{Problem Statement}
\label{sec:problemformulation}
We focus on the problem of event detection from online social networks, based on the absenteeism behavior observed in user activity in geographically proximal communities or group of cities.
We define this problem as following: \emph{given a graph and \textit{absenteeism score} vector, $\mathbf{G}(V,E,W;f^t)$ at time interval $t$, select a subset $\Sigma \subseteq V$, such that
\vspace{-0.5em}
\begin{eqnarray}
 \label{eq: problem}
    \Sigma=\underset{P\subseteq V, P \mbox{ is compact}}{\arg\min}\ \ \sum_{v_k\in P} {f(k)}
\end{eqnarray} }

A general solution to this problem is using a combinatorial optimization technique, where by defining a constrained objective function over a network one can identify subset of vertices which maximize the corresponding function~\cite{rozenshtein2014event}. Therefore, Equation~\ref{eq: problem} can be modified as:
\vspace{-0.5em}
\begin{eqnarray}
 \label{eq: problem_conventional}
    \Sigma=\underset{P\subseteq V}{\arg\min}\ \ \sum_{v_k\in P} {f(k)}+\lambda \mu(P)
\end{eqnarray}
, where $\mu(P)$ is the compactness penalty function of $P$ (e.g., the sum of distances among
all pairs of the vertices in $P$~\cite{rozenshtein2014event}), and $\lambda$ is the regularization parameter.
Such methods suffer from the following issues:
\begin{enumerate}
\item To define and measure the compactness of subset $P\subseteq V$ is challenging, considering the exponential varieties of complex graphs.
\item To determine a suitable regularization parameter $\lambda$ in the objective function is ambiguous, because simply combining multiple physically different concepts in the objective function makes the optima sensitive to $\lambda$.
\item To solve this objective function is often a \textbf{NP-hard} problem, which makes it unpractical in many real world applications. Sometimes, even the approximate solutions are of high computation complexity, if there are any.
\end{enumerate}

In contrast, our approach proposes a novel, absenteeism based events detection algorithm in social networks using spectral graph wavelet theory.
The graph wavelets focus on the intrinsic geometric structure of the graph by transversing each vertex $v_i\in V$, and mining the topological information of both local and globally centered vertices supports the ability to conduct a multiscale analysis.
In addition, the graph wavelet approach does not introduce any ``subjective'' objective functions or other compactness concepts, and thus provides a fair and low computational method in terms of complexity for identifying abnormal group behavior in a wide variety of application scenarios.

\subsection{Graph Wavelet}
Classic wavelet is called mathematical microscope since it is capable of showing signal abnormality with different scales. In the case of complex networks, graph wavelets render the graph with good localization properties both in frequency and vertex (i.e. spatial) domains. Their scaling property allows us to zoom in/out of the underlying structure of the graph.

It is useful to analyze $f$ by taking into account the intrinsic geometric structure of the graph $\mathbf{G}$. In order to identify and exploit structure of  $f\in \mathbb{R}^N$, the spectral graph $\sigma({\mathcal{L}}):=\{\chi_l\}_{l=0}^{N-1}$ can be used as a dictionary of atoms~\cite{shuman_ACHA_2013}. Thus, $f$ can be decomposed as a linear combination of $\{\chi_l\}_{l=0}^{N-1}$ as
\vspace{-0.5em}
\begin{equation}
\label{eq:graph_fourier}
f(n)= \sum\limits_{l=0}^{N-1}\hat{f}(l)\chi_l(n)
\end{equation}
\vspace{-0.5em}
, where
\vspace{-0.5em}
\begin{equation}
\label{eq:graph_fourier1}
\hat{f}(l):= \sum\limits_{n=0}^{N-1}\chi^*_l(n)f(n)
\end{equation}
$\chi_l$ is called the Fourier frequency of $f(n)$ based on the graph $\mathbf{G}$, and $\hat{f}(l)$ is the corresponding Fourier coefficient.
Equation~\ref{eq:graph_fourier1} and Equation~\ref{eq:graph_fourier} are called Fourier transform and inverse Fourier transform, respectively.
Equation~\ref{eq:graph_fourier1} gives a clear representation of the Fourier components in $f(n)$.
However, information concerning the vertex-location can not be identified from the Fourier transform. To address this issue, Hammond et al.~\cite{hammond2011wavelets} proposed constructing wavelet transforms of functions over the vertices using weighted graphs, described in the following steps:

\begin{enumerate}
\item Define a continuous generating kernel functions $g(x)$ on $\mathbb{R}^+$;
\item Then, select a central vertex $v_a \in {V}$ and scale $s$, set the frequency coefficients as $g(s\lambda_l)\chi^*_l(a)$ for each frequency component $\chi_l$;
\item Finally, sum up all those frequency components $\chi_l$.
\end{enumerate}
In this way, the graph wavelet at central vertex $v_a$ is constructed as:
\vspace{-0.5em}
\begin{equation}
\label{eq:graphwaveletdefinition}
\psi_{s,a}(n) = \sum\limits_{l=0}^{N-1}g(s\lambda_l)\chi_l^*(a)\chi_l(n)
\end{equation}
After setting up the graph wavelet, the wavelet coefficients for $f$ can be defined as
\vspace{-0.5em}
\begin{equation}
\label{eq:graph_graphwavelet}
W_f(s,a)=<\psi_{s,a}, f>=\sum\limits_{l=0}^{N-1}g(s\lambda_l)\hat{f}(a)\chi_l(n)
\end{equation}



\subsection{Two-pass Event Detection Model}

We intend to propose a two-pass absenteeism based event detection algorithm. The underlying rationale of this algorithm is based on the following concepts.
\begin{enumerate}
\item As discussed above, distribution of $f$ can be well reconstructed by the $J$ scaling and $NJ$ wavelet coefficients. Each of those normalized wavelet coefficients $W'_f(s,a)$ represents a distribution pattern of $f$ on $\mathbf{G}$.
It is equivalent to saying that $\psi_{s,a}$ represents a special distribution pattern, which shares a large and uniform value around the central vertex with scale $s$.
\item When a significant event occurs, preceded by group absenteeism behavior in social networks, such as a severe earthquake or a massive protest, it is likely to be succeeded by a spike or burst in online user activity.
With this observation, we can represent an absenteeism behavioral pattern as $\psi_{s_l,a_l}$ at time $l$ centering at vertex $v_{a_l}$, and a burst related pattern as $\psi_{s_{\tau,a_\tau}}$ at time $\tau$ centering at vertex $v_{a_\tau}$. We assume the burst pattern happens within the time window size of $L$ after absenteeism pattern is identified. Further, a notion of response time can represented using the time difference $t_{rsp}=\tau-l$.
\item Both absenteeism and burst signal must show a strong correlation, especially if they occur in close proximity spatially and temporally.
For instance, taking the power-cut-off for instance, usually only people who live in the affected area will ``yield at " this event a lot because it brings inconvenience to their life. However, people who live outside of the affected areas would hardly mention this event. Thus, to measure the correlation between absenteeism pattern and burst pattern is proposed as:
\begin{equation}
\label{eq:eventsimilarity}
\rho(\psi_{s_l,a_1}, \psi_{s_\tau,a_\tau})= \frac{<\psi_{s_l,a_1}, \psi_{s_\tau,a_\tau}>}{||\psi_{s_l,a_1}||\cdot ||\psi_{s_\tau,a_\tau}||}
\end{equation}
Based on these concepts, the higher the correlation, the higher probability that burst patterns is caused by the preceding group absenteeism. When $\rho$ is above the threshold (threshold is set at 0.5), we infer that an event occurred and that it evolved on social networks into distinct phases: first group absenteeism, followed by a spike or burst in user activity.
\end{enumerate}


\subsection{Experiments}
We seek to answer the following questions using our model:
To appropriately evaluate our detection approach, we seek to answer the following questions:
\begin{itemize}
\item What scale should we select to model the absenteeism groups? %Which node should be the central point?
\item What is the most efficient approach to select absenteeism groups that are spatially and temporally localized?
\item How do we model an absenteeism signal for event detection? Even though we have clear examples of real world events which can explain the observed absenteeism, not all absenteeism occurrences can be associated with underlying events. Therefore we must be able to differentiate absenteeism from noisy signals for event detection.
\item Given a time period, what is the recall and precision to identify events using group absenteeism as signals?
\end{itemize}
\subsection{Datesets}
We will uses tweets from 22 countries in Latin America that were collected over 24 months, from May 2013 to May 2015.


\section{Introduction} \label{sec:intro}
Social microblogs such as Twitter and Weibo are experiencing explosive growth, with billions of users globally sharing their daily status updates online.
For example, Twitter has more than 255 million average monthly active users (78\% from mobile) as of March 31, 2014, and an estimated increase of 25\% per year\footnote{http://solomozone.com/tag/revenues/}.
Various studies have shown that Twitter is viable as a social ``sensor'', and holds great promise for detecting and forecasting significant societal events~\cite{bugel2013multilingual,sakaki2010earthquake}.
In recent years, a significant body of research~\cite{aggarwal2012event,hong2012discovering,lappas2009burstiness,lappas2012spatiotemporal,sakaki2010earthquake,sayyadi2009event,watanabe2011jasmine,weng2011event,yin2011geographical} has focused on modeling bursts and increases of user activity in social media.

However, real world events are not only correlated with burst signals, but can also exhibit unusually low levels of activity in social networks.
As shown in Figure 1, a protest in the city of Natal, Brazil began at 5:00 PM (local time) at the Museum of the Republic, with people gradually joining the demonstration. %\footnote{http://www.jb.com.br/pais/noticias/2013/06/17/manifestantes-invadem-cobertura-do-congresso-nacional-em-brasilia/}.
On Twitter, there was an uncharacteristic lull in activity or {\it group absenteeism} behavior from 6:00 PM---8:00 PM on the same day.
%Another example comes from December 24, 2013, southern Brazil experienced widespread flash floods. According to news sources, more than 50,000 people were forced to flee their homes in Minas Gerais and Espirito Santo, in the southern states of Brazil. Immediately following the floods, Twitter activity in this region dropped by 51\%, and reached its lowest point that evening.
%Other examples of \textit{group absenteeism} that we observed from Latin American Twitter activity include bus strikes in Brazil on May 21, 2014, the Iquique earthquake in Chile on April 1, 2014, and a major power supply disruption in Argentina on December 30, 2013.



\begin{figure}[t]
\centering
\includegraphics[width=4.5in]{figures/Natal_example1.png}
\caption{Detected group absenteeism in Natal, Brazil beginning at 6:00 PM on June 17, 2013. This absenteeism event coincides with a large protest that happened in the region.}
\label{fig:natal-protest}
\end{figure}


Investigating this phenomenon of unusually calm behavior online holds enormous potential for understanding localized, disruptive societal events.
In this paper we focus on absenteeism based event detection, and introduce this important topic as a key data mining
task for social media analytics.
An \textit{absenteeism} event in social networks can be defined as an event which is characterized by a significant lull in activity such as a sudden, sharp decrease of Twitter volume within a short period of time (and which  often precedes a major burst in re-activity).
This paper presents the first study to systematically investigate group absenteeism in LBSNs.
Using graph wavelet techniques, we pose this problem as one of group anomaly detection.
To appropriately incorporate absenteeism concepts into our detection approach, we must first address the following questions:

\begin{itemize}
\item What scale should we select to model the absenteeism groups? %Which node should be the central point?
    
\item What is the most efficient approach to select absenteeism groups that are spatially and temporally localized?
    
\item How do we model an absenteeism signal for event detection? Even though we have clear examples of real world events which can explain the observed absenteeism, not all absenteeism occurrences can be associated with underlying events. Therefore we must be able to differentiate absenteeism from noisy signals for event detection.
\end{itemize}

Graph wavelets display two outstanding advantages to study the above
questions: scalability and low computational complexity.
In this scenario, the data objects are embedded in a general graph as vertices.
By employing wavelet transforms on the graph, we can construct a wavelet function with a graph structure, and we are able to select absenteeism groups at different scales.
Lastly, we propose a two-pass group anomaly detection method that first detects absenteeism, and then checks if there is a subsequent burst in activity within a specific time period.
By comparing correlations between the wavelet coefficients of both of these groups, we are able to relate observed absenteeism to a possible real world event.

Our contributions are thus:

\begin{itemize}
\item To the best of our knowledge this is the first study to modeling group absenteeism as a basis for event detection.
%Even though burst has been extensively discussed in previous work, however, the absenteeism has different patterns and plays an irreplaceable role in event detection.

\item We incorporate graph wavelets as a mechanism to detect the most anomalous subgraphs at different scales. We demonstrate how this is a powerful technique for social media analytics.

\item We propose a novel two-pass event detection method that uses correlation scores between the group depicting \textit{absenteeism} and the group demonstrating increased activity to probabilistically determine the likelihood of an event.
\end{itemize}

The rest of the paper is organized as follows. Section~\ref{sec:related} reviews related work and existing methodologies and Section~\ref{sec:preliminaries} formalizes the research problem. In Section~\ref{sec:algorithm}, we first discuss the graph wavelet formalism for group absenteeism detection, and subsequently demonstrate how it can be used for two-pass event detection. Section~\ref{sec:experiment} presents extensive experiments for event detection, and the paper concludes with a summary of the research in Section~\ref{sec:conclusion}.



%\begin{itemize}
%\item How do we differentiate absenteeism from noise signals?
%\item How can wavelet graph be used for group abnormality detection?
%\item How do we detect events from linking absenteeism group with burst group?
%\end{itemize}


\section{Related Work} \label{sec:related}

Anomaly detection in Graphs has been well studied using outlier detection~\cite{akoglu2009anomaly}. When considering group concept, two directions has been studied~\cite{akoglu2015graph}: one is anomalies in unlabeled/plain graphs~\cite{noble2003graph}, the other is in attributed graphs. In the plain graph anomaly detection, since the only given information is its structure, various features such as distances, communities~\cite{sun2005neighborhood} have been employed to define graph anomaly. In work of~\cite{henderson2010metric}, more metrics like vertices, edges, degree, weight, connected components are incorporated into detection framework. In attributed graphs, features regarding nodes behaviors make it possible to have a richer graph representation, which is usually tied with some real-world applications. Such as \cite{yu2014glad} defines the group based on the term of role, and model the normal groups follow the same pattern with respect to their role mixture rates. In our work, we consider both graph structure and nodes features, propose a graph wavelet based approach for group anomaly detection, which can guarantee the detected group to be automatically compact, with linear computation complexity and scalability.

Event detection based on LSBNs is a research area that has attracted significant attention in the last years. Traditional approaches focus on capturing spatiotemporal burstiness of keywords~\cite{lappas2009burstiness,lappas2012spatiotemporal}, Kalman filtering to track the geographical trajectories of hot spots of tweets related to earthquakes~\cite{sakaki2010earthquake}; detecting topics of interest that are coherent in geographic regions~\cite{eisenstein2010latent,hong2012discovering,yin2011geographical}; applying clustering-based approaches search for emerging clusters of documents or terms using predefined similarity metrics that consider factors such as term co-occurrences and social interactions~\cite{aggarwal2012event,sayyadi2009event,watanabe2011jasmine,weng2011event}; and using the notion of compactness of a graph~\cite{rozenshtein2014event} to detect events. Several statistical methods have also been used, based on Kulldroff’s spatial scan statistic~\cite{kulldorff1997spatial}, to detect spatial outliers~\cite{chen2008detecting} and have been applied to a wide variety of domains including transportation networks, civil unrest forecasting~\cite{zhao2014unsupervised}, and heterogeneous social media graphs~\cite{chen2014non}.

Our approach to event detection problem is conceptually different from above mentioned studies. It includes a graph-theoretic framework to detect absenteeism related anomalies and correlate them with future events. Although group absence behavior has been widely studied in the area of organizational behavioral studies~\cite{gaudine2001effects,seamonds1982stress}, it remains unexplored in the area of social network analysis. Resembling closely to group anomaly detection in complex networks, our detection approach is further distinguished by its focus on groups rather than individuals. Existing approaches to group anomaly detection include building generative models of group anomalies~\cite{xiong2011hierarchical,yu2014glad} where the goal is to automatically infer the groups and detect group anomalies in a social network. Typical to mixture models such methods suffer from high computational complexity due to the size of data and are heavily parameterized.

One of key challenges of our research problem is adapting the detection procedure for both missing and bursty activity groups. For this purpose, we incorporate spectral graph wavelets~\cite{hammond2011wavelets} into our algorithm. This strategy has been quite effectively used in multiscale community mining~\cite{tremblay2014graph}.
Wavelet methods based on spectral graph theory have been applied in a wide array data mining areas such as community detection, anomaly detection~\cite{calderara2011detecting} and other machine learning tasks~\cite{shuman_ACHA_2013,ghosh2003wavelet,rustamov2013wavelets,2000wavecluster}. By constructing wavelets over graphs we are able take advantage of local information encoded in graph structure and then cluster and identify nodes which are similar in a scale-dependent fashion.

% The relevant methods can be classified into three categories: burst detection, geographical topic modeling, and clustering. Burst detection methods search for space-time regions that %have abnormally high counts of some predefined terms~\cite{lappas2009burstiness,lappas2012spatiotemporal}. Sakaki et al. consider spatial-temporal Kalman filtering to track the %geographical trajectories of hot spots of Tweets related to earthquakes~\cite{sakaki2010earthquake}. Geographic topic modeling based methods detect topics of interest that are coherent %in geographic regions~\cite{eisenstein2010latent,hong2012discovering,yin2011geographical}. Clustering-based approaches search for emerging clusters of documents or terms using %predefined similarity metrics that consider factors such as term co-occurrences and social %interactions~\cite{aggarwal2012event,sayyadi2009event,teitler2008newsstand,watanabe2011jasmine,weng2011event}.


\section{Preliminaries} \label{sec:preliminaries}
In this section, we formalize our approach to event detection.
We first describe the accompanying notations in section~\ref{sec:notations} which will be used throughout the chapter.
Then we formally present our research problem statement, provide a brief comparison of our approach to a conventional solution, and review the challenging issues that are relevant to an event detection problem.
\subsection{Notations}
\label{sec:notations}
Let's assume we are given an undirected, weighted graph $\mathbf{G}(V,E,W;f)$, where $V=\{v_0,v_1,...,v_{N-1}\}$ represents the set of $N$ cities, $E$ refers to the connections between neighboring cities, and $W$ is a vector of non-negative weights associated with each edge $e_{ij}\in E$ as a function of geographical distance between a pair of vertices $(v_{i}, v_{j})$. The function, $f: V \rightarrow {\mathbb{R}}^N$ maps the vertices of graph $\mathbf{G}$, and $f(n)$ stands for the value on the vertex $v_n$. Graph $\mathbf{G}$'s adjacency matrix $\mathbf{A}$ is of size $N\times N$, where each element $a_{ij}$ is represented as:
\begin{equation}
a_{ij} = \left\{ \begin{array}{rl}
 w_{ij} &\mbox{ when $e_{ij}\in {E}$} \\
  0 &\mbox{ otherwise}
       \end{array} \right.
\end{equation}
Here, $\mathbf{A}$ is symmetric since $a_{ij}=a_{ji}$.
Let $d_i=\sum\limits_{v_j \in V}a_{ij}$ be the sum of all edge weights that are incident on $v_i$ and $\mathbf{D}$ be the diagonal matrix denoted as $\mathbf{D}=diag\{d_1,d_2,\ldots,d_N\}$. A Laplacian matrix $\mathcal{L}$ is defined as $\mathcal{L}=\mathbf{D-A}$. It is a symmetric matrix and has real eigenvalues $\lambda_{i}$ such that $0 = \lambda_{0} < \lambda_{1} \leq \lambda_{2} \leq \ldots \leq \lambda_{N-1} = \lambda_{max}$ and a complete set of $\mathcal{L}$'s normalized eigenvectors~\cite{bapat2010graphs} $\chi_{i}$ for $i=0,1,2,...,N-1$ described by:
\begin{equation}
\label{eq:eigenvalues}
\mathcal{L}\chi_{i}=\lambda_{i}\chi_{i}
\end{equation}
Obviously, $\chi_o(n)=\frac{\vec{\textbf{1}}}{\sqrt{N}}=\frac{1}{\sqrt{N}}\{1,1,...,1\}$, and is called direction component of $\mathbf{G}$.

\vspace{-1mm}
\paragraph{\textbf{Absenteeism Score}}
Although the raw volume of user interactions is a good indicator of users' online activities, it can be noisy and often exhibits a strong temporal dependence~\cite{cho2011friendship}.
For example, in Twitter, the number of user interactions tend to peak later in the day.
This can be attributed to the fact users tweet typically at home after work or school.
In order to differentiate event-related absenteeism from such noisy signals that arise from the periodicity of users' daily activities, we need to remove these artifacts from our time series data.
Empirically, tweeting locations can be modeled as a Gaussian distribution, which allows us to transform the raw counts of tweets from a given city, $v_{i}$, at time interval, $t$, into a \textit{z-score} measure, and in turn to calculate its \textit{absenteeism score} as:
\begin{equation}
	\label{eq:zscore}
	\begin{array}{l}
		f^t(i;T) =(X^t_i-\mu)/{\sigma}
	\end{array}
\end{equation}
, where $i$ denotes the index of the vertex, $X^t_i$ is the tweeting volume at time interval $t$, $\mu$ is the trailing $T$-day moving average of the volume at time $t$, and $\sigma$ is the standard deviation of that average volume.
Here, a positive absenteeism score indicates a high levels of user activity, while a negative score indicates lower levels in activity.
For the experiments described later in the chapter, we set the value of $T=30$ days. The notation for absenteeism function $f^t(i;T)$ is simplified to $f$ when $t$ and $T$ are obvious from the context.
\subsection{Graph Anomaly}
\label{sec:Graph_Anomaly}
%Classic wavelet is called mathematical microscope since it is capable of showing signal abnormality with different scales.
%In the case of complex networks, graph wavelets render the graph with good localization properties both in frequency and vertex (i.e. spatial) domains. Their scaling property allows us to zoom in/out of the underlying structure of the graph.
According to Equation~\ref{eq:eigenvalues}, eigenvalues of Laplacian matrix $\mathcal{L}$ can be presented as:
\begin{equation}
\label{eq:lambda}
\lambda_{l}=\chi_{l}^T\mathcal{L}\chi_{l}= \sum_{e_{mn}\in E} w_
{mn}[\chi_{l}(m)-\chi_{l}(n)]^2
\end{equation}
$\lambda_l$ summarizes all the eigenvector deviations on any directly connected vertex $v_m$ and $v_n$ in $\mathbf{G}$. Since each term in the summation of the right-hand side is non-negative, the eigenvectors associated with smaller eigenvalues are smoother; i.e., the component differences between neighboring vertices are
small. As the eigenvalue increases, larger differences in neighboring
components of the graph Laplacian eigenvectors may be present.
Hence, for larger $\lambda_l$, its corresponding eigenvector, $\chi_l(n)$, has larger deviation among connected vertices~\cite{shuman2015vertex}. For this reason, we call $\{(\lambda_l;\chi_l)\}$ the graph anomaly pattern (also called Fourier frequency by some researchers) of $\mathbf{G}$. As mentioned above, $\chi_0(n)$ is the direct component of $\mathbf{G}$ since $\chi_0(i)=\frac{1}{\sqrt{N}}$ for any $v_i\in V$.


It is useful to analyze $f$ by taking into account the intrinsic geometric structure of the graph $\mathbf{G}$. In order to identify and exploit structure of $f\in \mathbb{R}^N$, the spectral graph $\sigma({\mathcal{L}}):=\{\chi_l\}_{l=0}^{N-1}$ can be used as a dictionary of atoms~\cite{shuman_ACHA_2013}. Thus, $f$ can be decomposed as a linear combination of $\{\chi_l\}_{l=0}^{N-1}$ as
\begin{equation}
\label{eq:graphFourier}
f(n)= \sum\limits_{l=0}^{N-1}\hat{f}(l)\chi_l(n)
\end{equation}
, where
\begin{equation}
\label{eq:graphFourier1}
\hat{f}(l):= \sum\limits_{n=0}^{N-1}\chi^*_l(n)f(n)
\end{equation}
$\hat{f}(l)$ is the inner product of $f$ and anomaly pattern $\chi_l$, and is called the graph anomaly degree in this chapter, and is also called the corresponding Fourier coefficient.


\subsection{Generalized Graph Anomaly}
\label{sec:Generalized_Graph_Anomaly}
Equation~\ref{eq:graphFourier1} gives a clear representation of the anomaly patterns in $f(n)$ based on graph.  As discussed in section ~\ref{sec:Graph_Anomaly}, $\lambda_l$ only summarizes all deviations among all the direct connected vertices in graph $\mathbf{G}$.
However, in many applications,
deviations among vertices which are not connected directly might also carry important values. Taking social media network for instance, human behavior is not only being affected by his/her direct connected friends, but also by some ``far distance" friends in the network. Generalizes graph anomaly considers the deviations among all vertex pairs, which even not being connected directly, as long as they are close enough to each other.


Let $d_G(m,n)$ denote the minimum number of edges for any paths connecting $v_m$
and $v_n$ in graph $\mathbf{G}$, and $d_G(m,n)$ can be written as:
\begin{equation}
d_G(m,n)=\underset{p}\arg\min\{k_1,k_2,k_3,...,k_p\}
\end{equation}
subject\hspace{1mm}to
\begin{equation}
m=k_1, n=k_p, \hspace{1mm}and\hspace{1mm} w_{k_r,k_{r+1}}>0 \hspace{1mm} for \hspace{1mm} 1\leq r<p
\end{equation}
Note that $d_G$ disregards the values of the edge weights.
\newtheorem{thm}{Theorem}
\newtheorem{lem}[thm]{Lemma}
\begin{lem}
\label{lem:1}
Let $\mathbf{G}$ be a weighted graph, $\mathcal{L}$ the graph Laplacian and $p$ > 0 an integer. For any two
vertices $v_m$ and $v_n$ in graph $\mathbf{G}$, if $d_G(m,n) > p $ then $\mathcal{L}^p
(m,n) = 0$.
\end{lem}
The comprehensive proof of lemma~\ref{lem:1} can be found in~\cite{hammond2011wavelets}. $\mathbf{G}^p(V^p,E^p)$ denotes the graph with Laplacian matrix $\mathcal{L}^p$, and $\rho_{mn}$ denotes the weight of edge $e_{mn}$, where $e_{mn}\in E^p$. Obviously, $V^p=V$. According to the properties of  positive semi-definite, the eigenvalues and eigenvectors of $\mathcal{L}^p$ are $\{(\lambda_l^p;\chi_l)\}$, where $0\leq l \leq {N-1}$. According to Equation~\ref{eq:lambda}, for $\mathcal{L}^p$, we have
\begin{equation}
\label{eq:lambda2}
\lambda_{l}^p=\chi_{l}^T\mathcal{L}^p\chi_{l}= \sum_{e_{mn}\in E^p} \rho_
{mn}[\chi_{l}(m)-\chi_{l}(n)]^2
\end{equation}
Further, according to lemma~\ref{lem:1}, if $d_G(m,n)>p$, then $\mathcal{L}^p(m,n)=0$, which equivalently means $\rho_{mn}=0$. Hence, $\lambda_l^p$ only summarizes deviations among all vertex pairs which are closer than $p$ edges in graph $\mathbf{G}$. For this reason, we call $\{(\lambda_l^p;\chi_l)\}$ the generalized graph anomaly pattern of $\mathbf{G}$.

\subsection{Problem Statement}
\label{sec:problemformulation}
In this chapter, we focus on the problem of event detection from online social networks, based on the absenteeism behavior observed in user activity in geographically proximal communities or group of cities.
Conventionally, this problem can be described as following: \emph{given a graph and \textit{absenteeism score} vector, $\mathbf{G}(V,E,W;f^t)$ at time interval $t$, select a subset $\Sigma \subseteq V$, such that
\begin{eqnarray}
 \label{eq: problem}
    \Sigma=\underset{P\subseteq V, P \mbox{ is compact}}{\arg\min}\ \ \sum_{v_k\in P} {f(k)}
\end{eqnarray} }

A general solution to this problem is using a combinatorial optimization technique, where by defining a constrained objective function over a network one can identify subset of vertices which maximize the corresponding function~\cite{rozenshtein2014event}. Therefore, Equation~\ref{eq: problem} can be modified as:
\begin{eqnarray}
 \label{eq: problem_conventional}
    \Sigma=\underset{P\subseteq V}{\arg\min}\ \ \sum_{v_k\in P} {f(k)}+\lambda \mu(P)
\end{eqnarray}
, where $\mu(P)$ is the compactness penalty function of $P$ (e.g., the sum of distances among
all pairs of the vertices in $P$~\cite{rozenshtein2014event}), and $\lambda$ is the regularization parameter.
Such methods suffer from the following issues:
\vspace{-1.5mm}
\begin{enumerate}
\item To define and measure the compactness of subset $P\subseteq V$ is challenging, considering the exponential varieties of complex graphs.
\item To determine a suitable regularization parameter $\lambda$ in the objective function is ambiguous, because simply combining multiple physically different concepts in the objective function makes the optima sensitive to $\lambda$.
\item To solve this objective function is often a \textbf{NP-hard} problem, which makes it unpractical in many real world applications. Sometimes, even the approximate solutions are of high computation complexity, if there are any.
\end{enumerate}
\vspace{-1.5mm}
In contrast, our approach proposes a novel, absenteeism based events detection algorithm in social networks using spectral graph wavelet theory.
The graph wavelets focus on the intrinsic geometric structure of the graph by transversing each vertex $v_i\in V$, and mining the topological information of both local and globally centered vertices supports the ability to conduct a multiscale analysis.
In addition, the graph wavelet approach does not introduce any ``subjective'' objective functions or other compactness concepts, and thus provides a fair and low computational method in terms of complexity for identifying abnormal group behavior in a wide variety of application scenarios.


%\begin{figure}[h]
%	\centering
%    {
%		\includegraphics[width=2in] {figures/Z-Score-distribution.png}
%		\label{fig:distribution}
%	}
%	\caption{ Z-score distribution of city Sao Paulo, Brazil from Aug 1, 2012 to January 30, 2014 with time interval of five minutes. }
%	\label{fig:zscore-distribution}
%\end{figure}


\section{Algorithm}
\label{sec:algorithm}

In this section, we first introduce Graph Fourier Transform concept, explaining eigenvector and eigenvalue meanings, then based on that define anomaly index of graph. The next is to describe graph wavelet's features such as reconstruction and localization. In section~\ref{sec:Group_Anomaly_Detection_via_graph_wavelet}, we propose the group anomaly detection algorithm via graph wavelet. And in section~\ref{sec:Group Absenteeism Event Detection} we present the two-pass event detection algorithm.

\subsection{Graph Fourier Transform}
\label{sec:Graph_Fourier_Transform}
Given a signal $f$ defined on graph $\mathbf{G}$, its Graph Fourier Transform is considered as the projection of $f$ on the complete set of $\{\chi_l\}_{l=0}^{N-1}$, and is written as~\cite{hammond2011wavelets}:
\begin{equation}
\label{eq:Graph_Fourier_Transform1}
\hat{f}(l)=<\chi_{l},f>=\sum_{i=1}^{N}\chi^*_{l}(i)f(i)
\end{equation}
Since $\{\chi_l\}_{l=0}^{N-1}$ is complete, therefore, $f$ can be recovered by its Graph Fourier Transform coefficients $\hat{f}(l)$ as~\cite{hammond2011wavelets}:
\begin{equation}
\label{eq:Inverser_Graph_Fourier_Transform}
f(n)=\sum_{l=0}^{N-1}\hat{f}(l)\chi_{l}(n)
\end{equation}
$\hat{f}(l)$ is the coefficient of component $\chi_l$.


\begin{figure}[h]
	\centering
    {
		\includegraphics[height=1.4in] {figures/graph_G.png}
	}
	\caption{Graph $\mathbf{G_1}$, all edges's weight are $1$.}
	\label{fig:graph_G}
\end{figure}


\begin{figure}[ht]
	\centering
	\subfigure[]{
		\includegraphics[width= 3in, height=1.6in] {figures/frequency.png}
		\label{fig:frequency1}
	}
	\subfigure[]{
		\includegraphics[width= 2.8in, height=1.6in] {figures/g1_gamma.png}
		\label{fig:g1_gamma}
	}

	\subfigure[]{
		\includegraphics[width= 3in] {figures/same_graph.png}
		\label{fig:same_graph}
	}

	\caption{(a): Eigenvector distribution along each vertex in graphs $\mathbf{G_1}$.  (b): anomaly index $\gamma_f(l)$ of $f_1=[2,3,4,3,2,1]$ on graph $\mathbf{G_1}$. (c): anomaly index $\gamma_f(l)$ of $f_1=[2,3,4,3,2,1]$  and $f_2=[2,2,-3,4,3,1]$ on graph $\mathbf{G_1}$, where $\gamma_{f_1}=0.905$, and $\gamma_{f_1}=0.073$, labelled in red oval.}
	\label{fig:f_on_g2}
\end{figure}


\subsubsection{eigenvector $\chi_l$}
As an analog with classical signal processing, eigenvector $\chi_l$ is also called frequency of $\mathbf{G}$ by some researchers. In the later part of this paper, $\chi_l$ will be called eigenvector or frequency, alternatively. However, unlike the traditional frequency concept in classical signal processing fields, the frequencies of $\mathbf{G}$ is a set of discreet vectors with length of $|V|$. Interestingly, like the classical signal Fourier Transform, Parseval relation still holds; i.e.~\cite{shuman2015vertex},
\begin{equation}
\label{eq:Parseval}
||\hat{f}||_2^2=||f||_2^2
\end{equation}
Equation~\ref{eq:eigenvalues} means that energy in vertex domain and frequency domain is equal for any graph signal $f$. Without loss of generality, we assume $||f||_2 =1$, if there is no explicit notations.

\subsubsection{eigenvalue $\lambda_l$}
According to the definition of eigenvalue $\lambda_l$  in Equation~\ref{eq:eigenvalues}, the following equation holds:
\begin{equation}
\label{eq:lambda1}
\chi_{l}^T\lambda_{l}\chi_{l}=\chi_{l}^T\mathcal{L}\chi_{l}= \sum_{e_{mn}\in E} w_
{mn}[\chi_{l}(m)-\chi_{l}(n)]^2
\end{equation}Since $\chi_{l}$ is normalized, and $||\chi_{l}||_2 =1$, then,
\begin{equation}
\label{eq:lambda2}
\chi_{l}^T\lambda_{l}\chi_{l}=\lambda_l= \sum_{e_{mn}\in E} w_
{mn}[\chi_{l}(m)-\chi_{l}(n)]^2
\end{equation}
From equation~\ref{eq:lambda2}, we can see that $\lambda_l$ summarizes all the eigenvector deviations on any directly connected vertices $v_m$ and $v_n$ in $\mathbf{G}$. Since each term in the summation of the right-hand side is non-negative, the eigenvectors associated with smaller eigenvalues are smoother; i.e., the component differences between neighboring vertices are
small~\cite{shuman2015vertex}. As the eigenvalue increases, larger differences in neighboring
components of the graph Laplacian eigenvectors is present.
Hence, for larger $\lambda_l$, its corresponding eigenvector, $\chi_l(n)$, has larger deviation among connected vertices. According to the definition of Laplacian matrix $\mathcal{L}$, it is easy to verify that $\lambda_0=0$ since $\mathcal{L}\cdot\vec{\textbf{1}}= 0\cdot\vec{\textbf{1}}$, where $\vec{\textbf{1}}=\{1,1,1,...,1\}$, and $\chi_o(n)=\frac{\vec{\textbf{1}}}{\sqrt{N}}$. Thus, $\chi_o(n)=\frac{\vec{\textbf{1}}}{\sqrt{N}}$, means $\chi_o(n)$ is constant on each vertex, and there is no deviation among any two vertices in $\chi_0(n)$. For this reason, $\chi_0(n)$ is considered as the least abnormal component of $\mathbf{G}$. Similarly, $\chi_{N-1}(n)$ is considered the most abnormal component of $\mathbf{G}$.

Fig \ref{fig:graph_G} shows an undirected graph $\mathbf{G_1}$, and each edge's weight is $1$. Fig \ref{fig:frequency1} shows  $\mathbf{G_1}$'s six eigenvectors distributions along each vertex. We can see, $\chi_0(l)$ is constant on very vertex, and has the smallest deviations along each edge. $\chi_5$ has the largest deviations, and the difference of $\chi_5(l)$ along each edge is larger than any other eigenvector on average.



\begin{figure}[t]
	\centering
	\subfigure[]{
		\includegraphics[height=1.6in] {figures/All-0.png}
		\label{fig:brazil1}
	}
	\subfigure[]{
		\includegraphics[height=1.6in] {figures/All-08.png}
		\label{fig:brazil2}
	}
		\subfigure[]{
		\includegraphics[height=1.6in] {figures/All-18.png}
		\label{fig:brazil3}
	}
	\subfigure[]{
		\includegraphics[height=1.6in] {figures/All-26.png}
		\label{fig:brazil4}
	}
    \subfigure[]{
		\includegraphics[height=1.6in] {figures/All-80.png}
		\label{fig:brazil5}
	}
	\subfigure[]{
		\includegraphics[height=1.6in] {figures/All-400.png}
		\label{fig:brazil6}
	}
	\caption{Spectral graph wavelet on South America. (a) vertex at which wavelets are centered in red dot. (b)-(f) wavelets, scales at 0.8, 1.8, 2.6, 8, and 40 respectively.}
	\label{fig:example2}
\end{figure}


\subsection{Global Anomaly Index}
\label{sec:signal_anomaly_on_Graph}

To quantify the anomaly of a vector $f$ defined on a graph $\mathbf{G}$, it's necessary to combine the intrinsic structures of $\mathbf{G}$ and $f$. As discussed above, $\hat{f}(l)$ represents the coefficient of frequency $\chi_l$, and $\hat{f}^2(l)$ is considered as the energy of frequency $\chi_l$. In addition, according to equation~\ref{eq:lambda2}, $\lambda_l$ represents the deviation of frequency $\chi_l$ along all the connected vertex. Therefore, in this paper, we define the anomaly Index of $\chi_l$ in $f$ as:
\begin{equation}
\label{eq:lambda3}
\gamma_f(l;\mathbf{G})=\lambda_l\hat{f}^2(l)= \lambda_l<f,\chi_l>^2
\end{equation}
$\gamma_f(l;\mathbf{G})$ depends on two parts, frequency $\chi_l$'s deviation sum $\lambda_l$, and its energy $\hat{f}^2(l)$. If the energy $\hat{f}^2(l)$ is small, even $\lambda_l$ is large, the anomaly Index of $\chi_l$ still might be small. Obviously, $\gamma_f(0;\mathbf{G})$ is always $0$ since $\lambda_0=0$. Further, we use the maximal value of $\gamma_f(l;\mathbf{G})$ to represent the global anomaly of $f$ on $\mathbf{G}$:
\begin{equation}
\label{eq:lambda4}
\gamma_f(\mathbf{G})=\underset{0 \leq l \leq N-1}{\max}{\gamma_f(l;\mathbf{G})}.
\end{equation}
Roughly speaking, $\gamma_f(l;\mathbf{G})$ means the anomaly extension of $\chi_l$ in $f$ defined on $\mathbf{G}$, in stead of meaning anomaly extension of vertex $v_l$.
For brevity, $\gamma_f(l;\mathbf{G})$  and $\gamma_f(\mathbf{G})$ are shortened as $\gamma_f(l)$ and $\gamma_f$, respectively, in some circumstance.


\begin{figure}[t]
	\centering
	\subfigure[$\mathbf{G_2}$]{
		\includegraphics[height=1.1in] {figures/f_on_g1.png}
		\label{fig:scale1}
	}
	\subfigure[$\mathbf{G_3}$]{
		\includegraphics[height=1.1in] {figures/f_on_g2.png}
		\label{fig:scale2}
	}
	\caption{$f=[1,2,5,2]$ on two graphs $\mathbf{G_2}$ and $\mathbf{G_3}$.}
	\label{fig:f_on_g}
\end{figure}

\begin{figure}[t]
	\centering
    {
		\includegraphics[width= 4in] {figures/new_graph.png}
		\label{fig:distribution2}
	}
	\caption{Anomaly index of $\mathbf{G_2}$ and $\mathbf{G_3}$.}
	\label{fig:new_graph}
\end{figure}




Figure \ref{fig:g1_gamma} plots the anomaly Index $\gamma_f(l)$ of $f_1$ on graph $\mathbf{G_1}$, where $f_1=[2,3,4,3,2,1]$. The six markers on the dashed blue are the six eigenvalues of $\mathbf{G}$. The yellow line is $|\hat{f}(l)|$, and the pink line is the anomaly Index, $\gamma_f(l)$ for frequency $\chi_l$. Because $\gamma_f(l)$ depends on both $\lambda_l$ and its power $\hat{f}^2(l)$, in the yellow line, even though $\chi_0$ has the strongest power,  while its deviation $\lambda_0 = 0$, thus  $\gamma_f(0)=0$. On the other hand, $\chi_5$ has the largest deviation; but its power $|\hat{f}(5)|^2$ is small, which makes $\gamma_f(5)$ is also small. Considering $\chi_4$ has a high deviation (eigenvalue) and a strong power of frequency, thus $\chi_4$  has the largest anomaly Index index. To compare the influence of different $f$ on anomaly index, we show an example in Fig \ref{fig:same_graph}. Set $f_1=[2,3,4,3,2,1]$ and $f_2=[2,2,-3,4,3,1]$, we plot their anomaly index $\gamma_{f}$ and energy $|\hat{f}(l)|$ respectively.
%For comparison, we plot both anomaly index and $|\hat{f}(l)|$ for $f_1$ and $f_2$  on $\mathbf{G_1}$, as shown in Fig \ref{fig:same_graph}, where $f_2=[2,3,4,3,2,1]$.
The light blue color stands for anomaly index, and yellow stands for $|\hat{f}(l)|$. The solid line stands for $f_1$, and dashed line stands for $f_2$. As we can see, for high frequency $\chi_l$, $f_1$ has a larger power than $f_2$, and hence a higher anomaly Index than $f_2$, where $\gamma_{f_1}=0.905$ and $\gamma_{f_2}=0.073$. This is consistent with that $f_1$ has larger deviations than $f_2$.

 As we discussed before, anomaly index depends on graph structure and $f$. Shown in \ref{fig:same_graph}, different $f$ might have very different anomaly index because the power of $\chi_l$ distribution is different. Similarly, even same signal $f$ on two different graphs might have very different anomaly index. Fig \ref{fig:f_on_g} shows two graphs with the same $f=[1,2,5,2]$. Fig~\ref{fig:new_graph} illustrates the anomaly index of $f$ on $\mathbf{G_2}$ and $\mathbf{G_3}$, where $\gamma_{f}(\mathbf{G_2})=0.073$ and $\gamma_{f}(\mathbf{G_3})=0.235$. {This is because in $\mathbf{G_3}$ there is not edge connecting $v_2$ and $v_3$, the different between $f(2)$ and $f(3)$ is not considered as anomaly.}


{\textbf{Remarks:}}
In this subsection, we introduce the anomaly index $\gamma_f(l;\mathbf{G})$ to measure the anomaly of $\chi_l$ in $f$ defined on $\mathbf{G}$ by combing the spectrum structure of $\mathbf{G}$ and $f$. $\gamma_f(l;\mathbf{G})$ depends on two parts: (1) the eigenvalue which reflects the deviations of $\chi_l$; (2) the $|\hat{f}(l)|^2$  which represents the power of $\chi_l$ in $f$. $\gamma_f(l;\mathbf{G})$ reflects the anomaly index of $\chi_l$ and not about the vertex $v_l$. We use the maximal value of $\gamma_f(l;\mathbf{G})$ to define the anomaly index of $f$ defined on $\mathbf{G}$, and it denotes the global anomaly index of $f$ on $\mathbf{G}$.


\subsection{Graph Wavelets}
\label{sec:graph_wavelet}
Classic wavelet is called mathematical microscope because of its capability of showing signal abnormality with different scales.
In the case of complex networks, graph wavelets render the graph with good localization properties both in frequency and vertex (i.e. spatial) domains. Their scaling property allows us to zoom in/out of the underlying structure of the graph.

%It is useful to analyze $f$ by taking into account the intrinsic geometric structure of the graph $\mathbf{G}$. In order to identify and exploit structure of  $f\in \mathbb{R}^N$, the spectral graph $\sigma({\mathcal{L}}):=\{\chi_l\}_{l=0}^{N-1}$ can be used as a dictionary of atoms~\cite{shuman_ACHA_2013}. Thus, $f$ can be decomposed as a linear combination of $\{\chi_l\}_{l=0}^{N-1}$ as
%\begin{equation}
%\label{eq:graph_fourier}
%f(n)= \sum\limits_{l=0}^{N-1}\hat{f}(l)\chi_l(n)
%\end{equation}
%, where
%\begin{equation}
%\label{eq:graph_fourier1}
%\hat{f}(l):= \sum\limits_{n=0}^{N-1}\chi^*_l(n)f(n)
%\end{equation}
%$\chi_l$ is called the Fourier frequency of $f(n)$ based on the graph $\mathbf{G}$, and $\hat{f}(l)$ is the corresponding Fourier coefficient.
%Equation~\ref{eq:graph_fourier1} and Equation~\ref{eq:graph_fourier} are called Fourier transform and inverse Fourier transform, respectively.
%Equation~\ref{eq:graph_fourier1} gives a clear representation of the Fourier components in $f(n)$.

Recall from Equation~\ref{eq:Graph_Fourier_Transform1}, the anomaly pattern $\hat{f}(l)$ presents the anomaly components of $f$ from the whole graph prospective. However, information concerning the vertex-location can not be identified from the Fourier transform. To address this issue, Hammond et al.~\cite{hammond2011wavelets} proposed constructing wavelet transforms of functions over the vertices using weighted graphs, described in the following steps:

\begin{enumerate}
\item Define a continuous generating kernel functions $g(x)$ on $\mathbb{R}^+$;
\item Then, select a central vertex $a \in {V}$ and scale $s$, set the frequency coefficients as $g(s\lambda_l)\chi^*_l(a)$ for each frequency component $\chi_l$;
\item Finally, sum up all those frequency components $\chi_l$.
\end{enumerate}
In this way, the graph wavelet at central vertex $a$ is constructed as:
\begin{equation}
\label{eq:graphwaveletdefinition}
\psi_{s,a}(n) = \sum\limits_{l=0}^{N-1}g(s\lambda_l)\chi_l^*(a)\chi_l(n)
\end{equation}
%After setting up the graph wavelet, the wavelet coefficients for $f$ can be defined as
%\begin{equation}
%\label{eq:graph_graphwavelet}
%W_f(s,a)=<\psi_{s,a}, f>=\sum\limits_{l=0}^{N-1}g(s\lambda_l)\hat{f}(a)\chi_l(n)
%\end{equation}
%\paragraph{\textbf{Properties}}


Similar to classical wavelets, graph wavelets provide following three properties, which are presented in detail in~\cite{hammond2011wavelets}.
 \begin{enumerate}
 \item \textbf{Reconstruction.}
 When generating the kernel function $g(x)$ satisfies the admissibility condition and $g(0)=0$,  $f(n)$ can be reconstructed by the wavelet coefficients.
\item \textbf{Discretization and Wavelet Frames} For practical applications, scale $s$ of graph wavelet $\psi_{s,a}$ should be sampled with a finite number of scales. Given a real valued function $h(x)$, satisfying
\begin{equation}
\hat{h}(\omega) = \sqrt{\int_\omega^\infty\frac{|\hat{g}(\omega')|^2}{\omega'}d{\omega'} }
\end{equation}
, where $\hat{g}$ and $\hat{h}$ are the classical Fourier transform of $g(x)$ and $h(x)$, the scaling function $\phi_{a}(n)$ can be generated as:
\begin{equation}
\label{eq:graphscaledefinition}
\phi_{a}(n) = \sum\limits_{l=0}^{N-1}h(\lambda_l)\chi_l^*(a)\chi_l(n)
\end{equation}
%Accordingly, the scaling coefficients are defined as
%\begin{equation}
%S_f(a)=<\phi_a,f>
%\end{equation}
Using scale set $\Theta:=\{s_j\}_{j=1}^J$, the discretized graph wavelet set $\{\psi_{s_j,a}\}_{j=1}^{J}$ $_{a=0}^{N-1}$, and scaling function set $\{\phi_a\}_{a=0}^{N-1}$ constitute a frame~\cite{hammond2011wavelets}.
So there will be $NJ$ wavelet coefficients in the frame.
According to frame theory~\cite{daubechies1992ten}, $f\in \mathbb{R}^N$ can be reconstructed by a limited number of scaling and graph wavelet coefficients. A detailed algorithm and treatment concerning the choice of $\Theta$ can be found in~\cite{hammond2011wavelets}.


\item \textbf{Localization in vertex domains}. Given a central vertex $v_a$ and its graph wavelet $\psi_{s,a}$, suppose the kernel function $g$ is $K+1$ times continuously differentiable, let $v_n$ be an vertex of $\mathbf{G}$ with $d_G(n,a)>K$, then there exist constants $D$ and $s_0$, such that
\begin{equation}
\label{equ:waveletbound}
\frac{\psi_{s,a}(n)}{||\psi_{s,a}||}\leq Ds_0
\end{equation} for all $s<s_0$ .
$d_G(n,a)$ is the shortest path distance, which is the minimum number of edges in any path that connect vertices $v_n$ and $v_a$~\cite{hammond2011wavelets}. Equation~\ref{equ:waveletbound} shows for any vertex $v_n$ that is far away from center vertex $v_a$ ($d_G(n,a)>K$), its wavelet value $\psi_{s,a}(n)$ is upper bounded by $Ds_0$. In other words, for vertex $v_n$ which is far away form vertex $v_a$, its wavelet value is linearly attenuated by scale $s$. When the scale $s$ is small, their wavelet value will be vanished quickly. Figure~\ref{fig:graphwaveletscale} shows two graph wavelets centered on the same vertex $v_a$, but with two different scales, $\psi_{s_1,a}$ and $\psi_{s_2, a}$, where $s_1<s_2$. The length of the black bar on each vertex denotes its graph wavelet value. The highlighted areas denote the kernel vertex with larger graph wavelet values.

Figure~\ref{fig:scale3} is $f$'s distribution along each vertex, whose value is denoted as the vertical bar. As we can see, with the graph wavelet pattern of $\psi_{s_2,a}$, $f$ shows a larger value on most of the kernel vertices, and $W_f(s_2,a)$ a large value as well. That also means $\psi_{s_2,a}$ optimally partitions the set $V$ into two distinct groups - kernel and marginal vertices. As shown in Figure~\ref{fig:scale4}, $W_f(s_3,a)$ is where the absenteeism is most significant, and $W_f(s_2,a)$ is  where bursty behavior is observed.
 \end{enumerate}
 
 

\begin{figure*}[t]
	\centering
	\subfigure[wavelet $\psi_{s_1,a}$]{
		\includegraphics[width=1.4in, height=1.1in] {figures/wavelet1.png}
		\label{fig:scale1}
	}
	\subfigure[wavelet $\psi_{s_2,a}$]{
		\includegraphics[width=1.4in, height=1.1in] {figures/wavelet2.png}
		\label{fig:scale2}
	}
\subfigure[$f(n)$ vs vertices]{
		\includegraphics[width=1.4in, height=1.1in] {figures/wavelet3.png}
		\label{fig:scale3}
	}
\subfigure[$W'_f(s,a)$ vs scale $s$]{
		\includegraphics[width=1.5in, height=1.1in] {figures/wavelet4.png}
		\label{fig:scale4}
	}
	\caption{Using graph wavelets for abnormal group identification.}
	\label{fig:graphwaveletscale}
\end{figure*}


% {\textbf{Remarks:}} Essentially, the wavelet frame is generated by kernel function $g(x)$ and scaling function $h(x)$ with $J$ different scales. Those functions are also called filter banks. Figure~\ref{fig:brazil_filter} shows the wavelet filter banks for Brazil graph which we will mention in the experiment part.

% \begin{figure}[h]
% 	\centering
%     {
% 		\includegraphics[width= 3.2in] {figures/brazil_filter.png}
% 		\label{fig:distribution2}
% 	}
% 	\caption{Wavelet filter banks for Brazil graph.}
% 	\label{fig:brazil_filter}
% \end{figure}



\subsection{Group Anomaly Detection via graph wavelet}
\label{sec:Group_Anomaly_Detection_via_graph_wavelet}
For a center node $a$, its kernel vertices denoted by $\mathcal{K}(a)$, is defined a set of vertices such that $d_G(n,a)\leq K$. All the other vertices are called as marginal vertices. One thing need to mention is that for different scale $s$, the $\mathcal{K}({a})$ is the same.
According to Equation~\ref{equ:waveletbound}, when $s$ is small, the weights of the marginal vertices are severely attenuated.
Essentially, $W_f(s,a)$ is equivalent to the sum of $f$ with large weights on kernel vertices, and small weights on marginal vertices.
%, and can also be treated as a similarity between $f$ and $\psi'_{s,a}$.
When $f$ is of uniformly large negative/positive value on kernel vertices, then $W_f(s,a)$ will be a large negative/positive value with scale $s$.
We call the wavelets with minimal and maximal $W_f(s,a)$ absenteeism wavelet and burst wavelet, respectively.


The localization property of graph wavelet makes it appropriate for group anomaly detection since it automatically identifies the kernel vertices from $V$.  These kernel vertices form a compact subset since each one of them is close to the same center vertex $v_a$, which avoids the compactness constrain condition, thus its complexity is greatly reduced. We propose our anomaly Group detection based on graph wavelet in Algorithm~\ref{algo:event_detection1}.



\begin{algorithm}[ht]
\centering
\captionsetup{font=scriptsize}
\caption{Anomaly Group Detection Using Graph Wavelet}
{\footnotesize \begin{algorithmic}[1]
\STATE {\bf Input:} graph and absenteeism score vector $\mathbf{G}(V,E;f^l)$ at time interval $l$, and wavelet threshold $\omega_{th}$.
\STATE {\bf Output:} anomaly burst group set $\mathcal{I}^{bur}$ and absenteeism group set $\mathcal{I}^{abs}$.	
\STATE{compute the spectral $\sigma{(\mathcal{L})}$ of graph $\mathbf{G}$};
\STATE{set the graph wavelets $\psi_{s,a}(n)$ and scales set $\{s_j\}_{j=1}^J$ for all $a\in V$};
\FORALL {$v_n\in V$ and $s_j \in \{s_j\}_{j=1}^J$}
	    \STATE{compute $W_f(s_j, a)$};
		\IF {$W_f(s_j, a) \ge \omega_{th}$}
		    \STATE{add group $\mathcal{K}(s_j,v_n)$ to $\mathcal{I}^{bur}$}
	    \ENDIF
	
		\IF {$W_f(s_j, a)\le -1*\omega_{th}$}
		    \STATE{add group $\mathcal{K}(s_j,v_n)$ to $\mathcal{I}^{abs}$}
	    \ENDIF	
	
\ENDFOR	
\RETURN {anomaly burst group $\mathcal{I}^{bur}$ and absenteeism group set $\mathcal{I}^{abs}$.}
\end{algorithmic}}
\label{algo:event_detection1}
\end{algorithm}


{\textbf{Remarks:}}
\begin{enumerate}
\item As graph wavelet and scaling functions form a frame, the function $f$ can be reconstructed by their coefficients.
As long as the scale level $J$ is high enough, $f$ can be well decomposed into the frame basis. Thus, using graph wavelets to exploit structure of functions defined on graphs is much more reasonable.
\item Graph wavelet transforms select vertices that are close to the central vertex $v_a$, and attenuate the impact of other marginal vertices that are far away from $v_a$.
Unlike other conventional methods,  $\psi_{s,a}$ is automatically scalable, and maintains the graph's topological information.
\item The graph wavelet does not introduce any objective functions and constraints. In addition, the scale $\{s_j\}_{j=1}^J$ set is numerically small, and once the eigenvalues and eigenvectors of $\mathbf{G}$ are known, the computation complexity of graph wavelet coefficients is $O(NJ)$, which makes is easily adaptable to wide variety of application scenarios.
\end{enumerate}


\subsection{Group Absenteeism Event Detection}
\label{sec:Group Absenteeism Event Detection}
In our real world, there are various scenarios that causes people's silent on the social media for a certain amount of time. Take Earthquake emergence for example, when earthquake is happening, people hardly get a chance to use social media and even the communication infrastructure are paralyzed, thus it will cause twitter uniformly Absenteeism. Some other examples, like flooding, blackout, and so on. We call this kind of disaster events as Absenteeism Event. When the disaster is over, and people's circumstance recovered, there follows some kind of burst in social media since it's natural for people to comment their recent experience. Usually, the severer of the disaster events, the stronger Absenteeism signal the social media may present when the even is happening, and the stronger burst signal will follow after that.

We propose a novel two-pass absenteeism based event detection algorithm. The underlying rationale of this algorithm is based on the following concepts.
\begin{enumerate}
\item As discussed in section~\ref{sec:graph_wavelet}, distribution of $f$ can be well reconstructed by the $J$ scaling and $NJ$ wavelet coefficients. Each of those normalized wavelet coefficients $W_f(s,a)$ represents a distribution pattern of $f$ on $\mathbf{G}$.
It is equivalent to saying that $\psi_{s,a}$ represents a special distribution pattern, which shares a large and uniform value around the central vertex with scale $s$.
\item When a significant event occurs, preceded by group absenteeism behavior in social networks, such as a severe earthquake, it is likely to be succeeded by a spike or burst in online user activity.
With this observation, we can represent an absenteeism behavioral pattern as $\psi_{s_l,a_l}$ at time $l$ centering at vertex $v_{a_l}$, and a burst related pattern as $\psi_{s_{\tau,a_\tau}}$ at time $\tau$ centering at vertex $v_{a_\tau}$. We assume the burst pattern happens within the time window size of $L$ after absenteeism pattern is identified. Further, a notion of response time can represented using the time difference $t_{rsp}=\tau-l$.
\item Both absenteeism and burst signal must show a strong correlation, especially if they occur in close proximity spatially and temporally.
%For instance, taking the power-cut-off for instance, usually only people who live in the affected area will ``yield at " this event a lot because it brings inconvenience to their life. However, people who live outside of the affected areas would hardly mention this event. Thus, to measure the correlation between absenteeism pattern and burst pattern is proposed as:
\begin{equation}
\label{eq:eventsimilarity}
\rho(\mathcal{K},\mathcal{K'}) = \frac{|\mathcal{K}\cap\mathcal{K'}|}{|\mathcal{K}|\cdot|\mathcal{K'}|}
\end{equation}
Based on these concepts, the higher the correlation, the higher probability that burst patterns is caused by the preceding group absenteeism. When $\rho$ is above the threshold (threshold is set at 0.5), we infer that an event occurred and that it evolved on social networks into distinct phases: first group absenteeism, followed by a spike or burst in user activity.
\end{enumerate}



\begin{algorithm}[t]
\centering
\captionsetup{font=scriptsize}
\caption{Two-Pass Absenteeism Event Detection}
{\footnotesize \begin{algorithmic}[1]
\STATE {\bf Input:} graph and absenteeism score vector $\mathbf{G}(V,E;f^l)$ at time interval $l$, and time window size $L$.
\STATE {\bf Output:} absenteeism event set $\mathcal{E}$.	
\STATE{compute burst group set $\mathcal{I}^{bur}$ by algorithm~\ref{algo:event_detection1}};
		    	\FORALL {$\tau$ from $l+1$ to $l+L$}
		    	    \STATE{compute absenteeism group set $\mathcal{I}^{abs}$ by algorithm~\ref{algo:event_detection1}};
		    	    \FORALL {$\mathcal{K}(a)\in \mathcal{I}^{bur}$ and $\mathcal{K}(b)\in \mathcal{I}^{abs}$}
		
		    	    \STATE{compute $W_{f^\tau}(s_k, b)$};
		    	    		
		    	    		    \IF {$\rho(\mathcal{K},\mathcal{K'})\ge \rho_{th}$}
		    	    		    \STATE{add absenteeism event $e\{a,b,\tau\}$ to $\mathcal{E}$}
		    	    	    	\ENDIF

                   \ENDFOR

	            \ENDFOR	
		


\RETURN {absenteeism event set $\mathcal{E}$}.
\end{algorithmic}}
\label{algo:event_detection}
\end{algorithm}
In summary, the two-pass absenteeism event detection can be summarized as shown in Algorithm~\ref{algo:event_detection}. Because the computation of the graph spectral graph $\sigma(L)$ is a one-time computation, the two-pass algorithm has $O(NJ)$ complexity.
\begin{enumerate}
\item Compute $\mathbf{G}$'s spectral graph $\sigma(L)$. Because $\sigma(L)$ is independent of the time interval, it is computed only once and can be solved by classical matrix factorization methods.
\item Set scale set $\{s_j\}_{j=1}^J$ according to algorithms in~\cite{hammond2011wavelets}, and compute the $NJ$ normalized graph wavelet coefficients $W'_f(s_j, a)$ at current time interval $l$.
\item Identify the most negative $W'_f(s_l, a_l)$, and determine the corresponding $\psi_{s_l,a_l}$ as the absenteeism pattern.
\item For all the time interval $\tau$ from $l+1$ to $l+L$, compute all the $W'_f(s_j, a)$, detect the most positive one and identify the corresponding graph wavelet $\psi_{s_\tau,a_\tau}$ as the burst pattern at time $\tau$, and the correlation score $\rho_{\tau}$ between absenteeism pattern $\psi_{s_l,a_l}$ and burst pattern $\psi_{s_\tau,a_\tau}$.
\item Detect the largest $\rho_{\tau}$ as $\rho_{max}$; return $\rho_{max}$ and response time $t_{rsp}$.
\end{enumerate}


\begin{figure}[h]
	\centering
    {
		\includegraphics[width= 3.3in] {figures/brazil_wavelet_small.png}
	}
	\caption{Graph Wavelets with center city $v_{83}$.}
	\label{fig:brazil_wavelet_small}
\end{figure}




\begin{figure}[th]
	\centering
	\subfigure[wavelet $\psi_{s_1,a}$]{
		\includegraphics[width=2in] {figures/Brazil_W_coeff_date31_s3.png}
		\label{fig:Brazil_W_coeff_date31_s3}
	}
	\subfigure[wavelet $\psi_{s_2,a}$]{
		\includegraphics[width=2in] {figures/Brazil_W_coeff_date31_s5.png}
		\label{fig:Brazil_W_coeff_date31_s5}
	}
	\caption{Graph Wavelet coefficient with scale $s_3$ and $s_5$.}
	\label{fig:graphwaveletcoefficient}
\end{figure}




\section{Experimental Results}
\label{sec:experiment}
This section discusses the application of our approach for detecting group anomalies. We begin by briefly describing the dataset used for our experiments in Section~\ref{sec:data_collection} and then move on to discussing the implementation details of how the graph $\mathbf{G}$ is assembled; we construct the graph wavelets $\psi_{s,a}$ in Section~\ref{sec:experimental_setup}. The following section presents the group anomaly detection performance for identifying protest events. In Section~\ref{sec:highlighted_results}, we describe three case studies that illustrate how the graph wavelet model is able to capture absenteeism events such as disaster scenarios.

\begin{figure}[t]
	\centering
	\subfigure[]{
		\includegraphics[width=2.8in,height=2.6in] {figures/brazil_graph_nn_5.png}
		\label{fig:brazil_graph_nn_5}
	}
	\subfigure[]{
		\includegraphics[width=2.8in,height=2.6in] {figures/brazil_zscore_31.png}
		\label{fig:brazil_zscore_31}
	}
	\caption{(a) Brazil's 5-nearest-neighbor graph: 5321 cities, where all edges' weights are $1$. (b) Brazil's z-score distribution on July 31, 2013. The color bar shows the scale of z-score.}
	\label{fig:knn_zscore}
\end{figure}


\subsection{Data Collection and Preprocessing}
\label{sec:data_collection}
The study described in this chapter uses tweets geolocated to Latin America and
collected over a period of two years
(Jan 2013 to Dec 2014).
We query Datasift's streaming API to collect tweets that also have meta-information including geotag bounding boxes (structured geographical coordinates), Twitter places (structured data), user profile location (unstructured, unverified strings), and `mentions information' about locations present in the body of the tweet.
Typically, we found that
the number of tweets with readily available geo-coordinates is too low for conducting meaningful experiments.
To circumvent this drawback, we use the geo-enrichment algorithm described in~\cite{ramakrishnan2014beating}.
This algorithm uses a gazetteer-based approach to look-up location names and geo-coordinates.
To identify location-specific tweets, we configure the geocoding tool to first consider the tweet's text for mentions of place names and geographical landmarks (e.g., say, Plaza de la Independencia (Quito, Ecuador)).
In cases when no geographical location was found in the tweet text, it then proceeds to process the geographical coordinates and the self-reported location string in user's profile metadata.


%To prepare a dataset of ground truth events for our study, we focused on specific types of disruptive societal events, such as natural disasters.
%We assume that such events are the predominant reasons that can cause group absenteeism on social networks.
%To discern when major events occurred, we retrieved records of natural disaster related events involving earthquakes, floods, and landslides from European Emergency Response Coordination Center(ERCC)~\footnote{http://erccportal.jrc.ec.europa.eu/} and World Top Stories Timeline~\footnote{http://www.mapreport.com/}.



\subsection{Experimental Setup}
\label{sec:experimental_setup}

\paragraph{Graph Setup}
Each city $v_i$'s location is represented by its geographical coordinate pair $lat_i$ and $lon_i$. Instead of using the real physical distance, we define the distance of any two cities $v_i$ and $v_j$ as $d_{ij}=\sqrt{(lat_i-lat_j)^2+(lon_i-lon_j)^2}$. We setup graph $G$ as a $k$ neighbors graph, which means  each city is only  connected to its $k$-nearest-neighbors. In this chapter, we set $k=5$, and all the edges' weights  in $G$ are 1. Figure~\ref{fig:brazil_graph_nn_5} shows Brazil's $5$ nearest-neighbor graph with 5321 cities.

\begin{figure}[t]
	\centering
    {
		\includegraphics[width= 4in] {figures/kernel_function.png}
		\label{fig:kernel_function}
	}
	\caption{Kernel function g(x).}
	\label{fig:kernel_function}
\end{figure}


\paragraph{Absenteeism Score}
Considering that
the tweet volume $X$ varies vastly among cities, instead of using $X$ itself, we use the normalized value of z-score as absenteeism score, which is defined as:
\begin{equation}
\label{eq:Z_score}
\textrm{z-score} = \frac{X-\mu}{\sigma}
\end{equation}where $\mu$ is the mean value of the previous $30$ day tweets volume and $\sigma$ is the corresponding standard deviation. As shown in Figure~\ref{fig:brazil_zscore_31}, different
node colors denote different z-score values.



\begin{figure}[t]
	\centering
	\subfigure[wavelet $\psi_{s_1,a}$]{
		\includegraphics[width=2.8in] {figures/s1.png}
		\label{fig:Brazil_s1}
	}
	\subfigure[wavelet $\psi_{s_2,a}$]{
		\includegraphics[width=2.8in] {figures/s2.png}
		\label{fig:Brazil_s2}
	}
	\caption{Graph wavelets with center city $v_{83}$. $s_1$ = 1.31, $s_2$ = 0.68.}
	\label{fig:graphwaveletscale}
\end{figure}

\begin{figure}[th]
	\centering
	\subfigure[$W_f(s_1,a)$]{
		\includegraphics[width=2.4in,height=2.2in] {figures/wavelet_coeff_s1.png}
		\label{fig:Brazil_W_coeff_date31_s3}
	}
	\subfigure[$W_f(s_2,a)$]{
		\includegraphics[width=2.4in,height=2.2in] {figures/wavelet_coeff_s6.png}
		\label{fig:Brazil_W_coeff_date31_s5}
	}
	\caption{Graph wavelet coefficient $W_f(s_1,a)$ and $W_f(s_2,a)$.}
	\label{fig:graphwaveletcoefficient}
\end{figure}


\paragraph{Kernel function $g(x)$ and scaling function $h(x)$}
Our choice for the wavelet generating kernel function, $g(x)$, and scaling function $h(x)$ is motivated by our goal to achieve scale-dependent localization. We follow the kernel function setting in~\cite{hammond2011wavelets}, which behaves as a monic power near the origin, and has power law decay for large x, as shown in Figure~\ref{fig:kernel_function}. $g(x)$ and $h(x)$ are set as:
\begin{equation}
g(x) = \left\{ \begin{array}{rl}
 x &\mbox{ for $x<1$} \\
s(x) &\mbox{ for $1\leq x \leq 2$} \\
 2x^{-1} &\mbox{ for $x>2$} \\
       \end{array} \right.
\end{equation} where $s(x)=-5+11x-6x^2+x^3$.
\begin{equation}
h_{x}= 1.385\, exp(-(\frac{20x}{0.6\lambda_{max}})^4)
\end{equation}
The scale set $\{s_j\}_{j=1}^J$ is selected to be equally logarithmically spaced between the minimum and maximum scales $s_1$ and $s_J$, which are defined in~\cite{hammond2011wavelets}. We set $J=6$ in the experiment. Figure~\ref{fig:graphwaveletscale} shows two different scaled wavelets on Brazil's $5$-nearest-neighbor graph. Comparing Figure~\ref{fig:Brazil_s1} with Figure~\ref{fig:Brazil_s2}, we can see that, when scale increases, more cities (with deeper color) are selected. 
Figure~\ref{fig:graphwaveletcoefficient} shows the corresponding wavelet coefficients.
We also try another kernel function, i.e. the Mexican hat function, and find that as long as the kernel function monotonicity is the same,
the differences in wavelet coefficients are negligible.



\paragraph{Anomaly index $\gamma_f(G)$ and $\omega_{th}$}
We claim that the event frequency $\eta$ is linear to $\gamma_f(\mathbf{G})$, described as
\begin{equation}
\label{eq:linear_equation}
\eta = k_0*\gamma_f(\mathbf{G}) + k_1
\end{equation}We use historical data to train $k_0$ and $k_1$ by least square error criterion. Once we know
$k_0$ and $k_1$, given a new $\gamma_f'(\mathbf{G})$, the event number is estimated as $m=\left \lceil \eta' \right \rceil$. Subsequently the
threshold $\omega_{th}$ is set as the $m_{th}$ largest $W_f(s_j,a)$, for all $a\in V$, $0\le j \le J$.


\begin{table}[bt] %!htp
\renewcommand{\arraystretch}{1.1}
\caption{\label{table:models_compare} The performance of graph wavelet vs. baseline and Z-score.}
\scriptsize
\centering
\begin{tabular}{ l | l |l | l | l}
\hline
\textbf{Country} & \textbf{Method}& \textbf{Precision}  & \textbf{Recall}  & \textbf{F-measure} \\
\hline
Brazil & Baseline & 0.052 &0.104 & 0.060\\
       & Z-score & 0.117&0.307 & 0.159 \\
 & Graph wavelet& 0.404 &0.262 & 0.292 \\
\hline
Mexico & Baseline & 0.074 &0.124 & 0.090 \\
       & Z-score & 0.221 &0.147 & 0.168 \\
 & Graph wavelet& 0.397 &0.384 & 0.408 \\
\hline
Venezuela & Baseline & 0.078 &0.053 & 0.059 \\
       & Z-score & 0.197 &0.197 & 0.189 \\
 & Graph wavelet& 0.292 &0.554 & 0.355 \\
\hline
\end{tabular}
\end{table}



\subsection{Performance}
The data for this experiment was gathered for three countries experiencing major protest events, namely Brazil, Mexico and Venezuela, from Jan 2013 to Dec 2014. Taking the Gold Standard Report (GSR)~\cite{ramakrishnan2014beating} as representing ground truth, we applied our new graph wavelet approach as follows. For each day, we determine whether there is any anomaly. If there is, we
identify the group of anomalous cities and compare this set
with the GSR to determine if the selected cities actually experienced protest events on that day and thus show how many of the model's predictions matched the ground truth and how many did not. We use recall, precision, and the F-measure to evaluate the model's performance. To evaluate the effectiveness of our new graph wavelet approach, we also compared the results with those obtained using intuitive approaches such as frequency based random assignment, referred to here as the baseline model, and z-score based selection methods. The baseline model was built according to the historical protest records for each city and thus the model's predictions of the future occurrence of protests were based on frequency. The z-score approach entails
selecting the group of cities whose z-score crosses some threshold, say $|z-score|>3$.



\begin{figure}[h]
	\centering
	\subfigure[Brazil]{
		\includegraphics[width= 4.5in, height=1.7in] {figures/performance_compare_bar_graph_brazil.png}
		\label{Brazil_performance}
	}
	\hfill
	\subfigure[Mexico]{
		\includegraphics[width= 4.5in, height=1.7in] {figures/performance_compare_bar_graph_mexico.png}
		\label{Mexico_performance}
	}
	\hfill
	\subfigure[Venezuela]{
		\includegraphics[width= 4.5in, height=1.7in] {figures/performance_compare_bar_graph_venezuela.png}
		\label{Venezuela_performance}
	}
	\caption{Brazil, Mexico and Venezuela protest detection performance.}
\label{fig:threecountry_performance}
\end{figure}


We compared the performance of these three models over the two year test period; the overall results are shown in Table~\ref{table:models_compare}. Generally speaking, the new graph wavelet approach exhibited better precision, recall, and F-measure scores than the baseline model across all three countries. The mean F-measure for the graph wavelet detection across models and countries is greater than that achieved by either of the other prediction models. Interestingly, the graph wavelet approach appears to operate at different efficiency levels for each country. From Figure~\ref{fig:threecountry_performance} we can see that the graph wavelet model has a much higher recall in Venezuela than in Brazil, and an inferior quality of event detection in Mexico compared to Venezuela.



\subsection{Case Studies}
\label{sec:highlighted_results}

\begin{figure}[h]
	\centering
	\subfigure[]{
		\includegraphics[width=1.4in,height=1.8in] {figures/Chile_absent_zscore_3.png}
		\label{fig:absent_Chile_score}
	}
	\hfill
	\subfigure[]{
		\includegraphics[width=1.4in,height=1.8in] {figures/Chile_absent_wavelet_3.png}
		\label{fig:absent_Chile_wavelet}
	}
	\hfill
	\subfigure[]{
		\includegraphics[width=1.4in,height=1.8in] {figures/Chile_burst_zscore_3.png}
		\label{fig:burst_Chile_score}
	}
	\hfill
	\subfigure[]{
		\includegraphics[width=1.4in,height=1.8in] {figures/Chile_burst_wavelet_3.png}
		\label{fig:burst_Chile_wavelet}
	}
	\subfigure[]{
		\includegraphics[width=2.8in,height=1.6in] {figures/earthquake_example_10min_circle.png}
		\label{fig:earthquake}
	}
	\subfigure[]{
		\includegraphics[width=2.8in,height=1.6in] {figures/earthquake_cloud.png}
		\label{fig:earthquake-cloud}
	}
	\caption{Iquique Earthquake, Chile. (a-d) plots show differences in distributions of absenteeism score and wavelet coefficients calculated at 8:45 PM, April 1, 2014 (a-b) involving group absenteeism and later when burst in activity is captured at 11:00 AM, April 2, 2014 (c-d), respectively; (e) Tweet time series for Iquique on April 1, 2014; (f) Word cloud of tweets which mention `Iquique'.}
\label{fig:case1_wavelet}
\end{figure}



\textbf{Case study 1: Iquique Earthquake, Chile.}
On April 1, 2014 at around 8:46 PM (local time) a large earthquake struck off the coast of Chile, northwest of the port city of Iquique. We show the distribution of absenteeism scores and normalized wavelet coefficient values of the graph wavelets from the beginning of this event and throughout the subsequent 24 hours period. As shown in Figure~\ref{fig:absent_Chile_score}, we can clearly see absenteeism behavior, where the scores are dominated by very low (blue spectrum) z-score values (indicating high absenteeism). Likewise, Figure~\ref{fig:absent_Chile_wavelet} depicts low coefficient values for the northern regions of Chile, where the impact of the earthquake was most significant. As the news of earthquake spread throughout the next day, user activity on social media increased. This bursty behavior is clearly visible on April 2nd, at around 11:00 AM. Figure~\ref{fig:burst_Chile_score} shows that the z-scores increase (red spectrum) significantly and the coefficient value distribution (Figure~\ref{fig:burst_Chile_wavelet}) of the graph wavelets for northern regions of Chile are also in the red spectrum. The graph wavelet distributions in Figures~\ref{fig:absent_Chile_wavelet}~\ref{fig:burst_Chile_wavelet} show that the kernel area of the absenteeism/burst wavelets cover most large negative/positive values. In this way, the wavelets identify the abnormal negative/positive groups in absent/burst time intervals, respectively. Furthermore, a high correlation score of 0.726 was calculated for the wavelets from absenteeism and bursty periods of this episode, indicating a strong connection between the burst in activity and the previously observed absenteeism, signaling an event was detected.



\begin{figure}[h]
	\centering
	\subfigure[]{
		\includegraphics[width=1.4in, height=1.5in] {figures/Venuze_absent_zscore_3.png}
		\label{fig:absent_Venezuela_score}
	}
	\subfigure[]{
		\includegraphics[width=1.4in, height=1.5in] {figures/Venuze_absent_wavelet_3.png}
		\label{fig:absent_Venezuela_wavelet}
	}
	\subfigure[]{
		\includegraphics[width=1.4in, height=1.5in] {figures/Venuze_burst_zscore_3.png}
		\label{fig:burst_Venezuela_score}
	}
	\subfigure[]{
		\includegraphics[width=1.4in, height=1.5in] {figures/Venuze_burst_wavelet_3.png}
		\label{fig:burst_Venezuela_wavelet}
	}
	\subfigure[]{
		\includegraphics[width=2.8in,height=1.3in] {figures/Veneuela_power_count_all.png}
		\label{fig:power}
	}
	\subfigure[]{
		\includegraphics[width=2.8in,height=1.3in] {figures/power-English.png}
		\label{fig:power-cloud}
	}
	\caption{Power Outage in Venezuela. (a-d) plots show differences in distributions of absenteeism score and wavelet coefficients calculated at 7:40 PM, December 2, 2013 (a-b) involving group absenteeism and later when burst in activity is captured at 8:45 PM in the same day (c-d), respectively; (e) Time series of tweets volume on December 2, 2013; (f) Word cloud of tweets mentioning `Caracas'.}
\label{fig:case2_wavelet}
\end{figure}


The graph wavelets generated during the absenteeism time period Figure~\ref{fig:absent_Chile_wavelet} have a central node located in the city of `Iquique'. Looking at the time series (Figure~\ref{fig:earthquake}) of Twitter activity for Iquique and the associated word clouds (Figure~\ref{fig:earthquake-cloud}), we can see how events unfolded during the course of the earthquake. Strong absenteeism is observed from 8:45 PM to 9:20 PM. Examining user mobility via their geotagged tweets from the city of Iquique, on April 1, 2014, the user mobility fraction had increased by 15.4\%.


\textbf{Case Study 2: Massive power outage in Venezuela.}
A massive power outage in Venezuela plunged several major cities, including the capital city Caracas, into darkness around 7:40 PM (local time) on December 2, 2013.
News media reported\footnote{http://www.usatoday.com/story/news/world/2013/12/\hskip0ex 02/\hskip0ex power-failure-caracas-venezuela/3823327/}, that the power outage lasted for 10-15 minutes, and the people of Caracas quickly took to the streets to protest.
This action at the beginning of the episode coincides with the absenteeism period detected by our algorithm.
The scatter plots showing the distribution of absenteeism scores and wavelet coefficients (Figures~\ref{fig:absent_Venezuela_score},~\ref{fig:absent_Venezuela_wavelet}) indicate that most of the low values are less than $0$.
Shortly after the absenteeism, we detected a huge burst in activity around 8:45 PM, signaled by the increased z-scores (low absenteeism) and coefficient values (Figures~\ref{fig:burst_Venezuela_score},~\ref{fig:burst_Venezuela_wavelet}). A correlation score of 0.617 was calculated when comparing the graph wavelets from the absentee and burst periods.


The absenteeism related graph wavelets indicate that the city of Caracas was the central node. Taking a close look at the Twitter volume and tweets from Caracas and surrounding cities, there is a sharp decline in user activity around 7:40 PM and then a huge spike starting at 8:45 PM. The word clouds for the tweet content show a very similar story, with dominant words being `light' and `blackout'; the Spanish phrase `sin luz', which means `no light', became a trending hashtag \#sinluz on Twitter.



\begin{figure}[h]
	\centering
	\subfigure[]{
		\includegraphics[height=1.5in] {figures/Argentina_absent_zscore_3.png}
		\label{fig:absent_Argentina_score}
	}
	\subfigure[]{
		\includegraphics[height=1.5in] {figures/Argentina_absent_wavelet_3.png}
		\label{fig:absent_Argentina_wavelet}
	}
	\subfigure[]{
		\includegraphics[width=2.6in, height=1.4in] {figures/holiday-cloud.png}
		\label{fig:holiday-cloud}
	}
	\caption{The Christmas Day in Argentina: (a-b) plots show distributions of (a) absenteeism score and (b) wavelet coefficients calculated on December 25, 2013; (c) Time series comparing absenteeism score and user mobility corresponding to tweets between December 5 - 25, 2013.}
\label{fig:case3_wavelet}
\end{figure}


\textbf{Case Study 3: Christmas Day.}
As noted earlier, absenteeism behavior may not always lead to a spike in activity. For example, our model detected strong absenteeism in social media activity for major holidays such as Christmas Day that was not followed by a bursty period in Twitter activity. This is likely because people tend to travel to visit family during the holidays. This is supported by low values of z-scores or high absenteeism in Figure~\ref{fig:absent_Argentina_score} and wavelet coefficients in Figure~\ref{fig:absent_Argentina_wavelet} with respect to Argentinian tweets on December 25, 2013. Hence, no subsequent burst period was detected for this event. Interestingly as Christmas Day approached, Figure~\ref{fig:holiday-cloud} shows that user mobility gradually increases and the z-score decreases, signaling greater absenteeism. We used Pearson's correlation coefficient to measure the two time series and found a correlation score of -0.94.





%%%%%%%%%%%%%%%%%%%%%%%%%%%% end %%%%%%%%%%%%%%%%%%%%%%%%%
%\begin{table*}[th] %!htp
% \renewcommand{\arraystretch}{1}
% \caption{\label{table:list_events} Selected major events in South America countries}
% \scriptsize
% \centering
% \begin{tabular}{ p{0.5cm}| p{2cm} | p{2.2cm} | p{2.2cm} | p{2.2cm} | p{2.5cm} | p{3cm} }
%  \hline
%  \textbf{No.} & \textbf{Events}& \textbf{ Absenteeism } & \textbf{Response time} & \textbf{Correlation}&\textbf{Central location} \\ [1ex]
%  \hline
%        1& Earthquake & 8:45 PM & 3 hours & 0.73 &Iquique, Chile\\
%        2& Blackout & 7:40 PM & 1 hour & 0.81& Caracas, Venezuela \\
%        3& holiday & one day & 2 days & 0.33 & \\        \hline
% \end{tabular}
%\end{table*}


%Public holidays are typical events causing group absenteeism. One of the most dominant reason is, during public holidays, especially long-time holiday, people tend to travel, which resulting in a high level of local user mobility, and users' mobility will cause Twitter absenteeism accordingly.
%We calculating more cases Pearson's correlation, and plot their distribution in Figure~\ref{fig:pearson}, of which the median value of correlation score is -0.88, and the average correlation score is -0.79. We can see the user mobility plays a forceful role in influencing Twitter absenteeism.

%\subsection{Performance}
%We use the data set on February 27, 2014, and set the time window as one day. We plot the comparison results from two aspects:  running time complexity, and parameter sensibility in figure~\ref{fig:performance},~\ref{fig:running_time},~\ref{fig:sensibility}.
%\begin{figure}[ht]
%	\centering
%	\subfigure[matrix]{
%		\includegraphics[width=1.55in,height=1in] {figures/performance1.png}
%		\label{fig:performance1}
%	}
%	\subfigure[graph]{
%		\includegraphics[width=1.55in,height=1in] {figures/performance2.png}
%		\label{fig:performance2}
%	}
%	\caption{running time vs input parameter.}
%	\label{fig:performance}
%\end{figure}
%\paragraph{Running time}From Figure~\ref{fig:performance}, we can see that the running time of minimal matrix approach increases extremely fast when $A$ is larger than 0.09. While in graph wavelet approach, the increasing speed is much stable as $d_{th}$ increases. This is because minimal matrix approach's time complexity is in proportion to $A^2$, while graph wavelet approach's time complexity is proportional to $d_{th}$. From Figure~\ref{fig:running_time}, we can see clearly that for the minimal matrix algorithm, the running time complexity also increase sharply with the input size $n$, while in graph wavelet approach, the increase speed is moderate. This is because the minimal matrix approach's timing complexity is $O(N^3)$, while graph wavelet approach's time complexity is $O(N^2)$. Thus, the graph wavelet approach is better  than minimal matrix approach in term of running time for a larger absenteeism group.
%\begin{figure}[h]
%	\centering
%	\subfigure[matrix]{
%		\includegraphics[width=1.55in,height=1in] {figures/running_time1.png}
%		\label{fig:running1}
%	}
%	\subfigure[graph]{
%		\includegraphics[width=1.55in,height=1in] {figures/running_time2.png}
%		\label{fig:running2}
%	}
%	\caption{Running time vs input size.}
%	\label{fig:running_time}
%\end{figure}
%\paragraph{Parameter sensibility}In minimal matrix approach, set the input parameter as $A$, and the optimal absenteeism group as $P_{min}$. When $A$ is changed to $A$', the optimal absenteeism group is changed to $P_{min}$, define the output error as the city number that exists in $P_{min}$ but not in $P_{min}'$, and denoted as $P_{min}-P_{min}'$. We define the parameter sensibility as: $$sensibility=\frac{{|P_{min}-P_{min}'|}/{|P_{min}|}}{|A-A'|/{A}}.$$ We plot the minimal matrix approach and graph wavelet approach's sensibility in Figure~\ref{fig:sensibility}. In the minimal matrix approach, when the input parameter error is smaller than 20\%, the output absenteeism group error is less than 5\%. While in the graph wavelet approach, the output absenteeism group error is linear to the input error parameter. This is probably because minimal matrix approach aggregates all the absenteeism score covered by the region, and usually has a much larger city number than the graph wavelet approach, and makes minimal matrix approach better at anti-noise.  All in all, the minimal matrix algorithm focuses on all the cities in the cover group, and has a better global performance at anti-noise, while is inferior to the graph wavelet counterpart in term of running time complexity.
%\begin{figure}[h]
%	\centering
%	\subfigure[matrix]{
%		\includegraphics[width=1.55in,height=1in] {figures/sensibility1.png}
%		\label{fig:sensibility1}
%	}
%	\subfigure[graph]{
%		\includegraphics[width=1.55in,height=1in] {figures/sensibility2.png}
%		\label{fig:sensibility2}
%	}
%	\caption{Sensibility comparison of the two algorithms.}
%	\label{fig:sensibility}
%\end{figure}




% % % % % % % % % % % % % the end% % % % % % % %
%
%\begin{figure}[ht]
%	\centering
%	\subfigure[]{
%		\includegraphics[width=1.55in,height=1in] {figures/Curitiba-Brazil1-cloud.png}
%		\label{fig:holiday}
%	}
%	\subfigure[]{
%		\includegraphics[width=1.55in,height=1in] {figures/pearson1.png}
%		\label{fig:pearson}
%	}
%	\caption{(a) User mobility time series and corresponding absenteeism score, from Dec 5, 2013 to Dec 25, 2013. (b) Pearson correlation score distribution of user mobility and absenteeism score.}
%\label{fig:case3}
%\end{figure}

%Now we show experimental results of our algorithm on highlighted case studies (see Table~\ref{table:list_events}):
%
%\begin{table*}[th] %!htp
%	\renewcommand{\arraystretch}{1}
%	\caption{\label{table:list_events} Selected major events in South America countries}
%	\scriptsize
%	\centering
%	\begin{tabular}{ p{0.5cm}| p{1.5cm} | p{8cm} | p{1.5cm}}
%		\hline
%		\textbf{No.} & \textbf{Date}& \textbf{ Events} & \textbf{Test Areas}   \\ [1ex]
%		\hline
%        1& 2013-06-17 & Brazilian Spring: Protests in over 100 cities, over 2 million people & Brazil  \\		
%        2& 2013-12-02 & Power cut leaves much of Venezuela without electricity & Venezuela \\
%        3& 2013-12-24 & Floods, more than 50,000 people are forced to flee their homes & Brazil\\
%        4& 2013-12-25 & Christmas holiday  & Argentina \\
%        5& 2013-12-30 & Power supply disrupted in heatwave in Buenos Aires, Argentina & Argentina \\
%        6& 2014-04-01 &  M8.2 earthquake struck off the coast of Chile, epicenter is Iquique & Chile  \\
%        7 & 2014-05-21 & Bus strike paralyzes Brazil's biggest city as World Cup looms & Brazil \\			\hline
%	\end{tabular}
%\end{table*}


%The wavelet scales $t_j$ are selected to be logarithmically equispaced between the minimum and maximum scales $t_J$ and $t_1$, with the upper bound $\lambda_{max}$ of the spectrum of $L$. The placement of the maximum scales $t_1$ as well as the scaling function kernel $h$ will be determined by the selection of $\lambda_{min}=\frac{\lambda_{min}}{K}$, where $K$ is a design parameter of the transformation. We then set $t_1$ so that $g(t_1x)$ has power-law decay for $x>\lambda_{min}$, and set $t_J$ so that $g(t_Jx)$ provides monotonicity of the polynomial for $x < \lambda_{max}$. This is achieved by $t_1=\frac{x_2}{\lambda_{min}}$, and $t_J=\frac{x_2}{\lambda_{max}}$. For the scaling function kernel we take $h(x)=\gamma exp(-({{\frac{x}{\lambda_{min}}}})^4)$, where $\gamma$ is set such that $h(0)$ has the same value as the maximum value of $g$.
%
%At each time point the graph has an Z-score vector $f$ and when we get the lowest value for function $< \psi_{t,n},f>$ the corresponding wavelet $\psi_{t,n}$ is reported as an absenteeism pattern.



\section{Discussion}
\label{sec:conclusion}
Previous research has demonstrated the importance of burst detection in Twitter. In this study, we argue that group absenteeism can also be vital for detecting disruptive societal events. Modeling absenteeism is crucial because it can serve as a surrogate signal for event detection. For example, in the case of the Iquique earthquake, our new algorithm detected absenteeism behavior on Twitter that was closely followed by a spike in user activity. Unlike traditional event detection methods, which identify real time events only after they have occurred because the burst signal must first be identified, an absenteeism signal can be observed much earlier, thus providing greater foresight into future events. This means that our proposed approach offers a significant advantage over current strategies that focus solely on modeling spike or burst related patterns for event detection.

%Disruptive events which cause Twitter absenteeism, but also render burst detection methods less useful. In the case of the Natal protest event, a large portion of people were walking on the street to protest, and the city's tweet absenteeism score reached a minimum. During the Brazil floods, the tweets tended to become inactive as the severity of the floods increased. It reached the lowest point when the flood was at its worst. In these two cases in particular, using a burst signal alone it can be difficult to identify such events.

Existing approaches for event detection also suffer from an inherent latency in their detection process. This is because they are based on the use of bursty signals from abnormal activity on social networks, but miss the absenteeism signal that often precedes these bursts. Our approach addresses this shortcoming by successfully modeling the `lull before the storm'. In this study we defined an absenteeism score for groups of cities within the Twitter network and apply it to construct wavelet transforms that not only
detect anomalous subgraphs (including both burst and absenteeism groups) at different scales, but can also be used to identify the geographical focal point of the anomaly. This localization property of graph wavelets guarantees that the selected groups are compact automatically. The identified abnormal groups have been verified using real-world datasets and proven to be indicative of
events such as civil protests or natural disasters.

%In future work, we plan to extend our detection model to capture extent of an event's influence over network. Another interesting extension of our work would be to include absenteeism as feature to classify event of different nature (disruptive vs non-disruptive).



\endgroup
