%
% PROJECT: <ETD> Electronic Thesis and Dissertation Initiative
%   TITLE: LaTeX report template for ETDs in LaTeX
%  AUTHOR: Neill Kipp, nkipp@vt.edu
%     URL: http://etd.vt.edu/latex/
% SAVE AS: etd.tex
% REVISED: September 6, 1997 [GMc 8/30/10]
%

% Instructions: Remove the data from this document and replace it with your own,
% keeping the style and formatting information intact.  More instructions
% appear on the Web site listed above.

\documentclass[12pt,dvips]{report}

\setlength{\textwidth}{6.5in}
\setlength{\textheight}{8.5in}
\setlength{\evensidemargin}{0in}
\setlength{\oddsidemargin}{0in}
\setlength{\topmargin}{0in}

\setlength{\parindent}{0pt}
\setlength{\parskip}{0.1in}

% Uncomment for double-spaced document.
% \renewcommand{\baselinestretch}{2}

% \usepackage{epsf}
\usepackage[pdftex]{graphicx}
\usepackage{import}

\usepackage{subfigure}
\usepackage{mathrsfs}
\usepackage[normalem]{ulem}
\usepackage{array}
%\usepackage[lined,boxed,commentsnumbered]{algorithm2e}
\usepackage{amsmath}
\usepackage{amsfonts}
\usepackage{caption}
\usepackage{algorithm}
\usepackage{algorithmic}
\usepackage{multirow}



\usepackage[utf8]{inputenc}
\usepackage[english]{babel}

\newcommand{\argmin}{\arg\!\min}
\newcommand{\argmax}{\arg\!\max}


\renewcommand{\baselinestretch}{1}
\begin{document}

\thispagestyle{empty}
\pagenumbering{roman}
\begin{center}

% TITLE
{\Large
%Four Problems in Information Diffusion \\ over Social Networks
Algorithms for Modeling Mass Movements and their Adoption \\ in Social Networks
}
%Temporospatial Information Propagation in Social Networks
%Information Transition, Propagation and Absenteeism in Social Networks
%Event Classification, Detection and Prediction based on Information Diffusion in Social Networks


\vfill

Fang Jin

\vfill

Dissertation submitted to the Faculty of the \\
Virginia Polytechnic Institute and State University \\
in partial fulfillment of the requirements for the degree of

\vfill

Doctor of Philosophy \\
in \\
Computer Science and Applications

\vfill

Naren Ramakrishnan, Chair \\
Chang-Tien Lu \\
Chris North \\
Yang Cao \\
Feng Chen

\vfill

June 2, 2016\\
Arlington, Virginia

\vfill

Keywords: Mass Movements, Group Anomaly, Event Detection, Information Propagation, Social Networks
\\
Copyright 2016, Fang Jin

\end{center}

\pagebreak

\thispagestyle{empty}
\begin{center}

{\large
%Four Problems in Information Diffusion over Social Networks
Algorithms for Modeling Mass Movements and their Adoption \\ in Social Networks
}

\vfill

Fang Jin

\vfill

(ABSTRACT)
\vfill

\end{center}

Online social networks have become a staging ground for many modern movements, with the Arab Spring being the most prominent example. In an effort to understand and predict those movements, social media can be regarded as a valuable social sensor for disclosing underlying behaviors and patterns. To fully understand mass movement information propagation patterns in social networks, several problems need to be considered and addressed. Specifically, modeling mass movements that incorporate (i) multiple spaces (ii) a dynamic network structure (iii) misinformation and (iv) a swift outbreak/slowly evolving transmission can be exceptionally useful in understanding information propagation in social media.

This dissertation explores four research problems underlying efforts to identify and track the adoption of mass movements in social media. First, how do mass movements become mobilized on Twitter, especially in a specific geographic area? Second, can we detect protest activity in social networks by observing group anomalies in graph? Third, how can we distinguish real movements from rumors or misinformation campaigns? and Fourth, how can we infer the indicators of a specific type of protest, say climate related protest?

A fundamental objective of this research has been to conduct a comprehensive study of how mass movement adoption functions in social networks. For example, it may cross multiple spaces, evolve with dynamic network structures, or consist of swift outbreaks or long term slowly evolving transmissions. In many cases, it may also be mixed with misinformation campaigns, either deliberate or in the form of rumors. Each of those issues requires the development of new mathematical models and algorithmic approaches such as those explored here.  It is my hope that this work will facilitate advances in information propagation, group anomaly detection and misinformation distinction and, ultimately, help to improve our understanding of mass movements and their adoption in social networks.


%As social media (e.g., Twitter) continues to increase in popularity, it is becoming employed
%as a social sensor into real-world events. Most current research focuses on
%{\it massive} information transmission patterns that might correlate to world events.
%In this proposed work, we focus on four less-studied problems in information
%diffusion over social media. In contrast to popular memes, they constitute
%{\it rare}, {\it localized}, {\it absent}, and/or {\it slowly evolving} transmission paterns.
%(i) First, how does misinformation spread, i.e.,
%rumors, innuendo, and falsehoods? (ii) Second, how do mass movements get mobilized
%on Twitter, especially in a specific geographic area, (iii) Third, how do we detect
%absenteeism in social network activity and how can this information be employed?
%and (iv) Fourth, how do social
%networks enable the study of long-running, slowly evolving, societal phenomena (e.g., awareness
%of climate change)?
%Each of these issues requires the development of new mathematical models and algorithmic
%approaches which are explored here. Applications are drawn from the worlds of politics,
%civil unrest, and natural disasters.



\vfill

% GRANT INFORMATION

%This work was supported by the Intelligence Advanced Research Projects Activity (IARPA) via Department of Interior National Business Center (DoI/NBC) contract number D12PC000337. The US Government is authorized to reproduce and distribute reprints for Governmental purposes notwithstanding any copyright annotation thereon. The views and conclusions contained herein are those of the authors and should not be interpreted as necessarily representing the official policies or endorsements, either expressed or implied, of NSF, IARPA, DoI/NBC, or the US Government.

\pagebreak

% Dedication and Acknowledgments are both optional
%\chapter*{Dedication}

\chapter*{Acknowledgments}
Over the past four and half years I have received support and encouragement from a great many individuals.

My deepest debt is to my advisor, Dr. Naren Ramakrishnan. I have been amazingly fortunate to have an advisor who has given me the freedom to explore on my own, while at the same time pulling me back on track when I strayed too far afield. His far-sighted research attitude and the respect he has earned from his peers over the years have made the Discovery Analytics Center he leads a collaborative and productive place. His insight, wisdom and humor have made my graduate study a rich and rewarding journey that I will cherish forever.

I am especially grateful to Dr. Chang-Tien Lu, one of the best teachers that I have ever had. He introduced me to data analytics, opening up a whole new world of research and ideas to me. He has also provided unstinting support and encouragement throughout my search for an academic job.

I would like to thank Dr. Feng Chen, a mentor and a friend, for his support over the past several years as I moved from an idea to a completed study. I am deeply grateful to him for the countless discussions that helped me sort out the technical details of my research.

I appreciate the efforts of Dr. Yang Cao, for his encouragement and practical advice during the work reported in Chapter 5. He has also spared no effort in supporting my job hunt and his belief in me and what I can achieve has lifted my spirits when I became discouraged.

I would like to say thank you to Dr. Chris North for his support, feedback, and many valuable discussions that have helped me understand my research area better. It has always been a pleasure to get to know such a knowledgeable and amiable professor.

I would like to thank Dr. Huzefa Rangwala, who has always been generous with advice and ready to provide help. I hope one day I will become as good an advisor to others as Dr. Rangwala has been to me.

I am especially indebted to all the members of Discovery Analytics Center. Thank you for making our lab group such a warm and joyful family.

I would like to thank my collaborator Edward Dougherty, one of the best collaborators ever. His enthusiasm and efficiency has always inspired me.

Most importantly, I would like to thank my parents for their constant source of strength. I am so grateful to my husband for his care, sacrifice, and love. I hope he knows how much I appreciate the way he has walked alongside me through hardship, challenge and setbacks. His faith that I will succeed has kept me going. Thank you.




\tableofcontents
\pagebreak

\listoffigures
\pagebreak

\listoftables
\pagebreak

\pagenumbering{arabic}
\pagestyle{myheadings}

\subimport{ch1/}{ch1.tex}%introduction
\subimport{ch2/}{ch2.tex}%related work
\subimport{ch3/}{ch3.tex}%KDD GBM
\subimport{ch4/}{ch4.tex}%absenteeism
\subimport{ch5/}{ch5.tex}%rumor
\subimport{ch6/}{ch6.tex}%climate
\subimport{ch7/}{ch7.tex}%conclusion



% If you are using BibTeX, uncomment the following:
% \thebibliography
%
% Otherwise, uncomment the following:
% \chapter*{Bibliography}

\bibliographystyle{abbrv}
%Warning: don't put space after the comma in the following commands
\bibliography{bib/intro,bib/rumor,bib/ebola,bib/GBM,bib/wavelet,bib/pnas-sample}

% \appendix

% In LaTeX, each appendix is a "chapter"
% \chapter{Program Source}


\end{document}
