\section{Experimental Results}
\subsection{Fitting Results}

For each of the Twitter datasets, we were interested in quantifying the transitions of users through the different compartments of the SIS and SEIZ models.
Figures~\ref{fig:Boston_bombing} -~\ref{fig:Mexico_riot} display the results for the best fit of SIS and SEIZ models (Equations~\ref{eq:sis} and~\ref{eq:seiz}) to the eight Twitter stories. Also displayed for each figure are the relative error in 2-norm $$\frac{||I(t) - tweets(t)||_2}{||tweets(t)||_2}$$ and the mean error deviation $$\frac{\sum_{i=1}^n|I(t_i)-tweets(t_i)|}{n},$$  where $n$ is the number of data points.



\begin{figure}[h]
\centering
\subfigure[Boston]{
   \includegraphics[width=2in,height=1.5in] {pictures/Boston_SEIZ_total.png}
  \label{fig:Boston_time}
 }
  \subfigure[Pope]{
   \includegraphics[width=2in,height=1.5in] {pictures/Pope_SEIZ_total.png}
   \label{fig:Pope_time}
 }
  \subfigure[Amuay]{
   \includegraphics[width=2in,height=1.5in] {pictures/Amuay_SEIZ_total.png}
   \label{fig:Gas_time}
 }
\subfigure[Michelle]{
   \includegraphics[width=2in,height=1.5in] {pictures/Michelle_SEIZ_total.png}
  \label{fig:Michelle_time}
 }
  \subfigure[Obama]{
   \includegraphics[width=2in,height=1.5in] {pictures/Obama_SEIZ_total.png}
   \label{fig:Obama_time}
 }
  \subfigure[Doomsday]{
   \includegraphics[width=2in,height=1.5in] {pictures/Doom_SEIZ_total.png}
   \label{fig:Doomsday_time}
 }
   \subfigure[Castro]{
   \includegraphics[width=2in,height=1.5in] {pictures/Castro_SEIZ_total.png}
  \label{fig:Castro_time}
 }
 \subfigure[Riot]{
   \includegraphics[width=2in,height=1.5in] {pictures/Riot_SEIZ_total.png}
  \label{fig:Mexico_time}
 }
\vspace{-0.5em}
\caption{SEIZ compartment time-course results.
\label{fig:Time_course}
}
\end{figure}


The error metrics for these eight stories clearly indicate that the SEIZ model fits the Twitter data much more accurately than the SIS model. Furthermore, the low relative error of the SEIZ model fit suggests that this model accurately represents the Twitter data for each of the eight stories; see Table~\ref{table:error}. A common observation about all the eight stories is that the SEIZ model is far more accurate in modelling the initial spread of the news on Twitter as compared with the SIS model. This behaviour can be explained by the delay caused by individuals in the ``Exposed" category taking some time before posting a story themselves~\cite{powerofgoodidea:2006}.

Given that the SEIZ model is superior to the SIS model in this application, and that the SEIZ model demonstrates an accurate representation of information diffusion on Twitter, a natural question arises ``How can this model help us?" The answer is really simple. Since we have a mathematical model for the Twitter data, we can study solutions to some of the constraints as mentioned in the ``Practical Issues" section. A well fitted SEIZ model provides values for all contact rates and transition probabilities as defined by Equation~\ref{eq:seiz}. These parameters empower us to investigate the dynamics of news and rumor spread on Twitter in a fashion that is not possible without a mathematical model. Table~\ref{table:parameters} specifies the SEIZ model parameters that we can now examine to assess news and rumor propagation on Twitter.

\subsection{Boston Marathon Bombing Analysis}

To demonstrate a line of analysis that is now possible with the SEIZ mathematical model, we use quantities from the SEIZ model fit of the Boston Marathon bombing Twitter data (Table~\ref{table:parameters}). Results are summarized in Table~\ref{tab:seiz_boston_bombing}.

Here we discuss the dynamics of all 4 compartments, so we specially show all 4 compartments in the SEIZ time-course plot only for Boston Marathon bombing (Figure~\ref{fig:Boston_time}). These results suggest that the effective rate of susceptible individuals becoming skeptics is much greater than those that becoming infected. The decrease in $S(t)$ occurs directly with an increase in $Z(t)$, and $S(t)$ becomes stable at the same time that $Z(t)$. $I(t)$ does increase as $S(t)$ decreases, but its rate of change is much slower, and the majority of $I(t)$ increase occurs after $S(t)$ has stabilized to a minimal value, demonstrating that the continued change in the infected compartment has no further influence on the change in the susceptible compartment.

Table~\ref{tab:seiz_boston_bombing} also demonstrates that the skeptics compartment is more influential on transitioning susceptible users to the exposed class than does infected users. Figure~\ref{fig:Boston_time} shows this as the increase in $E(t)$ is strongly correlated with the increase in $Z(t)$. $E(t)$ also peaks as $Z(t)$ peaks, and $E(t)$ begins to decrease at a rate negative to that of the $I(t)$ increase. In fact, the increase in $I(t)$ directly coincides with a comparable decrease in $E(t)$. These data suggest that the increase in infected users is not due in large part by recruitment of susceptible users, but rather from the natural transition to the infected compartment by exposed individuals.

Putting this all together, we can deduce that virtually all individuals are initially in the susceptible compartment. Most susceptible users become skeptics from interaction with skeptics, and those susceptible users that do transition to the exposed class do so by their interaction with skeptics. The infected compartment increases predominately from the exposed class, and not from direct recruitment of susceptible individuals. Thus, these findings suggest that it was in-fact non-Twitter mediums that most greatly aided in the generation of Twitter propagation! Further, the $\frac{\epsilon}{\rho}$ ratio indicates that the exposed users became infected more so due to information incubation and self-adoption, and not so much from direct contact with infected users.

\begin{table}
\renewcommand{\arraystretch}{2.5}
\small
\centering
\caption{Ratios of SEIZ model for Boston dataset.}
\vspace{0.5em}
\begin{tabular}{|p{2.5cm} |  p{1.5cm}|} \hline
$\displaystyle \frac{bl}{\beta p}$ & 3.1E5\\ \hline
$\displaystyle \frac{b(1-l)}{\beta (1-p)}$ & 1.0E4 \\ \hline
$\displaystyle \frac{\epsilon}{\rho}$ & 7.8\\
\hline\end{tabular}
\label{tab:seiz_boston_bombing}
\end{table}

% I added this sentence to answer review 1.4
The remaining instances of SEIZ time-course plots are shown in Figure~\ref{fig:Time_course}, we can see how S, E, and I dynamic change over time. These analyses exemplify the types of analyses that can be used to study Twitter dynamics via the SEIZ population model.

\subsection{Rumor Detection}
We next examined if our implementation of the SEIZ model, applied to our Twitter examples, could be utilized to facilitate the discrimination of true news from rumors. We began by assembling an equation to relate the key parameters of the SEIZ model. In our first attempt at performing this, we restricted our attention to the exposed compartment; this class has direct or indirect interconnections between the other three compartments, and is a key path to the infected compartment. To exemplify this, consider the extreme case where susceptible individuals are attempted to be recruited by skeptics, and ultimately end up in the infected compartment (Figure~\ref{fig:seiz-framework}). This can only be accomplished by passing through the exposed compartment.


We quantify a ratio through E as the ratio of the sum of the effective transition rates \textit{entering} this compartment (from $S$) to the sum of the transition rates \textit{exiting} this compartment (to $I$). We define this ratio as $R_{SI}$, using the subscripts to denote the contributions from the susceptible and infected compartments in this quantity:

\begin{equation}
\label{eq:R_es}
R_{SI}=\frac{(1-p)\beta +(1-l)b}{\rho + \epsilon}
\end{equation}

$R_{SI}$ possesses all rate constants and probability values of the SEIZ model and relates them to the exposed compartment with a kind of flux ratio, viz.
the ratio of effects entering E to those leaving E. A $R_{SI}$ value greater than 1 implies that the influx into the exposed compartment is greater than the efflux. Similarly, a value less than 1 indicates that members are added to the exposed group more slowly than they are removed. We hypothesized that this measure could potentially aid in the distinction of rumor topics from news topics; all parameters of the SEIZ fit are utilized in this measure, and they are related via the $R_{SI}$ value to a key compartment of this model. If a distinction between rumors and true news stories is to be seen with the SEIZ model, we identify the $R_{SI}$ measure to be a probable candidate in aiding this process.

\begin{figure}[h]
\centering
  \includegraphics[width=4in]{pictures/reproductive-ratio.png}
   \caption{Ultimate $R_{SI}$ values for eight Twitter datasets.}
  \label{fig:static_ratio}
\end{figure}


\paragraph{Ultimate $R_{SI}$ value}
We then computed $R_{SI}$ using the specific parameter values attained from our model fits of the eight cases (Figure~\ref{fig:static_ratio}).
Here we can see that the true news about the Boston Marathon bombing, Pope resignation, and Amuay refinery explosion do in fact have much higher $R_{SI}$ values than the rumor topics: Doomsday, Fidel Castro death, Mexico City riots, and Barrack Obama injury which each have much lower $R_{SI}$ values. However, the Michelle presence at the Oscars, which is classified as true news, has a very low $R_{SI}$ value. This particular case is interesting since Michelle did not really show up to the 2013 Oscar Awards Ceremony. She simply participated remotely via video telecast. It is thus arguable that this topic could have been discussed in the media in terms similar to rumors.

\paragraph{Dynamic $R_{SI}$ value}

\begin{figure}[h]
\centering
  \includegraphics[width=4.5in]{pictures/Castro-Boston-realtime-RSI.png}
   \caption{Dynamic $R_{SI}$ values for Castro rumor and Boston bombing real news.}
  \label{fig:dynamic_castro_ratio}
\end{figure}

For one story, if we collect related Tweets at the very beginning stage, then the $R_{SI}$ value would be a time series. Figure~\ref{fig:dynamic_castro_ratio} shows the dynamic $R_{SI}$ values for Castro rumor and Boston bombing real news. We can see for the Castro rumor, at the beginning stage, the $R_{SI}$ value was pretty high, which indicate considerable people believe this story. With time pass by, the $R_{SI}$ values decreased sharply. However, for the Boston Marathon bombing real news, the $R_{SI}$ value was very low at beginning, but increased rapidly within several hours and reached around 28.31 very soon. The $R_{SI}$ time series indicate people's confidence with those stories is dynamically changing.


These findings suggest, for these specific topics, that the parameters in the SEIZ model can potentially aid in the challenge of distinguishing rumor versus true news. We are not claiming that the $R_{SI}$ value is the unique measure to accomplish this, nor are we claiming that the SEIZ model itself is the sole tool to do this. As is suggested by our findings, we postulate that a fit of a compartmental model, in the spirit of the SEIZ model, to Twitter data provides valuable propagation information that can be coupled with other data analysis strategies  (e.g., content modeling)
to augment the accuracy and reliability of true news story and rumor topic discrimination.

