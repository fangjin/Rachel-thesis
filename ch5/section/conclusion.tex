\section{Conclusion}
In this chapter, we have demonstrated how true news and rumor stories being propagated over Twitter can be modelled by epidemiologically-based population models. We have shown that the SEIZ model, in particular, is accurate in capturing the information spread of a variety of news and rumor topics, thereby generating a wealth of valuable parameters to facilitate the analysis of these events. We then demonstrated how these parameters can also be incorporated into a strategy for supporting the identification of Twitter topics as rumor or news. As of now, we are modeling propagation over static data. In future, we plan to adapt this model for capturing news and rumors in real-time.

%In this chapter, we have quantified information propagation, including both news or rumors,
%on Twitter using epidemic models.
%We have demonstrated that the SEIZ model is accurate in capturing
%information diffusion, especially in the initial portion of the curve.
%Using the SEIZ model, we have developed a method to detect rumors.
%In the future, we plan to incorporate more detailed follower information in
%our models, to determine exactly how many twitter users receive
%tweets from a sender.
%Such data can be used to determine SEIZ transition rates more accurately.
%Another interesting approach that we plan to explore in the  future
%is to use sentiment analysis on tweets to analyze whether an individual is
%a skeptic or infected or just exposed. Finally, we would like to develop
%an automated method to separate rumors from truth using a multiplicity of features.


