
%\chapter{Epidemiological Modeling of News and Rumor on Twitter}
\chapter{Epidemiological Modeling Real Movements from Rumors}
\label{ch:rumor}
%Distinguish Real Protests from Rumors

% if you want to define paper specific  macros
% then put everthing between begingroup and endgroup
\begingroup
\newcommand{\score}{S}
\newcommand{\myalgo}{CoolAlgo}

%
%\begin{abstract}
%Characterizing information diffusion on social platforms like Twitter
%enables us to understand the properties of underlying media
%and model communication patterns. As Twitter gains
%in popularity, it has also become a venue to broadcast rumors and
%misinformation. We use epidemiological models to characterize
%information cascades in twitter resulting from both news and rumors.
%Specifically, we use the SEIZ enhanced epidemic
%model that explicitly recognizes skeptics to
%characterize eight events across the world and spanning
%a range of event types.
%We demonstrate that our approach is accurate at capturing diffusion in these
%events. Our approach can be
%fruitfully combined with other strategies that use content modeling
%and graph theoretic features to detect (and possibly disrupt) rumors.
%\end{abstract}

%As social media (e.g., Twitter) continues to increase in popularity, they are becoming mainstream venues to stage mass movements, such as protests and uprisings.
\section{Contributions}
As social media such as Twitter continue to increase in popularity, they are effectively becoming social sensors that can be utilized for real-world mass movement event detection. Modeling and studying the related adoption patterns provide new insights for investigating both the social and physical aspects of these events and their precursors. This dissertation has presented several approaches and strategies aimed at detecting and predicting mass movements and further inferring their causality, given information composed of a volatile mixture of real news and rumors. Techniques have been presented that are designed to capture information propagation across multiple spaces, as well as a new graph wavelet approach that broadens predictive capabilities to capture group anomalies within networks. A number of different types of mass movements have been investigated and diverse aspects of modeling addressed.

Using social media as indicators for real-word event detection can indeed be a helpful tool, but several limitations apply, perhaps most notably when applied to a specific event type such as the mass movements studied here. First, modeling protest-related topic propagation across networks is never trivial. One challenge is that social protest propagation through online media can spread over large areas far more quickly than traditional methods since users are geographically distributed, while other challenges include the way mass protest information can be spread by multiple social media types and paths, such as word of mouth, and radio and TV news broadcasts. Second, even detecting group anomalies on social media is challenging. For example, Twitter's user network embodies many subgraphs based on social ties with dynamic graph structures that are constantly changing since users are actively contributing content. The other challenge includes real world events that are not only correlated with burst signals, but can also exhibit unusually low levels of activity in social networks. Despite these restrictions, graph wavelets have in fact proven to offer a powerful capacity for capturing graph anomalies (in terms of both burst behavior and absenteeism behavior), even on dynamic changing networks.


A fundamental objective of this research has been to model mass movement adoption behavior, and in doing so, several significant additional advantages have been gained. One contribution is the ability to model information propagation across multiple networks/spaces, and capture the propagation speed and possible propagation paths, as demonstrated in Chapter~\ref{ch:GBM}. Another benefit gained through enhancing the mass movement detection capability is the opportunity this provided to introduce a new group anomaly detection approach, as described in Chapter~\ref{ch:absenteeism}. The use of graph wavelets made it possible to develop a more appropriate definition of group anomaly that covers both bursts and absenteeism utilizing different scales, thereby increasing the probability of capturing protest behaviors. Another benefit is the ability to quantify compartment transition dynamics using the epidemic model SEIZ, thus facilitating the development of screening criteria that can distinguish real movements from rumors on Twitter, as demonstrated in Chapter~\ref{ch:rumor}.

Understanding information propagation over dynamic social networks is becoming a highly popular way to address real-world problems in social network analysis. The research reported in this dissertation analyzed several fundamental questions that underlie a broad range of propagation-like processes, focusing particularly on mass movement adoption and rumor transmission. These methodologies can easily be extended to applications such as infectious diseases, public health and marketing, among others.

\section{Future Directions}
Given the wide scope of this relatively new field, there are many opportunities for conducting potential fruitful research. For example, one very attractive area would be to continue to focus on social network analysis, specifically research into information propagation over dynamic rapidly changing social networks. Here, future research directions fall into two general categories, with one being to deepen the existing theories and algorithms by exploring them in more detail while the other is to broaden the current research by adopting a wider perspective. Examples of these and other research possibilities are presented below.

\paragraph{Extend GBM model}
What happens to the geometric Brownian motion model if the underlying mention network changes over time? How can this this model be adopted or modified to apply to multiple networks? In addition to these theoretical questions, a number of practical applications are worth further investigation, for example would it be possible to introduce the GBM model into the infectious disease domain to model the spread of the zika virus? Assuming that bispace is composed of both a connection network and physical space, can we train the GBM model to estimate individuals' infection probability based on their environment?


\paragraph{Further studies involving graph wavelets}
It should be possible to extend the graph wavelet applications presented in Chapter 4 into other areas based on the two of the major distinguishing properties of graph wavelets. The ability to detect graph anomalies could be utilized to detect the wealth gap between rich and poor in a given region, identify brain neural network anomalies, or detect traffic congestion through road network analysis. Similarly, the ability to identify the central point of a subgraph could be employed to rank key players within networks, detect those spreading rumors in some cascade, or find the sources of infection for certain diseases.

\paragraph{Broaden rumor detection scenarios}
Instead of determining whether a particular story is true or false, it may instead be more practical to predict how likely people are to believe it. Newspapers in particular would find it very useful, especially when it comes to a breaking news story that has not yet been officially confirmed. Before reporting to the public, they need to know how believable the story is. Also this offers a valuable way to decide whether vendors are cheating during online shopping.


\paragraph{Deepen our understanding of personalized information propagation}
It would be useful to know how users' behavior either delays or boosts information propagation, especially when accompanied by strong sentiments. This may help formulate better advertisement strategies if a better way to manipulate the information flow can be identified. Further, it would be interesting to explore opportunities to extract personalized information spread patterns. What kind of news arouse the interest of individual consumers? What kind of role is he or she likely to play? and What kind of push strategy might stimulate their activity? This type of study would be invaluable for precision marketing or personalized recommendations, providing refined content filtering.


\paragraph{Build an intelligent disaster detection system}
Another interesting option would be to build an intelligent system that is efficient at event detection, especially for disasters, protests, and other extraordinary events. Such a system would not only be capable of immediate reporting, but also able to track events and perform causality analyses and even provide future predictions. This type of system would require several critical building blocks: natural disaster detection could be based on the use of graph wavelets for detecting group anomalies, rumors or news detection could employ an epidemic SEIZ model, story causality analysis and event coding. Consider the Flint water crisis for example. First, it would need to identify the water crisis events from a social media analysis and confirm it to be a true story, after which it would trace all the relevant historical news to identify the causality, leading finally to the generation of a complete report using event coding.


\paragraph{Combine social network analysis with physical data.}
The advent of social media provides unprecedented opportunities to access vast quantities of information that could potentially benefit our research. This opens up new possibilities for reinvigorating traditional research in many areas, hopefully gaining new insights and leading to extraordinary discoveries. Take vaccines and the study of their potentially adverse effects, for example. Traditionally, vaccine research depends heavily on the raw data collected by the CDC, hospitals, patient reports, and vaccine adverse event reports. However, this data frequently suffers from problems such as lengthy time delays and incomplete information. Worse, most of the data is considered in isolation. If we can find a way to combine conventional statistical data with social media data such as tweets and Facebook posts, it may be possible to extract more information and create a complete picture showing what kind of people likely to be more vulnerable to a specific vaccine. In this way, it may be possible to better predict adverse events or even contribute to the design of new vaccine approaches that minimize or eliminate serious vaccine-related reactions. Given current advances in data mining, this is a revolutionary time for research in real-world applications using social network analysis.


% comments from prilim
%\begin{itemize}
%\item North: lessons learned are?
%\item North: what does this say about the real phenomenon?
%\item North: help us learn about the semantics of the phenomenon
%\item North: why not detect rumors in other ways? Why not look at the source?
%\item CT: What's the relationship between rumors and absenteeism.. Spatial granularity can be common and can be harnessed. relationship between community and group anomaly,
%\item CT: climate keywords. Static keywords or dynamic keywords?
%\item Yang: why did u choose these particular techniques? no consistent theme across
%\item North: replace "4 problems in.." with a better title
%\item Feng: a treemap to connect the 4 topics
%\end{itemize}

%\section{Related work}
Related work falls in three categories.
\paragraph{Information Diffusion}
Significant work has gone into research on information diffusion on
social media,
%Information diffusion in social networks is a well-researched topic, by
%both social scientists and data miners. One of the earlier ideas in this
%space was proposed by Everett Rogers in his theory ``Diffusion of
%Innovations" ~\cite{Rogers-diffusion1962} where he explained how, why, and
%at what rate do ideas spread through cultures. In recent years, online social networks have provided social scientists with
%an unprecedented amount of information, leading to a wide range of algorithmic
%approaches. There are some chapters focusing on the information diffusion across topics and time on Twitter, such as
e.g., see~\cite{Cha10measuringuser,Kwak:2010:TSN:1772690.1772751,Romero:2011:DMI:1963405.1963503,Yang_predictingthe}.
Recently, Matsubara etc.~\cite{matsubara2012rise}
conducted research on the rise and fall patterns of information diffusion, and managed to
capture the power-law fall pattern and periodicities inherent in such data.
Gomez-Rodriguez et al.~\cite{GomezRodriguez:2010:IND:1835804.1835933}
built a cascade transmission model to track cascading process taking place over a network; they traced overall blogs and news for a one-year period and found that the top 1000 media sites and blogs tend to have a
core-periphery structure.
%Kimura et al.\cite{Kimura:2009:EEI:1661445.1661772} discuss
%an approach to reduce the computational complexity of identifying most influential nodes in a large-scale network. They used two stochastic diffusion models and proposed a greedy algorithm on the basis of bond percolation and graph theory to reduce computational complexity. By modeling an email network as a graph, Wu et al \cite{WuFang2004b} found that the probability of a node forwarding a meme to a neighboring node decays as the graph distance between the node and the source increases. Gruhl et al. \cite{Gruhl04informationdiffusion} proposed a transmission graph to study information diffusion in the blogosphere, where they computed the probability that the topic will traverse the edge, following this they recomputed the posteriors and cycled through the process until convergence.
%%comprises a set of authors connected in a directed graph. Using the current version of the transmission graph, for each topic and each pair of authors they computed the probability that the topic will traverse the edge.
%
\paragraph{Epidemiological models}
Mathematical modeling of disease spread not only provides vital information about the propagation of the disease in a human network, but also offers insight into the strategies that can be used to control them.
%Kermack and McKendrick~\cite{Kermack1927Royal} study the effect of various factors that govern the spread of contagious epidemics in a population and are considered as one of the pioneers in this area. Later, Daley et al. \cite{Daley-nature-1964} suggested an analogy between the spread of disease and the dissemination of information and used epidemiological models to describe the growth and decay of rumor spreading processes. They divided the population into three mutually exclusive classes based on their dissemination of a rumor: have not heard it,
%are actively spreading it, and have stopped spreading the rumor.
%%This %division of the human population allowed them to apply the
%Kermack-McKendrick epidemic model to the propagation of a rumor.
The classification of the human population into different groups forms
the basic premise of using epidemiological models for modeling
information diffusion.
The two widely used such models are SIR (Susceptible, Infected, Recovered) and SIS (Susceptible, Infected, Susceptible) models. Newman et al.~\cite{PhysRevE.66.016128} showed that a large class of standard
epidemiological models, viz. the SIR models, can be solved exactly on a wide variety of networks, and confirmed the correctness of solutions with numerical simulations of SIR epidemics on networks. Kimura et al.~\cite{Kimura:2009:EEI:1661445.1661772} proposed the application
of the SIS model to study information diffusion where the nodes can be activated multiple times.
%In our chapter, we also applied SIS model for news and rumor spreading over Twitter.
%They created a layered graph from the social network where layers gets added on top of each other and then apply bond percolation with a pruning strategy with an intent to lower the computational complexity of the SIS model.
% \cite{Alison2010-Society} use a modified SIS model to prove emotions spreads very much like an infectious disease.
%tried to evaluate the spread of long-term emotional states across a social network using a modified SIS model which included the possibility of spontaneous infection. They were able to provide evidence that emotions spreads very much like an infectious disease.
Zhao et al.~\cite{RePEc:eee:phsmap:v:392:y:2013:i:4:p:987-994} proposed an SIHR (Spreaders, Ignorants, Hibernators, Removed) rumor spreading model,
with forgetting and remembering mechanisms to simulate rumor spreading in inhomogeneous networks.
Xiong et al.~\cite{Xiong20122103} proposed a diffusion model with four different states: susceptible, contacted, infected, and refractory (SCIR) to identify the threshold value of the spreading rate approaches almost zero.
%They study information diffusion on Twitter using the retweeting mechanism.
%There were able to identify that the threshold value of the spreading rate approaches almost zero and that the degree-based density of infected agents increases with the degree monotonously.
Bettencourt et al.~\cite{powerofgoodidea:2006} proposed the SEIZ (susceptible, exposed, infected, skeptic) model to capture the adoption of Feynman diagrams by using the publication counts after World War II. They extract the general features for idea spreading and estimate the idea adoption process. Their result showed that
the SEIZ model can fit the long term idea adoption process with reasonable error, but does not demonstrate whether this model can be applied on large scale datasets, or whether can be applied on Twitter, where the story unfolds in real-time.

\paragraph{Rumor modeling}
As far as we know, Daley~\cite{Daley-nature-1964} first proposed the similarity between epidemics and rumors using mathematical analysis. Some researchers have studied rumor propagation modeling in different network topologies~\cite{nekovee2007theory,zanette2002dynamics}; however,
they do not provide any discussion of propagation differences between news and rumors.
Shah et al.~\cite{shah2011rumors} detect rumor sources in network using maximum likelihood modeling.
In~\cite{budak2011limiting}, Budak et al. prove that minimizing the spread of the misinformation (i.e., rumors)
in social networks is an NP-hard problem and also provide
a greedy approximate solution. Castillo et al.~\cite{castillo2011information} delve into twitter content modeling, such as sentiment analysis and
hashtags to identify rumors, while Qazvinian et al.~\cite{qazvinian2011rumor} try to address this issue using broader linguistic methods,
to learn possible features of rumor and determine whether a twitter
user believes a rumor or not. More related work appears
in~\cite{Isham-physica-2009,PhysRevE.81.056102}. Our goal is to develop an understanding of
these processes using diffusion models.
%
%Trpevski et al. \cite{PhysRevE.81.056102} used the SIS model to study how two different rumors propagate in a network, Zhou et al.~\cite{Zhou2007458} studied rumor propagation in complex networks using the SIR model, while Zhao et al \cite{RePEc:eee:phsmap:v:392:y:2013:i:4:p:987-994} proposed a modified SIR model where they assumed that ignorants will inevitably change their status to either spreaders or stiflers after getting contacted by the spreaders, and Ishama et al. \cite{Isham-physica-2009} try to find the final size of a rumor on a homogeneous network using the stochastic SIR epidemic model, where they applied embedded Markov chain techniques to derive a set of equations that can be solved numerically to identify the spread of a rumor.
%These two rumors are initiated with different probabilities of acceptance. They found that one of
%the rumors is typically more dominant in the network compared to other.
%However, all their research works are based on simulation modeling whereas our work here are trying to model actual news and rumor propagation on Twitter.
%
% Compared with our approach, we are trying to identify rumors from diffusion properties, via an extended epidemic model SEIZ \cite{powerofgoodidea:2006}, from which we can get a ratio of the rate at which people get exposed (and not believe) to the rate at which people move from exposed to belief. We think the different distributions of the max ratio might shed some light on news and rumor detection. while they didn't address the dynamic rumor propagation models in twitter.
%\narenc{is it skeptics or is it skeptic? Pick one and stick with it.}

%\narenc{The related work in general is very poorly written. It reads like this:
%
%A et al.. did this.
%B et al. did this.
%C et al. did this.
%
%There is no structure, no organization. You need a good story. The sentences
%have to be more tightly linked. Look at some of my chapters to get
%an idea. You don't need 1 sentence for each chapter. You can group chapters
%further and cite them together. For instance: Some chapters [CITE1, CITE2]
%take the approach of WHATEVER. The whole related work must be in 1 page
%but without losing any citations. }

%\narenc{Looks like this is not just
%tweet news dataset. It contains both news and rumors. And it is not
%a dataset. It is many datasets. You can just call it Tweet datasets studied
%in this chapter. Replace 'Real' by 'Type' so that the types are
%News and Rumor, not True and Rumor. Remove Lang, Collection columns.}


\section{Datasets}

\begin{table*}[!htp]
\tiny
\caption{Twitter datasets studied in this chapter.} % title of Table
\vspace{0.5em}
\centering % used for centering table
\begin{tabular}{| p{0.3cm} | p{1.2cm} | p{1.4cm} | p{1.6cm} | p{0.8cm} | p{1.1cm} | p{0.8cm} | p{1.4cm} | p{4.2cm} | }
\hline %inserts double horizontal lines
\textbf{No.}& \textbf{Dataset} & \textbf{Date}  & \textbf{Area} & \textbf{Type} & \textbf{Country} & \textbf{\#Tweets} & \textbf{Response ratio} & \textbf{Keywords \& Hashtag}  \\ [1ex] % inserts table
%heading
\hline % inserts single horizontal line
1& Boston & 04-15-2013&  terrorism& news & USA &  501259 &68.3\% & Marathon, (\#)bostonmarathon \\[1ex]
\hline
2&  Pope & 02-11-2013 &religion&news & Vatican & 31365 &56.75\% &Pope, (\#)Benedict \\[1ex]
\hline
3& Amuay &08-25-2012&  accident &news & Venezuela&49015 &62.89\% &Amuay, refinery, explosion \\[1ex]
\hline
4& Michelle &02-24-2013&  entertainment & news & USA& 3762& 54.45\% &Michelle Obama, Oscars   \\[1ex]
\hline
5& Obama & 04-23-2013&  politics& rumor & USA & 791&46.14\% &White House, explosions    \\[1ex]
\hline
6& Doomsday & 12-21-2012&  mythology& rumor & Global &11833 &52.19\%& Doomsday, Mayan, doom  \\[1ex]
\hline
7& Castro & 10-16-2012&  politics & rumor & Cuba & 3862& 54.45\% &Fidel Castro, Dr. Marquina \\ [1ex]
\hline
8& Riot & 09-05-2012& crime & rumor & Mexico &4631& 47.17\% & Antorcha Campesina, Nezahualcoyotl\\ [1ex]
\hline
\end{tabular}
\label{table:story-intro} % is used to refer this table in the text
\end{table*}


We focus on twitter datasets that have reliable coverage of the
events being studied; the volume of tweets ranges from as low as
791 to nearly three orders of magnitude greater.
As described in Table~\ref{table:story-intro},
the news and rumors studied were drawn from a variety of
regions and across a diversity of topics.
Data collection was aimed at gathering tweets highly related to the events
under study. We employed customized sets of keywords and hashtags pertaining
to each incident.
%%%%%%%%%%%%%%%%% I removed this sentence, since I did not apply geolocation algorithms.  %%%%%%%%%%%%%%%%%%
%In addition, since our stories spanned many geographic regions, we employed geolocation algorithms that inspect either (latitude, longitude) coordinates or other user profile information.
Finally, date range restrictions were used to define relevant tweets for
each event. It is also pertinent to note that the tweets analyzed spanned
a variety of languages: English, Spanish, Italian, and Portuguese.

%This section describes the twitter data sets which were used for our analysis. Since there are millions of news related tweets published every day on Twitter, we selected only the ones which met the following criteria: (1) the volume of tweets related to the event should be more than 2000. (2) Reliable news sources have mentioned the news/rumor in their tweets. In this chapter, we used eight news events for our analysis which are described in table , Out of the eight news events five were true news and the rest three were rumors. The nature of the news events were determined by confirming with the appropriate reliable sources. In order to avoid any biases, the news events were chosen from wide range of topics and vary widely in their geographical scope.

\subsection{News topics}

\noindent
\textbf{\emph{Boston Marathon Bombings.}}
Two pressure cooker bombs exploded near the finish line of 2013 Boston Marathon on April 15, 14:49:12 local time,
killing three people and injuring more than 264 others.
The FBI released photographs and surveillance videos on online social networks which spread like wildfire and provided crucial leads for identifying the suspects\footnote{http://www.cnn.com/2013/04/15/us/boston-marathon-explosions}.

\noindent
\textbf{\emph{Pope Resignation.}}
Pope Benedict XVI announced his resignation on the morning of February 11, 2013. In nearly 6 centuries,
this was the first time a pope has stepped down from his office. This news
received reactions from all across the world\footnote{http://www.cnn.com/2013/02/11/world/europe/pope-resignation-q-and-a}.

\noindent
\textbf{\emph{Amuay Refinery Explosion.}}
Propane and butane gas leakage caused an explosion at the Amuay refinery in Venezuela on August 25, 2012 1:11 am local time. The blast killed 48 people, injured 151 others and damaged
1600 homes\footnote{http://www.cnn.com/2012/08/25/world/americas/venezuela-refinery-blast}.

\noindent
\textbf{\emph{Michelle Obama at the 2013 Oscars.}}
In the 2013 Oscar awards ceremony, a big surprise was
the appearance of US first lady Michelle Obama for presenting the `Best Picture' award\footnote{http://www.mediaite.com/tv/michelle-obama-makes-cameo-at-the-oscars-announces-be
st-picture-winner/}. %\narenc{In the table, her name is written as Michele (STILL - NEEDS TWO Ls).}


\begin{figure}[t]
\centering
\subfigure[Amuay explosion]{
\includegraphics[width=3in,height=1.8in] {pictures/volume1.png}
\label{fig:Volume1}
}
\subfigure[Castro rumor]{
\includegraphics[width=3in,height=1.8in] {pictures/volume2.png}
\label{fig:Volume2}
}
\vspace{-1em}
\caption{Tweet volume.}
\label{fig:Volume}
\end{figure}

\subsection{Rumors}

\noindent
\textbf{\emph{Obama injured.}}
A fake associated press (AP) tweet originated on April 23, 2013 that President
Obama was hurt in White House explosions which caused a brief period of
instability in financial markets. The information was false
and it was determined that the Twitter account was hacked.

\noindent
\textbf{\emph{Doomsday.}}
December 21, 2012 was rumored to be the Doomsday as it marked the end date of a 5126 year long cycle in the Mesoamerican long count calendar. This rumor
spread like wildfire and social networks were flooded with panic and anxiety posts. Considering that we are still alive, Doomsday turned out to
be nothing more than a rumor on a massive scale\footnote{http://en.wikipedia.org/wiki/Doomsday}.

\noindent
\textbf{\emph{Fidel Castro's death.}}
On October 16, 2012 a Naples doctor claimed that former Cuban leader, Fidel Castro suffered a cerebral hemorrhage and is near a
neurovegetative state. However, on October 21, 2012, these rumors were denied
by Elias Jauva, former Venezuelan vice president, who released pictures of him
meeting Castro a few days back\footnote{http://www.inquisitr.com/371007/fidel-castro-allegedly-appears-in-public-after-stroke-rumors/}.

\noindent
\textbf{\emph{Riots and shooting in Mexico.}}
A very interesting example that highlights the perils of rumor spreading on social networks pertains to the false reports of violence and impending attack
in Nezahualcoyotl, %\narenc{Is it the correct spelling? Please checkup carefully. correct}
Mexico. (False) rumors spreading on Twitter and Facebook
about shootouts caused (real) panic and chaos in Mexico City
on September 5, 2012. Interestingly, authorities themselves
turned to Twitter to deny
these rumors\footnote{http://www.foxnews.com/world/2012/09/08/tweets-false-shootouts-cause-panic-in-mexico-city/}.



\begin{figure}[t]
\centering
\subfigure[Amuay explosion]{
\includegraphics[width=3in,height=1.8in] {pictures/rumor1-scatter.png}
\label{fig:Follow1}
}
\subfigure[Castro rumor]{
\includegraphics[width=3in,height=1.8in] {pictures/rumor2-scatter.png}
\label{fig:Follow2}
}
\vspace{-1em}
\caption{Followers/followees distributions. Followers: people who follow the person; Followees: people who are followed by the person.}
\label{fig:Followers}
\end{figure}
%\footnote{http://confusedofcalcutta.com/2009/01/11/of-followers-and-followees-and-friends/}

%\subsection{Twitter news collection }
%During data collection from Twitter, our main concern was to obtain pure tweets which are highly related to the news events. We used a customized set of keyword lists and hashtags for filtering the tweets. For each story, we created topic-exclusive keyword list and filtered the relevant tweets by matching the keywords. More story-specific tweets were identified by using a list of $\#$hashtags.
%
%As our stories span across geographic locations and languages, we used several different measures to collect the data. For example, four out of the eight stories are from South America, and hence during data collection for these stories we restricted ourselves to the tweets originating from South America and written either in Spanish or Portuguese. Table 1 list out all the restrictions that were used in collecting the tweets. Further, for each of these topics, we tried to collect the tweets from the beginning of the topic birth to its death. The only exception is the Doomsday rumor for which there is not fixed beginning date. Date range for data collection, along with the keywords, hashtags and tweet counts are mentioned in Table~\ref{table:story-intro} for each of the news story.

\begin{figure*}[t]
\centering
\subfigure[Amuay Cascade]{
   \includegraphics[width=3in,height=2.5in] {pictures/news.png}
   \label{fig:gas-cascade}
 }
  \subfigure[Castro Cascade]{
   \includegraphics[width=3in,height=2.5in] {pictures/rumor.png}
   \label{fig:Castro-cascade}
 }

\caption{Retweet cascade for the Amuay Explosion news and Castro rumor. Each node is a user id, and each edge connects the retweet user to the original user.}
\label{fig:full-cascade}
\end{figure*}


\subsection{Preliminary Analysis}
We compare the basic properties of news and rumor propagation,
by characterizing
tweet volume over time, follower/followee distributions, the `response ratio'
of a story, and the retweet cascades. In order to maintain brevity, we show
results from only two stories in this section: one from our news
collection (the Amuay explosion) and one from our rumor collection (Fidel
Castro's purported death).

\textbf{Tweet Volume.} For both examples, we plot the tweet volume over time from the beginning of the story. Figure~\ref{fig:Volume1} shows the activity for the
2012 Amuay refinery explosion example. An activity burst was formed immediately after the news was made public. The number of tweets dropped progressively as the days went by. This activity trend displays attributes similar to breaking news propagation as described by Mendoza
et al.~\cite{Mendoza:2010}. In contrast, Figure~\ref{fig:Volume2} depicts
the volume of tweets about a rumor regarding the health of the
former Cuban leader Fidel Castro. Here we see
occasional spikes of
tweet volume; note the increase in tweet volume
around October 21st, when the rumors were officially denied.

\textbf{Followers and Followees Distributions.} Figure~\ref{fig:Follow1} is a log-log
scatter plot of the followers/followees distribution about
the Amuay explosion news, and Figure~\ref{fig:Follow2} is the corresponding
plot about Fidel Castro's death rumor. There is no significant
qualitative or quantitative difference in this case; in particular both
plots show that
the number of followees is less than the number of followers.
%Table~\ref{table:follower} gives the details: %in Amuay dataset, 33936 users
%%register more followers than followees (69\%),
%%14936 users register more followees than followers, and 143 users register
%%the same number of followers and followees.
%Amuay news and the Fidel Castro rumor exhibits comparable percentages for follower and followee.

%\begin{table}[ht]
%\centering
%\caption{Follower (ER) and followee (EE) distribution for two datasets.}
%\label{table:follower}
%\begin{tabular}{|c|c|c|l|} \hline
%&ER>EE&ER=EE&ER<EE\\ \hline
%Amuay news & 33936 (69\%)& 143 (1\%)&14936 (30\%)\\
%Castro rumor &2549 (66\%)& 13 (0.4\%)&1073 (33\%)\\ \hline
%\end{tabular}
%\end{table}

\textbf{Response Ratio.} A tweet can either be a post made by the
user's initiative,
or a responsive post  to some other user's post (e.g., retweets and replies).
As Starbird et al.~\cite{Starbird:2012} discuss, retweets reveal how
information propagates through a social network: the `deeper' a retweet, the
more relevant the tweet is for the community. Based on this idea,
we define the response ratio of a story as the fraction of responsive tweets
to the total number of tweets in the story.
Table~\ref{table:story-intro} lists the response ratio for all the 8 stories. As we can see, response ratios for news are higher than that for the rumors.
%\narenc{You keep using different names: response ratio, responsiveness ratio,
%response rate, responsiveness rate. Pick one and stick with it.
%Lets use response ratio because I don't think you are computing any `rate'.}
%\narenc{You have now removed that column from the table. Put it back.}

\textbf{Retweet Cascades.} A retweet cascade reflects how the social media network propagates information. Figure~\ref{fig:full-cascade} depicts the evolution of the retweet graphs for the Amuay news and Castro rumor dataset. For Amuay news, we plot four graphs with intervals of 6 hours,
% to show the dynamic process of the information flow on the Twitter.
depicting that a burst has been formed during 6am-12am, only 5 hours after the accident.
 %In the following hours, ten of thousands of users got this news from many sources.
 Fig.~\ref{fig:Castro-cascade} shows the retweet graphs of the rumor for several days. We can see
% that there are several central information sources to expand the rumor on Oct 16th, 2012.
 even after one day, there is no burst of tweets related to this rumor.
% After the clarification of the rumor, a little more users are involved in the re-tweet network.
 Compared with the network between the news and rumors, we find several features about the rumor.
% 1) In the initial  stage of the rumor, several central information sources posts the rumors.
1) The network for the news instance
is more complex and users can obtain news from many sources, while users obtain
the rumor information only from limited information centers. 2) There is an immediate burst after a news is made public while there is no obvious burst for the rumors.

%\narenc{What do people learn from this paragraph and the picture? Can you plot
%it for the Castro rumor? Does it look different?}

%\textbf{Can we learn more?} Although these properties provide us
%with some discriminating information, they still do not help
%develop a model-based approach for how news propagates differently
%from rumors, which is our focus next.

%with good amount of information we still can't quantify how news propagates over time. Also it would be interesting to know how information cascades were generated in a step by step manner over space. Hence, in this chapter, we propose a diffusion method and an event-driven model to find answers to the above mentioned problems. We represent Twitter network as a graph with vectors and edges and apply two epidemic models to simulate how vectors and edges grow with time. More specifically, we propose a 1-step graph transition method to simulate the graph growing process. By monitoring the step by step propagation process, we quantify transfer deviation using our event-driven model.

\section{Our Approach}
As stated earlier, we used compartmental population models to quantify the propagation of news and rumors on Twitter, focusing primarily on the SIS and SEIZ models.

\subsection{SIS}
As described earlier, this model divides the population into two compartments, or classes: susceptible and infected.
Note that in this model, infected individuals return to the susceptible class on recovery because the disease confers no immunity against reinfection.


In order to adapt this model for Twitter, we have given new meaning to these terms. An individual is identified as infected (I) if he posts a tweet about the topic of interest, and susceptible (S) if he has not. A consequence of this interpretation is that an individual posting a tweet is retained to the infected compartment indefinitely; hence, he can not propagate back to the susceptible class as is possible in an epidemiological application. At any given time period $t$, we use $N(t)$ to denote the total population size, $S(t)$ the susceptible population size, and $I(t)$ the infected population size, such that $N(t) = I(t) + S(t)$. As shown in Figure~\ref{fig:sis-framework}, the SIS spreading rule can be summarized as follows:

\begin{figure}[ht]
\centering
%\epsfig{file=word.pdf, width=3in}
\includegraphics[width=2.5in]{pictures/SIS.png} %?????????????
\vspace{-1em}
\caption{SIS model framework}
\label{fig:sis-framework}
\end{figure}


\begin{figure}[ht]
\centering
%\epsfig{file=word.pdf, width=3in}
\includegraphics[width=2.5in]{pictures/SEIZ.png} %?????????????
\vspace{-1em}
\caption{SEIZ model framework}
\label{fig:seiz-framework}
\end{figure}


\begin{itemize}
\item An individual that tweets about a topic is regarded as infected.
\item A susceptible person has not tweeted about the topic.
\item A susceptible person coming into contact with an infected individual (via a tweet) becomes infected himself, thus immediately posting a tweet.
\item Susceptible individuals remain so until coming into contact with an infected person.
\end{itemize}

\begin{figure}[t]
\centering
  \includegraphics[width=3in]{pictures/implementation_flow.png}
  \vspace{-1em}
  \caption{Numerical implementation work-flow.
 % For both the SIS and SEIZ models, a nonlinear least squares fit of each ODE system to Twitter data identifies an optimal parameter set by minimizing the difference between the Infected compartment and the Twitter data.
  }
  \label{fig:implementation_flow}
\end{figure}

\noindent
The SIS model is mathematically represented by the following system of ordinary differential equations (ODEs)~\cite{murray2002mathematical}:

\begin{subequations}\label{eq:sis}
\begin{align}
&\frac{d [S]}{dt} = -\beta S I + \alpha I \\
&\frac{d [I]}{dt} = \beta S I - \alpha I
\end{align}
\end{subequations}


\subsection{SEIZ}
One drawback of the SIS model is that once a susceptible individual gets exposed to disease, he can only directly transition to infected status. In fact, especially on Twitter, this assumption does not work well; people's ideologies are complex and when they are exposed to news or rumors, they may hold different views, take time to adopt an idea, or even be skeptical to some facts. In this situation, they might be persuaded to propagate a story, or commence only after careful consideration themselves. Additionally, it is quite conceivable that an individual can be exposed to a story (i.e. received a tweet), yet never post a tweet themselves.

Based on this reasoning, we considered a more applicable, robust model, the SEIZ model which was first used to study the adoption of Feynman diagrams~\cite{powerofgoodidea:2006}. In the context of Twitter, the different compartments of the SEIZ model can be viewed as follows: Susceptible (S) represents a user who has not heard about the news yet; infected (I) denotes a user who has tweeted about the news; skeptic (Z) is a user who has heard about the news but chooses not to tweet about it; and exposed (E) represents a user who has received the news via a tweet but has taken some time, an exposure delay, prior to posting. We note that referring to the Z compartment as skeptics is in no way an implication of belief or skepticism of a news story or rumor. We adopt this terminology as this was the nomenclature used by the original authors of the SEIZ model~\cite{powerofgoodidea:2006}.

A major improvement of the SEIZ model over the SIS model is the incorporation of exposure delay. That is, an individual may be exposed to a story, but not instantaneously tweet about it. After a period of time, he may believe it and then be promoted to the infected compartment. Further, it is now possible for an individual in this model to receive a tweet, and not tweet about it themselves. As shown in Figure~\ref{fig:seiz-framework}, SEIZ rules can be summarized as follows:

\begin{table}[t] %!htp
\small
\caption{Parameter definitions in SEIZ model\cite{powerofgoodidea:2006}}
\vspace{0.5em}
\centering
\begin{tabular}{ p{4cm}  p{8cm} }
\hline
\textbf{Parameter} & \textbf{Definition} \\ [1ex]
\hline
$\beta$ & S-I contact rate \\[1ex]
b & S-Z contact rate \\[1ex]
$\rho$ & E-I contact rate \\[1ex]
$\epsilon$ & Incubation rate \\[1ex]
1/$\epsilon$ &  Average Incubation Time \\[1ex]
bl & Effective rate of S -\textgreater Z  \\ [1ex]
$\beta\rho$ & Effective rate of S -\textgreater I  \\ [1ex]
b(1-l) & Effective rate of S -\textgreater E via contact with Z  \\ [1ex]
$\beta (1-p)$ & Effective rate of S -\textgreater E via contact with I \\ [1ex]
l & S-\textgreater Z Probability given contact with skeptics \\[1ex]
1-l & S-\textgreater E Probability given contact with skeptics \\ [1ex]
p & S-\textgreater I Probability given contact with adopters \\ [1ex]
1-p & S-\textgreater E Probability given contact with adopters \\ [1ex]
\hline
\end{tabular}
\label{table:parameters}
\end{table}



\begin{itemize}
\item Skeptics recruit from the susceptible compartment with rate $b$, but these actions may result either in turning the individual into another skeptic
(with probability $l$), or it may have the unintended consequence of sending that person into the exposed (E) compartment with probability $(1 - l)$.
\item A susceptible individual will immediately believe a news story or rumor with probability $p$, or that person will move to the exposed (E) compartment with probability $(1 - p)$.
\item Transitioning of individuals from the exposed compartment to the infected class can be caused by one of two separate mechanisms: (i) an individual in the exposed class has further contact with an infected individual (with contact rate $\rho$), and this additional contact promotes him to infected; (ii) an individual in the exposed class may become infected purely by self-adoption (with rate $\epsilon$), and not from additional contact with those already infected. \end{itemize}


The SEIZ model is mathematically represented by the following system of ODEs. A slight difference of our implementation of this model is that we do not incorporate vital dynamics, which includes the rate at which individuals enter and leave the population $N$ (represented by $\mu$~\cite{powerofgoodidea:2006}). In epidemiological disease applications, this encompasses the rate at which people become susceptible (e.g. born) and deceased. In our application, a Twitter topic has a net duration not exceeding several days. Thus, the net entrance and exodus of Twitter users over these relatively short time periods is not expected to noticeably impact compartment sizes and our ultimate findings\footnote{http://www.statisticbrain.com/twitter-statistics/}.

\begin{subequations}\label{eq:seiz}
\begin{align}
\frac{d [S]}{dt} &= -\beta S \frac{I}{N} - bS \frac{Z}{N} \\
\frac{d [E]}{dt} &= (1-p)\beta S\frac{I}{N} + (1-l)bS\frac{Z}{N}-\rho E\frac{I}{N} -\epsilon E \\
\frac{d [I]}{dt} &= p\beta S \frac{I}{N} + \rho E \frac{I}{N} + \epsilon E \\
\frac{d [Z]}{dt} &= lbS \frac{Z}{N}
\end{align}
\end{subequations}


\subsection{Practical Issues}
During our adoption of the SIS and SEIZ models to understand Twitter datasets, we were constrained by several factors. The first constraint was the unknowns in the models. For example, we do not know the transition rates between the compartments nor the initial sizes of the compartments.
%The only data available is the Twitter data itself, which we used to leverage information about the infected compartment.


\begin{figure}[t]
\centering
\subfigure[SIS]{
   \includegraphics[width=2in,height=1.5in] {pictures/Boston_SIS.png}
  \label{fig:Boston_bombing_sis}
 }
 \subfigure[SEIZ]{
   \includegraphics[width=2in,height=1.5in] {pictures/Boston_SEIZ.png}
  \label{fig:Boston_bombing_seiz}
 }
\vspace{-1em}
\caption{Best fit modeling for Boston news.}
\label{fig:Boston_bombing}
\end{figure}


\begin{figure}[t]
\centering
\subfigure[SIS]{
   \includegraphics[width=2in,height=1.5in] {pictures/Pope_SIS.png}
  \label{fig:Benedict_sis}
 }
  \subfigure[SEIZ]{
   \includegraphics[width=2in,height=1.5in] {pictures/Pope_SEIZ.png}
  \label{fig:Benedict_seiz}
 }
\vspace{-1em}
\caption{Best fit modeling for Pope news.
\label{fig:Benedict}
}
\end{figure}


\begin{figure}[t]
\centering
\subfigure[SIS]{
   \includegraphics[width=2in,height=1.5in] {pictures/Amuay_SIS.png}
  \label{fig:Gas_explosion_sis}
 }
  \subfigure[SEIZ]{
   \includegraphics[width=2in,height=1.5in] {pictures/Amuay_SEIZ.png}
  \label{fig:Gas_explosion_seiz}
 }
\vspace{-1em}
\caption{Best fit modeling for Amuay news.}
\label{fig:Gas_explosion}
\end{figure}



\begin{figure}[t]
\centering
\subfigure[SIS]{
   \includegraphics[width=2in,height=1.5in] {pictures/Michelle_SIS.png}
  \label{fig:Michelle_sis}
 }
  \subfigure[SEIZ]{
   \includegraphics[width=2in,height=1.5in] {pictures/Michelle_SEIZ.png}
  \label{fig:Michelle_seiz}
 }
\vspace{-1.08em}
\caption{Best fit modeling for Michelle news.}
\label{fig:Michelle}
\end{figure}


\begin{figure}[ht]
\centering
\subfigure[SIS]{
   \includegraphics[width=2in,height=1.5in] {pictures/Obama_SIS.png}
  \label{fig:Obama_sis}
 }
 \subfigure[SEIZ]{
   \includegraphics[width=2in,height=1.5in] {pictures/Obama_SEIZ.png}
   \label{fig:Obama_seiz}
 }
\vspace{-1em}
\caption{Best fit modeling for Obama news.}
\label{fig:Obama}
\end{figure}


\begin{figure}[ht]
\centering
\subfigure[SIS]{
   \includegraphics[width=2in,height=1.5in] {pictures/Doom_SIS.png}
   \label{fig:Doomsday_sis}
 }
   \subfigure[SEIZ]{
   \includegraphics[width=2in,height=1.5in] {pictures/Doom_SEIZ.png}
   \label{fig:Doomsday_seiz}
 }
\vspace{-1em}
\caption{Best fit modeling for Doomsday rumor.}
\label{fig:Doomsday}
\end{figure}





\begin{table*}[t]
\tiny
\caption{Fitting error of SIS and SEIZ models}
\vspace{0.5em}
\centering
\begin{tabular}{| p{0.8cm}| p{1.2cm}| p{1cm} | p{1cm}| p{1.2cm} | p{1.3cm} | p{1.5cm} | p{1.2cm} | p{1.2cm}| p{1.3cm}| }
\hline
&\textbf{Boston}& \textbf{Pope} & \textbf{Amuay} & \textbf{Michelle} & \textbf{Obama} &\textbf{Doomsday} &\textbf{Castro}& \textbf{Riot}& \textbf{Average}   \\ [1ex]
\hline
\textbf{$SIS$} & 0.058 &0.041 & 0.058 & 0.088 & 0.102 & 0.028 & 0.082 & 0.088 & 0.068\\[1ex]
\hline
\textbf{$SEIZ$} & 0.010 & 0.004 &0.027 & 0.061 &0.101 & 0.029 & 0.073 & 0.093 & 0.050 \\[1ex]
\hline
\end{tabular}
\label{table:error} % is used to refer this table in the text
\end{table*}


Another constraint is the inability to quantify the total population size. This value appears to simply be the total number of Twitter accounts; however the value that we truly want is the number of individuals {\bf who could be exposed to the news or rumor topic}. This value shows to be very different from the total number of Twitter accounts. Consider the $\sim$175 million (M) registered Twitter accounts. Of these, (i) $\sim$90 million have no followers, and (ii) $\sim$56 million follow no one\footnote{http://www.businessinsider.com/chart-of-the-day-how-many-users-does-twitter-really-have-2011-3}. To further complicate the matter, there exists an abundance of
``fake'' Twitter accounts, which are never used by any real person. They are simply sold to users wishing to enhance their perceived popularity. Coupling these facts with sporadic Twitter usage due to night-time inactivity and user ``unplugging", it is clear that establishing a reliable estimate of users who could receive a tweet is quite difficult.

Synthesizing all of these factors, we assume the following in our SEIZ model implementation:
 %1) We do not have reliable population specifics:   		we do not know $N$, total population size; we do not know $S(t_0), E(t_0), I(t_0), $ or $Z(t_0)$,  the initial values of each population compartment. 2) Infected individuals ($I$) submit a tweet. 3) Skeptics ($Z$) have been exposed to story, but do not tweet. 4) Vital dynamics do not contribute to the overall population size. Thus, $N$ is a constant.

\begin{enumerate}
\item   We do not have reliable population specifics.
	\begin{enumerate}
    		\item We do not know $N$, total population size.
    		\item We do not know $S(t_0), E(t_0), I(t_0), $ or $Z(t_0)$,  the initial values of each population compartment.
  	\end{enumerate}
\item  {\bf Infected} individuals ($I$) submit a tweet.

\item  {\bf Skeptics} ($Z$) have been exposed to story, but do not tweet.

\item  Vital dynamics do not contribute to the overall population size. Thus, $N$ is a constant.
\end{enumerate}


The implication of these assumptions is that total population size N and initial population sizes for each compartment $S(t_0), E(t_0), I(t_0), $ and $Z(t_0)$ are viewed as unknowns. They are therefore treated as parameters in the parameter fit routine, and fit along with the other model parameters \cite{powerofgoodidea:2006}.

\subsection{Parameter Identification}

For each of the population models (SIS and SEIZ), represented by equation sets~\ref{eq:sis} and~\ref{eq:seiz}, we performed a nonlinear least squares fit of the model to Twitter data. As shown in Figure~\ref{fig:implementation_flow}, each step of this fitting process involved selecting a set of parameter values (rate constants and probabilities in equations \ref{eq:sis} and \ref{eq:seiz}, and initial compartment sizes), and numerically solving the system of ODEs with these parameter values. The set of parameter values that minimized $|I(t)-tweets(t)|$ was identified as the optimal parameter set.

The experimental implementation was done in Matlab. The {\bf lsqnonlin} function performed the least squares fit. The ODE systems were solved with a forward Euler function that we developed. This algorithm was selected due to its computational efficiency, and used a time-step of no more than 0.05. This threshold demonstrated to be numerically stable; in several instances we compared the forward Euler solution to those generated by Matlab's ode45 ($5^{th}$-order Explicit Runge-Kutta with embedded $4^{th}$-order error control), and observed nearly identical solutions.



\begin{figure}[t]
\centering
\subfigure[SIS]{
   \includegraphics[width=2in,height=1.5in] {pictures/Castro_SIS.png}
   \label{fig:Castro_sis}
 }
  \subfigure[SEIZ]{
   \includegraphics[width=2in,height=1.5in] {pictures/Castro_SEIZ.png}
   \label{fig:Castro_seiz}
 }
\vspace{-1.2em}
\caption{Best fit modeling for Castro rumor.}
\label{fig:Castro}
\end{figure}


\begin{figure}[t]
\centering
\subfigure[SIS]{
   \includegraphics[width=2in,height=1.5in] {pictures/Riot_SIS.png}
  \label{fig:Mexico_riot_sis}
 }
  \subfigure[SEIZ]{
   \includegraphics[width=2in,height=1.5in] {pictures/Riot_SEIZ.png}
  \label{fig:Mexico_riot_seiz}
 }
\vspace{-1.2em}
\caption{Best fit modeling for Riot news.}
\label{fig:Mexico_riot}
\end{figure}


\section{Epidemiological modeling of rumors}
Another way to study the spread of rumors (versus news) is from an epidemiological modeling standpoint. An epidemiological model helps capture the likelihood of an individual getting infected with a virus or, here, of adopting an idea that he or she has been exposed to.

\begin{figure}[t]
\centering
\includegraphics[width=2in]{pictures/SEIZ.png} %?????????????
\caption{The SEIZ compartmental model. The various states denote: (S) Susceptible. (I) Infected. (E) Exposed. (Z) Skeptic.}
\label{fig:Ebola_SEIZ_raw}
\end{figure}


\begin{figure}[t]
\centering
\subfigure[white]{
   \includegraphics[width=1.45in] {pictures/b-white_SEIZ_timecourse.png}
  \label{fig:b-white_SEIZ_timecourse}
 }
 \subfigure[zombies]{
   \includegraphics[width=1.45in] {pictures/c-zombies_SEIZ_timecourse.png}
  \label{fig:c-zombies_SEIZ_timecourse}
 }
 \subfigure[airborne]{
   \includegraphics[width=1.45in] {pictures/d-airborne_SEIZ_timecourse.png}
  \label{fig:d-airborne_SEIZ_timecourse}
 }
  \subfigure[patent]{
   \includegraphics[width=1.45in] {pictures/e-patent_SEIZ_timecourse.png}
  \label{fig:e-patent_SEIZ_timecourse}
 }
 \subfigure[white]{
   \includegraphics[width=1.45in] {pictures/Fig7-white-fitting.png}
  \label{fig:Ebola-white-fitting}
 }
 \subfigure[zombies]{
   \includegraphics[width=1.45in] {pictures/Fig7-zombies-fitting.png}
  \label{fig:Ebola-zombies-fitting}
 }
 \subfigure[airborne]{
   \includegraphics[width=1.45in] {pictures/Fig7-airborne-fitting.png}
  \label{fig:Ebola-airborne-fitting}
 }
  \subfigure[patent]{
   \includegraphics[width=1.48in] {pictures/Fig7-patent-fitting.png}
  \label{fig:Ebola-patent-fitting}
 }
\caption{Model fits of SEIZ to different rumors: (from left to right) �white�, �zombies�, �airborne�, and �patent�. (top row) Fitting results. (bottom row) time-course profiles of different compartments. }
\label{fig:Ebola_SEIZ_fitting}
\end{figure}

In earlier work~\cite{jin2013epidemiological} we demonstrated how we can accomplish this objective using the SEIZ epidemiological model that was originally proposed to study the adoption of ideas~\cite{powerofgoodidea:2006}. The SEIZ model is particularly suited to studying rumor propagation as it captures distinctions in how people respond to ideas: whether they adopt it readily or are initially skeptical.

The idea in the SEIZ model is to compartmentalize a population into four categories, denoted as S, E, I, and Z. We interpret these categories with specific reference to Twitter propagation. Susceptible (S) represents a user who has not heard the information; infected (I) denotes a user who has (re)tweeted about the information; skeptic (Z) denotes a user who has heard about the information but chooses not to (re)tweet about it; and exposed (E) represents a user who has received the information via a tweet but has taken some time, an exposure delay, prior to reposting or sharing that information.


The transitions between these states are modeled as shown in Fig~\ref{fig:Ebola_SEIZ_raw}. We caution that referring to the Z compartment as a �skeptic� is in no way an implication of the underlying truth or falsehood of the information; it simply helps capture whether users readily adopt an idea or take some time to adopt it.

Model fits of SEIZ to the different rumors and time course information for each of the state variables is given in Figure~\ref{fig:Ebola_SEIZ_fitting}. As can be seen the SEIZ model is capable of capturing a variety of information spread patterns: quasi-linear (e.g., �patent�), sigmoidal (�white�), and other non-linear patterns (�zombies� and �airborne�).


Time course results from the SEIZ compartmental model as shown in Figure~\ref{fig:Ebola_SEIZ_fitting} depict broadly similar patterns. Here �N� denotes the total size of the population (distinct Twitter users). High values of S rapidly decrease with a relatively comparable increase in Z, and a gradual increase in I that continues as E decreases. However, the patent rumor time-course data has a noticeably different response profile than the other rumor examples.

Here, the initial value of the S group begins with less than half of the total population size, and only slightly higher than the initial values of the Z and E groups. Second, the Z group is essentially constant, meaning that the number of skeptics does not change throughout the propagation time course. Third, the decrease in S does not correspond to a change in Z, as is observed in the other rumor examples. Rather, the drop in S is met with a near identical increase in E.


These findings hint that a large influx into E without a corresponding efflux to I combined with a stagnant Z group will produce an elevated response ratio. In other words, there is a large exposure to the rumor topic without significant change in skepticism.

In our earlier work on characterizing rumors~\cite{jin2013epidemiological}, we defined the notion of a �response ratio� which quantifies transitions through the exposed compartment. The response ratio provides a relative measure of the population influx into the E compartment versus the efflux from this compartment. We hypothesize that this ratio could be one of the factors useful in discriminating rumors from true news, with larger response ratios associated with factual news topics.

To compare response ratios across rumor and news, we select three breaking (true) news stories pertaining to Ebola: `Dallas' refers to the story of the first Ebola patient (Duncan) identified in the US; `NYC' refers to the first confirmation of an Ebola patient (Spencer) in New York City; and `Spencer' refers to the specific symptoms and travel activities of Spencer in the days before he was diagnosed.

\begin{figure}[ht]
\centering
\includegraphics[width=5in]{pictures/R_si_values.png} %?????????????
\caption{Ultimate response ratios for 3 news stories (left) and 10 rumors related to Ebola.}
\label{fig:Ebola_response_ratio}
\end{figure}


\begin{figure}[th]
\centering
  \includegraphics[width=5in]{pictures/real-time-RSI-value-Ebloa.png}
   \caption{Dynamic $R_{SI}$ values for Zombie rumor and Dallas real news.}
  \label{fig:dynamic_zombie_ratio}
\end{figure}



\paragraph{Ultimate $R_{SI}$ value}

The ultimate response ratios for these three news stories and other rumors are shown in Figure~\ref{fig:Ebola_response_ratio}. It can be seen that all three news stories (blue bars) have response ratios higher than 25, with a mean value of approximately 38, while eight of the 10 rumors stories (red bars) have a response ratio less than or equal to 6.4, with a mean of only 3.33. Two of the 10 rumors (green bars; `paten' and `airborne') have elevated response values, suggesting that there was greater belief associated with these topics than the other eight rumors.

\paragraph{Dynamic $R_{SI}$ value}
Figure~\ref{fig:dynamic_zombie_ratio} illustrates the dynamic response ratio of Zombie rumor and Dallas real news. We can see the response ratio for Ebola Zombie rumor swings between 0 to 4, which means people's believe extent to Zombie rumor has been very low. However, the dynamic response ratio for Dallas real news kept increasing and reached as high as nearly 100, which reveals more and more people accept this real news. 


The study here has shown that propagation of misinformation can sometimes have the same characteristics as genuine newsworthy developments.  In an age where many consumers receive their news from real-time social media platforms, it is imperative that rumors and half-truths be characterized as such and able to be distinguished from news. The tools presented here can support the quantitative evaluation of information spread as it happens.

\section{Case Study of Ebola Related Rumor}
Mark Twain is credited with saying that a lie can travel halfway around the world while the truth is putting on its shoes. As the Ebola disease rages on in West Africa, the only other epidemic being talked about is the rapid spread of misinformation on social media about the disease, its origins and impact, and response strategies. We sought to characterize the spread of both news and rumors on Twitter about the deadly disease with a view to understanding the prevalence of misinformation.

For context, although Ebola is not a new disease, the current outbreak happening in West Africa is believed to be more than three times worse than all the previous Ebola outbreaks in history combined. The three countries that have the most widespread transmission, viz. Guinea, Liberia, and Sierra Leone, are also those where public health experts fear massive under-reporting due to a variety of social considerations. Even syndromic surveillance strategies, e.g., social media mining and participatory surveillance, are not effective in these countries due to poor penetration of Internet use, and lack of roads and communication infrastructure where Ebola is most prevalent.

Social media has become one of the primary sources by which people learn about worldwide developments so it is instructive to study the spread of Ebola related information on Twitter. Most of the current chatter on Twitter about Ebola reached its peak during late Sep-mid Oct (2014) during which period there have been Ebola-related developments in the US and Europe. (In contrast, Twitter penetration in the three specific West African countries is low.)

A brief timeline of these developments will help in the discussion that follows. On September 30, 2014, the CDC confirmed the first importation of Ebola into the United States when Thomas Eric Duncan traveled from Liberia to visit family in Dallas. On October 6, in Madrid, Spain, Teresa Romero, a nurse, was reported to be the first person to have contracted the disease outside of West Africa.

On October 8, back in the US, Duncan succumbed to Ebola. A few days later, a healthcare worker at Texas Presbyterian Hospital in Dallas who provided care for Duncan, tested positive for Ebola. On the morning of Oct. 14, a second healthcare worker, who also provided care for Duncan, reported to the hospital with a low-grade fever and was isolated. This healthcare worker also tested positive for Ebola subsequently.

Many states and cities began making contingency plans and issuing travel advisories and guidelines. Lawmakers called for screening passengers and proposed travel bans for Ebola-stricken countries. On October 23, Craig Spencer, a doctor returning from work in Guinea, was rushed to Bellevue Hospital Center with a 100.3 fever and became New York City's first Ebola patient.


\subsection{Rumors on Twitter}
\begin{figure}[h]
\centering
\subfigure[09/29/14]{
   \includegraphics[width=2in,height=1.1in] {pictures/Ebola/Word_cloud_0929.png}
  \label{fig:worldcloud29}
 }
 \subfigure[09/30/14]{
   \includegraphics[width=2in,height=1.1in] {pictures/Ebola/Word_cloud_0930.png}
  \label{fig:worldcloud30}
 }
 \subfigure[10/01/14]{
   \includegraphics[width=2in,height=1.1in] {pictures/Ebola/Word_cloud_1001.png}
  \label{fig:worldcloud01}
 }
\caption{Word clouds constructed from Ebola-related tweets.}
\label{fig:Ebola_word_cloud}
\end{figure}

The period from end of Sep to mid-late Oct, when Ebola activity happened in the US, is also the period when conspiracy theories, innuendo and rumors began to propagate wildly on Twitter. We gathered tweets during this period and filtered them by either the mention of the keyword `ebola' or relevant hashtags such as \#ebola, \#EbolaVirus, \#EbolaOutbreak, \#EbolaWatch, \#EbolaEthics, \#EbolaChat, \#nursesfightebola, \#ebolafacts, \#StopEbola, \#FightingEbola, and \#UHCRevolution.

From the gathered tweets, we removed stopwords for further processing. Figure~\ref{fig:Ebola_word_cloud} depicts word clouds constructed from the tweets for specific days. As can be seen, on 2014-09-29, when there was no Ebola incident in the US, people's attention were primarily focused on Liberia and other African countries. On 2014-09-30, after CDC confirmed that Mr. Duncan in Dallas tested positive for Ebola, related keywords rose to the fore.


%Figure~\ref{fig:Ebola_time_series} depicts a simple frequency plot of specific keywords in Ebola-related tweets highlighting significant upticks for keywords such as Dallas, USA, CDC as well as �enfermera� (referring to the Spanish Nurse Romero and possibly other health professionals) during the studied period. Into mid October, we notice upticks in mentions of President Obama, likely due to proposed mitigation and response strategies by the US Government.
%
%
%\begin{figure}[ht]
%\centering
%\includegraphics[width=5in]{pictures/time_series.png} %?????????????
%\caption{Frequency plot of specific keywords in Ebola-related tweets from 09/20/2014 to 10/18/2014}
%\label{fig:Ebola_time_series}
%\end{figure}

\begin{figure}[h]
\centering
\includegraphics[width=6.5in, height=2.3in]{pictures/Ebola/10_rumors_1.png} %?????????????
\caption{Top 10 Ebola-related rumors (by volume; from 09/28/2014 to 10/18/2014).}
\label{fig:Ebola_10rumors}
\end{figure}

\begin{figure}[h]
\centering
\includegraphics[width=5in]{pictures/Ebola/20141008-map-country1.png} %?????????????
\caption{Distribution of top-10 rumors obtained from geolocated tweets. Data from 10/08/2014 is used for this plot. Icon size is proportional to the logarithm of the tweet volume.}
\label{fig:Ebola_map}
\end{figure}


Next, we studied information cascades in our collection of tweets with a view toward identifying misinformation and spread of falsehoods. We identified several widespread rumors circulating on Twitter, the top 10 of which are shown in Figure~\ref{fig:Ebola_10rumors}. (We focus on rumors in English only for our study.) Most are self-explanatory as to their intent and interpretation.


Two other rumors are note worthy. The `snake' rumor (which originated at least as early as late summer 2014) asserts that Ebola came across the border from Guinea to Sierra Leone via a snake in a bag. As stated in [4], ``a lady had a snake in a bag. When somebody opened the bag, that made the lady die.'' The Maldives rumor pertains to an uncorroborated report that Ebola patients have been reported (and quarantined) in the Maldives.

For each of these rumors, we geocoded tweets participating in the spread of such rumors with a view to understanding their geographical scope. As Figure~\ref{fig:Ebola_map} shows, the ``airborne'' and ``inject'' rumors were most prevalent in the US with specific other rumors (e.g., ``patent'') being prevalent in other parts of the world.



\begin{figure}[h]
\centering
\subfigure[09-29-2014]{
   \includegraphics[width=3in] {pictures/Ebola/rumor-cluster-20140929.png}
  \label{fig:worldcloud29}
 }
 \subfigure[10-06-2014]{
   \includegraphics[width=3in] {pictures/Ebola/rumor-cluster-20141006.png}
  \label{fig:worldcloud30}
 }
\caption{How rumors cluster: (a) 09/29/2014. (b) 10/06/2014. Rumors are color coded consistently across the two frames.}
\label{fig:Ebola_cluster}
\end{figure}

Next, we employed a dynamic query expansion model~\cite{zhao2014unsupervised} to study the rumors in greater detail. The DQE model begins with a seed set of keywords (e.g., ``ebola'', ``rumor''), identifies tweets that mentions these keywords, and iteratively expands them into a larger set of keywords. By conducting a modularity-based optimization over the underlying network of expanded tweets connected by shared keywords, DQE can identify specific localized instantiations of rumors.

As shown in Figure~\ref{fig:Ebola_cluster}, on 09/29/2014 (when there was no incidence of Ebola in the US), the dominant rumor is the zombie rumor. By 10/06/2014, other rumors pertaining to how Ebola can be airborne and that it is a potential terrorist weapon gained hold.



Although Figure~\ref{fig:Ebola_cluster} might suggest that rumors are quite rampant, it is important to keep in perspective that they are but a small fraction of all information propagation related to Twitter. Figure~\ref{fig:Ebola_patent_propagation} and~\ref{fig:Ebola_news_propagation} compare the time-indexed spread of the `patent' rumor versus a true news story (about the first US incidence of Ebola in Dallas). Here different colors denote different communities participating in information propagation, not different rumors/news stories. Each node in these graphs denotes a Twitter user, and an edge between nodes denotes a reply or retweet relationship. As can be seen, news stories permeate better whereas rumors are more localized, distributed, and comparatively smaller in permeation.

\begin{figure}[h]
\centering
\subfigure[09-30-2014]{
   \includegraphics[width=2in] {pictures/Ebola/patent_20140930.png}
  \label{fig:patent_20140930}
 }
 \subfigure[09-30-2014]{
   \includegraphics[width=2in] {pictures/Ebola/patent_20141001.png}
  \label{fig:patent_20141001}
 }
 \subfigure[10-01-2014]{
   \includegraphics[width=2in] {pictures/Ebola/patent_20141002.png}
  \label{fig:patent_20141002}
 }
\caption{Ebola-related patent rumor propagation over time.}
\label{fig:Ebola_patent_propagation}
\end{figure}



\begin{figure}[h]
\centering
\subfigure[09-30-2014]{
   \includegraphics[width=2in] {pictures/Ebola/dallas_20140930.png}
  \label{fig:dallas_20140930}
 }
 \subfigure[09-30-2014]{
   \includegraphics[width=2in] {pictures/Ebola/dallas_20141001.png}
  \label{fig:dallas_20141001}
 }
 \subfigure[10-01-2014]{
   \includegraphics[width=2in] {pictures/Ebola/dallas_20141002.png}
  \label{fig:dallas_20141002}
 }
\caption{Ebola-related Dallas news propagation over time.}
\label{fig:Ebola_news_propagation}
\end{figure}


\subsection{Epidemiological Modeling of Rumors}
Another way to study the spread of rumors (versus news) is from an epidemiological modeling standpoint. An epidemiological model helps capture the likelihood of an individual getting infected with a virus or, here, of adopting an idea that he or she has been exposed to.

\begin{figure}[h]
\centering
\includegraphics[width=2in]{pictures/Ebola/SEIZ.png} %?????????????
\caption{The SEIZ compartmental model. The various states denote: (S) Susceptible. (I) Infected. (E) Exposed. (Z) Skeptic.}
\label{fig:Ebola_SEIZ_raw}
\end{figure}


\begin{figure}[h]
\centering
\subfigure[white]{
   \includegraphics[width=1.45in] {pictures/Ebola/b-white_SEIZ_timecourse.png}
  \label{fig:b-white_SEIZ_timecourse}
 }
 \subfigure[zombies]{
   \includegraphics[width=1.45in] {pictures/Ebola/c-zombies_SEIZ_timecourse.png}
  \label{fig:c-zombies_SEIZ_timecourse}
 }
 \subfigure[airborne]{
   \includegraphics[width=1.45in] {pictures/Ebola/d-airborne_SEIZ_timecourse.png}
  \label{fig:d-airborne_SEIZ_timecourse}
 }
  \subfigure[patent]{
   \includegraphics[width=1.45in] {pictures/Ebola/e-patent_SEIZ_timecourse.png}
  \label{fig:e-patent_SEIZ_timecourse}
 }
 \subfigure[white]{
   \includegraphics[width=1.45in] {pictures/Ebola/Fig7-white-fitting.png}
  \label{fig:Ebola-white-fitting}
 }
 \subfigure[zombies]{
   \includegraphics[width=1.45in] {pictures/Ebola/Fig7-zombies-fitting.png}
  \label{fig:Ebola-zombies-fitting}
 }
 \subfigure[airborne]{
   \includegraphics[width=1.45in] {pictures/Ebola/Fig7-airborne-fitting.png}
  \label{fig:Ebola-airborne-fitting}
 }
  \subfigure[patent]{
   \includegraphics[width=1.48in] {pictures/Ebola/Fig7-patent-fitting.png}
  \label{fig:Ebola-patent-fitting}
 }
\caption{Model fits of SEIZ to different rumors: (from left to right) `white', `zombies', `airborne', and `patent'. (top row) Fitting results. (bottom row) time-course profiles of different compartments. }
\label{fig:Ebola_SEIZ_fitting}
\end{figure}

In earlier work~\cite{jin2013epidemiological} we demonstrated how we can accomplish this objective using the SEIZ epidemiological model that was originally proposed to study the adoption of ideas~\cite{powerofgoodidea:2006}. The SEIZ model is particularly suited to studying rumor propagation as it captures distinctions in how people respond to ideas: whether they adopt it readily or are initially skeptical.

The idea in the SEIZ model is to compartmentalize a population into four categories, denoted as S, E, I, and Z. We interpret these categories with specific reference to Twitter propagation. Susceptible (S) represents a user who has not heard the information; infected (I) denotes a user who has (re)tweeted about the information; skeptic (Z) denotes a user who has heard about the information but chooses not to (re)tweet about it; and exposed (E) represents a user who has received the information via a tweet but has taken some time, an exposure delay, prior to reposting or sharing that information.


The transitions between these states are modeled as shown in Fig~\ref{fig:Ebola_SEIZ_raw}. We caution that referring to the Z compartment as a ``skeptic'' is in no way an implication of the underlying truth or falsehood of the information; it simply helps capture whether users readily adopt an idea or take some time to adopt it.

Model fits of SEIZ to the different rumors and time course information for each of the state variables is given in Figure~\ref{fig:Ebola_SEIZ_fitting}. As can be seen the SEIZ model is capable of capturing a variety of information spread patterns: quasi-linear (e.g., `patent'), sigmoidal (`white'), and other non-linear patterns (`zombies' and `airborne').


Time course results from the SEIZ compartmental model as shown in Figure~\ref{fig:Ebola_SEIZ_fitting} depict broadly similar patterns. Here `N' denotes the total size of the population (distinct Twitter users). High values of S rapidly decrease with a relatively comparable increase in Z, and a gradual increase in I that continues as E decreases. However, the patent rumor time-course data has a noticeably different response profile than the other rumor examples.

Here, the initial value of the S group begins with less than half of the total population size, and only slightly higher than the initial values of the Z and E groups. Second, the Z group is essentially constant, meaning that the number of skeptics does not change throughout the propagation time course. Third, the decrease in S does not correspond to a change in Z, as is observed in the other rumor examples. Rather, the drop in S is met with a near identical increase in E.


These findings hint that a large influx into E without a corresponding efflux to I combined with a stagnant Z group will produce an elevated response ratio. In other words, there is a large exposure to the rumor topic without significant change in skepticism.

In our earlier work on characterizing rumors~\cite{jin2013epidemiological}, we defined the notion of a `response ratio' which quantifies transitions through the exposed compartment. The response ratio provides a relative measure of the population influx into the E compartment versus the efflux from this compartment. We hypothesize that this ratio could be one of the factors useful in discriminating rumors from true news, with larger response ratios associated with factual news topics.

To compare response ratios across rumor and news, we select three breaking (true) news stories pertaining to Ebola: `Dallas' refers to the story of the first Ebola patient (Duncan) identified in the US; `NYC' refers to the first confirmation of an Ebola patient (Spencer) in New York City; and `Spencer' refers to the specific symptoms and travel activities of Spencer in the days before he was diagnosed.

\begin{figure}[h]
\centering
\includegraphics[width=5in]{pictures/Ebola/R_si_values.png} %?????????????
\caption{Ultimate response ratios for 3 news stories (left) and 10 rumors related to Ebola.}
\label{fig:Ebola_response_ratio}
\end{figure}


\begin{figure}[h]
\centering
  \includegraphics[width=5in]{pictures/Ebola/real-time-RSI-value-Ebloa.png}
   \caption{Dynamic $R_{SI}$ values for Zombie rumor and Dallas real news.}
  \label{fig:dynamic_zombie_ratio}
\end{figure}



\paragraph{Ultimate $R_{SI}$ value}

The ultimate response ratios for these three news stories and other rumors are shown in Figure~\ref{fig:Ebola_response_ratio}. It can be seen that all three news stories (blue bars) have response ratios higher than 25, with a mean value of approximately 38, while eight of the 10 rumors stories (red bars) have a response ratio less than or equal to 6.4, with a mean of only 3.33. Two of the 10 rumors (green bars; `paten' and `airborne') have elevated response values, suggesting that there was greater belief associated with these topics than the other eight rumors.

\paragraph{Dynamic $R_{SI}$ value}
Figure~\ref{fig:dynamic_zombie_ratio} illustrates the dynamic response ratio of Zombie rumor and Dallas real news. We can see the response ratio for Ebola Zombie rumor swings between 0 to 4, which means people's believe extent to Zombie rumor has been very low. However, the dynamic response ratio for Dallas real news kept increasing and reached as high as nearly 100, which reveals more and more people accept this real news.


The study here has shown that propagation of misinformation can sometimes have the same characteristics as genuine newsworthy developments.  In an age where many consumers receive their news from real-time social media platforms, it is imperative that rumors and half-truths be characterized as such and able to be distinguished from news. The tools presented here can support the quantitative evaluation of information spread as it happens.

\section{Conclusion}
In this chapter, we have demonstrated how true news and rumor stories being propagated over Twitter can be modelled by epidemiologically-based population models. We have shown that the SEIZ model, in particular, is accurate in capturing the information spread of a variety of news and rumor topics, thereby generating a wealth of valuable parameters to facilitate the analysis of these events. We then demonstrated how these parameters can also be incorporated into a strategy for supporting the identification of Twitter topics as rumor or news. As of now, we are modeling propagation over static data. In future, we plan to adapt this model for capturing news and rumors in real-time.

%In this chapter, we have quantified information propagation, including both news or rumors,
%on Twitter using epidemic models.
%We have demonstrated that the SEIZ model is accurate in capturing
%information diffusion, especially in the initial portion of the curve.
%Using the SEIZ model, we have developed a method to detect rumors.
%In the future, we plan to incorporate more detailed follower information in
%our models, to determine exactly how many twitter users receive
%tweets from a sender.
%Such data can be used to determine SEIZ transition rates more accurately.
%Another interesting approach that we plan to explore in the  future
%is to use sentiment analysis on tweets to analyze whether an individual is
%a skeptic or infected or just exposed. Finally, we would like to develop
%an automated method to separate rumors from truth using a multiplicity of features.





\endgroup
