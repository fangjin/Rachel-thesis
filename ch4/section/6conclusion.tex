\section{Discussion}
\label{sec:conclusion}
Previous research has demonstrated the importance of burst detection in Twitter. In this study, we argue that group absenteeism can also be vital for detecting disruptive societal events. Modeling absenteeism is crucial because it can serve as a surrogate signal for event detection. For example, in the case of the Iquique earthquake, our new algorithm detected absenteeism behavior on Twitter that was closely followed by a spike in user activity. Unlike traditional event detection methods, which identify real time events only after they have occurred because the burst signal must first be identified, an absenteeism signal can be observed much earlier, thus providing greater foresight into future events. This means that our proposed approach offers a significant advantage over current strategies that focus solely on modeling spike or burst related patterns for event detection.

%Disruptive events which cause Twitter absenteeism, but also render burst detection methods less useful. In the case of the Natal protest event, a large portion of people were walking on the street to protest, and the city's tweet absenteeism score reached a minimum. During the Brazil floods, the tweets tended to become inactive as the severity of the floods increased. It reached the lowest point when the flood was at its worst. In these two cases in particular, using a burst signal alone it can be difficult to identify such events.

Existing approaches for event detection also suffer from an inherent latency in their detection process. This is because they are based on the use of bursty signals from abnormal activity on social networks, but miss the absenteeism signal that often precedes these bursts. Our approach addresses this shortcoming by successfully modeling the `lull before the storm'. In this study we defined an absenteeism score for groups of cities within the Twitter network and apply it to construct wavelet transforms that not only
detect anomalous subgraphs (including both burst and absenteeism groups) at different scales, but can also be used to identify the geographical focal point of the anomaly. This localization property of graph wavelets guarantees that the selected groups are compact automatically. The identified abnormal groups have been verified using real-world datasets and proven to be indicative of
events such as civil protests or natural disasters.

%In future work, we plan to extend our detection model to capture extent of an event's influence over network. Another interesting extension of our work would be to include absenteeism as feature to classify event of different nature (disruptive vs non-disruptive).
