\section{Conclusion}
\label{sec:conclusion}
Existing approaches for event detection suffer from an inherent latency in their detection process. It is because they use the bursty signals from abnormal activity on social networks, but miss the absenteeism signal that precedes these bursts. Our approach bridges this shortcoming by successfully modeling this \textit{lull-ness}. We have presented a systematic and unified framework for detecting, identifying event's location and distinguishing anomalous groups in Twitter. From the three case studies, we have shown that the initial phase in the evolution of an disruptive, event is characterized by group absenteeism behavior. This behavior is further underlined by an increase in user mobility. As in the case of ``Christmas Day" event we observed absenteeism from Argentina Twitter users in days leading to December 25th was characterized by increased mobility (inferred from geolocated tweets). We defined an absenteeism score over the groups of cities that form our Twitter network and used it construct wavelet transforms, that not only to detect the anomalous subgraphs at different scales, but also to find the geographical focal point of the anomaly.

%In future work, we plan to extend our detection model to capture extent of an event's influence over network. Another interesting extension of our work would be to include absenteeism as feature to classify event of different nature (disruptive vs non-disruptive).
