\section{Related Work} \label{sec:related}

Anomaly detection in Graphs has been well studied using outlier detection~\cite{akoglu2009anomaly}. When considering group concept, two directions has been studied~\cite{akoglu2015graph}: one is anomalies in unlabeled/plain graphs~\cite{noble2003graph}, the other is in attributed graphs. In the plain graph anomaly detection, since the only given information is its structure, various features such as distances, communities~\cite{sun2005neighborhood} have been employed to define graph anomaly. In work of~\cite{henderson2010metric}, more metrics like vertices, edges, degree, weight, connected components are incorporated into detection framework. In attributed graphs, features regarding nodes behaviors make it possible to have a richer graph representation, which is usually tied with some real-world applications. Such as \cite{yu2014glad} defines the group based on the term of role, and model the normal groups follow the same pattern with respect to their role mixture rates. In our work, we consider both graph structure and nodes features, propose a graph wavelet based approach for group anomaly detection, which can guarantee the detected group to be automatically compact, with linear computation complexity and scalability.

Event detection based on LSBNs is a research area that has attracted significant attention in the last years. Traditional approaches focus on capturing spatiotemporal burstiness of keywords~\cite{lappas2009burstiness,lappas2012spatiotemporal}, Kalman filtering to track the geographical trajectories of hot spots of tweets related to earthquakes~\cite{sakaki2010earthquake}; detecting topics of interest that are coherent in geographic regions~\cite{eisenstein2010latent,hong2012discovering,yin2011geographical}; applying clustering-based approaches search for emerging clusters of documents or terms using predefined similarity metrics that consider factors such as term co-occurrences and social interactions~\cite{aggarwal2012event,sayyadi2009event,watanabe2011jasmine,weng2011event}; and using the notion of compactness of a graph~\cite{rozenshtein2014event} to detect events. Several statistical methods have also been used, based on Kulldroff’s spatial scan statistic~\cite{kulldorff1997spatial}, to detect spatial outliers~\cite{chen2008detecting} and have been applied to a wide variety of domains including transportation networks, civil unrest forecasting~\cite{zhao2014unsupervised}, and heterogeneous social media graphs~\cite{chen2014non}.

Our approach to event detection problem is conceptually different from above mentioned studies. It includes a graph-theoretic framework to detect absenteeism related anomalies and correlate them with future events. Although group absence behavior has been widely studied in the area of organizational behavioral studies~\cite{gaudine2001effects,seamonds1982stress}, it remains unexplored in the area of social network analysis. Resembling closely to group anomaly detection in complex networks, our detection approach is further distinguished by its focus on groups rather than individuals. Existing approaches to group anomaly detection include building generative models of group anomalies~\cite{xiong2011hierarchical,yu2014glad} where the goal is to automatically infer the groups and detect group anomalies in a social network. Typical to mixture models such methods suffer from high computational complexity due to the size of data and are heavily parameterized.

One of key challenges of our research problem is adapting the detection procedure for both missing and bursty activity groups. For this purpose, we incorporate spectral graph wavelets~\cite{hammond2011wavelets} into our algorithm. This strategy has been quite effectively used in multiscale community mining~\cite{tremblay2014graph}.
Wavelet methods based on spectral graph theory have been applied in a wide array data mining areas such as community detection, anomaly detection~\cite{calderara2011detecting} and other machine learning tasks~\cite{shuman_ACHA_2013,ghosh2003wavelet,rustamov2013wavelets,2000wavecluster}. By constructing wavelets over graphs we are able take advantage of local information encoded in graph structure and then cluster and identify nodes which are similar in a scale-dependent fashion.

% The relevant methods can be classified into three categories: burst detection, geographical topic modeling, and clustering. Burst detection methods search for space-time regions that %have abnormally high counts of some predefined terms~\cite{lappas2009burstiness,lappas2012spatiotemporal}. Sakaki et al. consider spatial-temporal Kalman filtering to track the %geographical trajectories of hot spots of Tweets related to earthquakes~\cite{sakaki2010earthquake}. Geographic topic modeling based methods detect topics of interest that are coherent %in geographic regions~\cite{eisenstein2010latent,hong2012discovering,yin2011geographical}. Clustering-based approaches search for emerging clusters of documents or terms using %predefined similarity metrics that consider factors such as term co-occurrences and social %interactions~\cite{aggarwal2012event,sayyadi2009event,teitler2008newsstand,watanabe2011jasmine,weng2011event}.
